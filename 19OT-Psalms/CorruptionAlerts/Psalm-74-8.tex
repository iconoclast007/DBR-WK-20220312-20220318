\begin{center}

\begin{table}[ht]
\centering
\begin{tabular}{|p{.5in}|p{3.5in}|}
\hline

\textcolor[rgb]{0.00,0.00,1.00}{AV} & \textcolor[rgb]{0.00,0.00,1.00}{They said in their hearts, Let us destroy them together: they have burned up all the synagogues of God in the land.} \\ \hline 

\hline
\hline


ASV &  They said in their heart, Let us make havoc of them altogether: They have burned up all the synagogues of God in the land. (footnote has: ``places of assembly.'' \\ \hline
CEB &  They said in their hearts, We’ll kill all of them together! They burned all of God’s meeting places in the land.\\ \hline
ESV & They said to themselves, ``We will utterly subdue them''; they burned all the meeting places of God in the land. \\ \hline
NASV &  They said in their heart, ``Let us completely subdue them.'' They have burned all the meeting places of God in the land. \\ \hline
MEV & They said in their hearts, ``Let us destroy them together.''     They have burned up all the meeting places of God in the land.\\ \hline
NIV &  They said in their hearts, ``Let us destroy them together.''     They have burned up all the meeting places of God in the land.\\ \hline
NKJV &  They said in their hearts, ``Let us destroy them altogether'' They have burned up all the meeting places of God in the land. \\ \hline
RSV &  They said to themselves, ``We will utterly subdue them'';  they burned all the meeting places of God in the land.\\ \hline

\hline
\hline

\multicolumn{2}{|p{4.3in}|}{{\textcolor{jungle}{Removal of the word ``synagogues'' from the verse support an Amillennial or Post-Millennial reading, devoid of any Jewish presence. The is also fueled by an entirely historical interpretation of Psalm 74, where the destruction is taken to refer to the first Temple destruction. This word change is usually accompanied by an excision of the word ``congregations'' in verse 4, which speaks of people instead of a place. Commentators (such as Ryrie) claim that ``synagogues'' cannot be correct as these develop later in history, although the word choice is perfectly consistent with the Tribulation context of the rest of the psalm.}}} \\ \hline

\end{tabular}
\caption[Corruption Alert: Psalm 74:8]{Corruption Alert: Psalm 74:8} \label{table:Corruption Psalm 74:8}

\end{table}

\end{center}




\subsection{Psalm 74:9}
The signs that the Jews seek do not cease until the Jews had completely rejected the kingdom offer in the Book of Acts. 

\subsection{Psalm 74:10}
Suggestions for specific identity of ``the adversary'' are provided in scripture. See, for example, Lamentations 4:12 (tribulation content), Matthew 5:25 (Tribulation content), 1 Timothy 5:14, 1 Peter 5:8.\footnote{\textbf{Lamentations 4:12} - The kings of the earth, and all the inhabitants of the world, would not have believed that the adversary and the enemy should have entered into the gates of Jerusalem.}\footnote{\textbf{Matthew 5:25} - Agree with thine adversary quickly, whiles thou art in the way with him; lest at any time the adversary deliver thee to the judge, and the judge deliver thee to the officer, and thou be cast into prison.}\footnote{\textbf{1 Timothy 5:14} - I will therefore that the younger women marry, bear children, guide the house, give none occasion to the adversary to speak reproachfully.}\footnote{\textbf{1 Peter 5:8} - I will therefore that the younger women marry, bear children, guide the house, give none occasion to the adversary to speak reproachfully.}
