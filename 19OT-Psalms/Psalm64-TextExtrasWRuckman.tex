\chapter{Psalm 64}
\footnote{\textcolor[rgb]{0.00,0.25,0.00}{\hyperlink{TOC}{Return to end of Table of Contents.}}}\textcolor[rgb]{0.00,0.00,1.00}{To the chief Musician, A Psalm of David.}\\
\\
\textcolor[rgb]{0.00,0.00,1.00}{Hear my voice, O God, in my prayer: preserve my life from fear of the enemy.}\footnote{Again the Psalm is devotional, in the main, and can find application to any saint in any dispensation. However ``the perfect'' in verse 4 messes up the applications to the Church Age, and the ``arrow'' in verse 7 is aimed at Ahab--one of the greatest types of the Son of Perdition in the Bible (see 1 Kings 22:34; Psalm 18:14). Still, most of the Psalm can be applied spiritually. ``From fear of the enemy'' is the original of FDR’s famous boo-boo: we have nothing to fear ``but fear itself,'' which, of course, is muddled madness. David has a specific fear to be delivered from: ``fear of the enemy.'' This  is an unreasonable fear according to verse 9. It is a demoralizing fear, and if it is a fear greater than the fear of God, it is destructive (see Psalm 34:11, 103:13, 111:10; Proverbs 29:25; and Matthew 10:26--28). \cite{Ruckman1992Psalms}}
[2] \textcolor[rgb]{0.00,0.00,1.00}{Hide me from the secret counsel of the wicked; from the insurrection of the workers of iniquity:}\footnote{``Hide me from the secret counsel'' (vs. 2) like Peter was hidden in Acts 12:17, like Paul was hidden in Acts 9:25, like Elijah was hidden in 1 Kings 17:3, and like Joash and Jesus were hidden in 2 Kings 11:2 and Matthew 2:13. The ``insurrection''” is led by a man like Barabbas (see Mark 15:7) or the
``demonstrators'' of Acts 18:12. ``The insurrection'' would be the one under the Man of Sin (see Dan. 11:21--2 and Psalm 2:1--3). \cite{Ruckman1992Psalms}}
[3] \textcolor[rgb]{0.00,0.00,1.00}{Who whet their tongue like a sword, \emph{and} bend \emph{their} \emph{bows} \emph{to} \emph{shoot} their arrows, \emph{even} bitter words:}
[4] \textcolor[rgb]{0.00,0.00,1.00}{That they may shoot in secret at the perfect: suddenly do they shoot at him, and fear not.}
[5] \textcolor[rgb]{0.00,0.00,1.00}{They encourage themselves \emph{in} an evil matter: they commune of laying snares privily; they say, Who shall see them?}
[6] \textcolor[rgb]{0.00,0.00,1.00}{They search out iniquities; they accomplish a diligent search: both the inward \emph{thought} of every one \emph{of} \emph{them}, and the heart, \emph{is} deep.}\footnote{Bitter words are like “arrows”: they are aimed; they have force behind them, and they can wound or kill. (Compare verse 4 with verse 7). Before, these words were like a “razor” (Ps. 52:2); now they are likened to swords and arrows. Slander and innuendos wound in “hand-to-hand” close-ups (the razor), at arm’s length (the sword), and at a distance (the arrows). The razor is stropped; the sword is whetted, and the bow is bent. This means the liar and the slanderers have to “encourage themselves in an evil matter” before they wound or kill; this is done in secret (vs. 5) after a “diligent search” (vs. 6) into evil. They are “indepth” thinkers (“the inward thought of every one of them, and the heart, is deep”). They commune, encourage, search diligently, and then counsel. Biblical examples are \cite{Ruckman1992Psalms}:
\begin{compactenum}
\item the plotting of the Sanhedrin against Christ (John 11:47--48; Matt. 26:59--66), 
\item the plotting of the Jews to kill Paul (Acts 23:12), 
\item the plotting of Jezebel to get Naboth killed (1 Kings 21:8--10), 
\item the plotting of the Chaldeans to get Daniel killed (Dan. 6:4--5), 
\item the plotting of Joseph’s brethren to kill him (Gen. 37:20), 
\item  the plotting of Ahithophel and Absalom to get David (2 Sam. 15:1--6).  
\end{compactenum}
Historical examples number into the hundreds: 
\begin{compactenum}
\item  the Catholic plot in the 1600s to murder 40,000 Bible-believing people in Northern Ireland; 
\item  the Catholic plot to murder another 50,000 on Saint Bartholomew’s Day in France; 
\item  the Catholic plot to bomb the British Parliament in 1605; 
\item  the Catholic plot to attack England with an Armada and murder Queen Elizabeth; 
\item  the Catholic plot to start World War I by getting an “incident” to take place in Serbia; 
\item  the Catholic plot to start World War II by aligning the Vatican with three dictators in Europe; 
\item  the Catholic plot to get the U.S.A. into Vietnam to back up the Diems; 
\item  the work of the NEA in America to convert the public schools into a Federal jungle filled with dopeheads, sex perverts, atheists, and Communists—completely contrary to all of the goals, intentions, purposes and ideals of the founding fathers; 
\item  the works of the Illuminati, the New Agers, the International Bankers, the Federal Reserve System, the CFR, the Bilderbegers, etc. “They accomplish a diligent search.” 
\end{compactenum} }
[7] \textcolor[rgb]{0.00,0.00,1.00}{But God shall shoot at them \emph{with} an arrow; suddenly shall they be wounded.}\footnote{It can be spiritualized, so the ``arrow'' is the arrow of God’s word but the reality is found in Habakkuk 3:11 and Psalms 18:14, 7:13. Ahab is the prototype; see also Psalm 7:12. ``So shall they make their own tongue  to fall upon themselves.''  The idea is ``I shot an arrow into air, it fell to earth $\hdots$ on my own pate'' (see Psalm 7:16). What is left of the Devil's crowd at the end of the Tribulation will run for their lives (see Ahab's crowd in 1 Kings 22:17, 36). They are described in Revelation 6:15--17. \cite{Ruckman1992Psalms}} 
[8] \textcolor[rgb]{0.00,0.00,1.00}{So they shall make their own tongue to fall upon themselves: all that see them shall flee away.}
[9] \textcolor[rgb]{0.00,0.00,1.00}{And all men shall fear, and shall declare the work of God; for they shall wisely consider of his doing.}\footnote{Verse 9 is the doctrinal mate to Exodus 14:31, which symbolizes the overthrow of the “white horse” rider (Exod. 15:2, 19) of Revelation 6:2. It is a Second Advent passage. All of the commentators miss it. \cite{Ruckman1992Psalms}}
[10] \textcolor[rgb]{0.00,0.00,1.00}{The righteous shall be glad in the LORD, and shall trust in him; and all the upright in heart shall glory.}\footnote{Verse 10 is about “the righteous,” a term that never applies one time to any Christian in the Body of Christ. “The righteous” are defined by the Holy Spirit in the infallible English text of Matthew 25:37 to be Tribulation saints who helped the Jews.  First Peter 4:18 is a citation of Proverbs 11:31. Matthew 13:43 is a companion passage, as is Deuteronomy 32:43. Note the ``shall trust him'' is in the future, after the man is already ``righteous.'' Did you get that? The Millennium follows the Tribulation. Did you get that? \cite{Ruckman1992Psalms}}
