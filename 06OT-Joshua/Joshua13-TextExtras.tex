\chapter{Joshua 13}
\footnote{\textcolor[rgb]{0.00,0.25,0.00}{\hyperlink{TOC}{Return to end of Table of Contents.}}}\textcolor[rgb]{0.00,0.00,1.00}{Now Joshua was old \emph{and} stricken in years; and the LORD said unto him, Thou art old \emph{and} stricken in years, and there remaineth yet very much land to be possessed.}\index[AWIP]{Now!Joshua!Jsh 13:001}\index[AWIP]{Joshua!Joshua!Jsh 13:001}\index[AWIP]{was!Joshua!Jsh 13:001}\index[AWIP]{old!Joshua!Jsh 13:001}\index[AWIP]{\emph{and}!Joshua!Jsh 13:001}\index[AWIP]{stricken!Joshua!Jsh 13:001}\index[AWIP]{in!Joshua!Jsh 13:001}\index[AWIP]{years!Joshua!Jsh 13:001}\index[AWIP]{and!Joshua!Jsh 13:001}\index[AWIP]{the!Joshua!Jsh 13:001}\index[AWIP]{LORD!Joshua!Jsh 13:001}\index[AWIP]{said!Joshua!Jsh 13:001}\index[AWIP]{unto!Joshua!Jsh 13:001}\index[AWIP]{him!Joshua!Jsh 13:001}\index[AWIP]{Thou!Joshua!Jsh 13:001}\index[AWIP]{art!Joshua!Jsh 13:001}\index[AWIP]{old!Joshua!Jsh 13:001 (2)}\index[AWIP]{\emph{and}!Joshua!Jsh 13:001 (2)}\index[AWIP]{stricken!Joshua!Jsh 13:001 (2)}\index[AWIP]{in!Joshua!Jsh 13:001 (2)}\index[AWIP]{years!Joshua!Jsh 13:001 (2)}\index[AWIP]{and!Joshua!Jsh 13:001 (2)}\index[AWIP]{there!Joshua!Jsh 13:001}\index[AWIP]{remaineth!Joshua!Jsh 13:001}\index[AWIP]{yet!Joshua!Jsh 13:001}\index[AWIP]{very!Joshua!Jsh 13:001}\index[AWIP]{much!Joshua!Jsh 13:001}\index[AWIP]{land!Joshua!Jsh 13:001}\index[AWIP]{to!Joshua!Jsh 13:001}\index[AWIP]{be!Joshua!Jsh 13:001}\index[AWIP]{possessed!Joshua!Jsh 13:001}\index[NWIV]{31!Joshua!Jsh 13:001}\index[PNIP]{LORD!Joshua!Jsh 13:001}\index[PNIP]{Joshua!Joshua!Jsh 13:001}\footnote{“Now Joshua was old and stricken in
years” (vs. 1). Again, using the age of Caleb in
14:10 as a guide, Joshua is over 85 years old at
this point, since this is after the conquest of the
Anakim in Hebron and Debir (11:21 cf. 15:13–
15); an estimate based on 23:1 and 24:29 would
probably place Joshua between 90 and 100
years old here.
“There remaineth yet very much land to
be possessed.” After reading about everything
Joshua did in chapters 11–12, the reader might
wonder what the Lord means here. After all,
you’ve been reading where Joshua goes in and
takes all those cities and kills all the inhabitants.
What do you mean, “there remaineth yet very
much land to be possessed”?
Well, any military commander understands what’s going on here. Just because you conquer a land and temporarily drive out the inhabitants, doesn’t mean you “possess” the land. Sometimes the people you whip in war end up owning more of the land you took in battle than they did before you took it away from them. The Japanese own more of the island of Oahu today than the Americans do. The Japanese didn’t need to bomb Pearl Harbor to “possess” Oahu; they simply lost the war and bought it out. The children of Israel have conquered the land and have it under military occupation; but they haven’t “possessed” it yet in the sense of moving in, building houses and cities, plowing the ground and raising crops,
etc. Now the Christian is involved in spiritual conquest. Jerry Falwell used to talk about
“taking a city for Christ.” Well, if you mean getting the Gospel and the truths of the
Scripture to every person in the town, you might have something there. But in the New
Testament, there is no such thing as Christians taking over physical territory to enforce
“Christian laws” on a lost populace, like John Calvin tried to do in Geneva. You might have
Christians ban together in a community and adopt laws based on the Scriptures like the early
settlers and the founders of this country did, but the job of New Testament believers is not to
“Christianize” a country. It’s to evangelize a country.
The Christian spiritually “conquers” now
and physically “possesses” later in the
Millennium. In the Parable of the Pounds, the
“conquest” was the servants “occupying” until
their Lord’s return (Luke 19:13); that is, trading
and making a profit with the money he left
them. The “possession” was the cities they were
given to reign over by their lord as a reward for
their faithfulness. In Pauline terms, “If we
suffer [there’s the “conquest”], we shall also
reign with him [and there’s the “possession”]”
(2 Tim. 2:12).
Again, “And whatsoever ye do, do it heartily, as to the Lord, and not unto men [conquest]; knowing that of the Lord ye shall receive the reward of the inheritance [possession]” (Col. 3:23–24). No Christian is in the Millennium reigning now, like some Preterists, Postmillennialists, and Amillennialists teach; and no Christian is going to before the Rapture. The reigning and the inheritance are future, when your Lord returns. It’s the cross now and the crown later.}
[2] \textcolor[rgb]{0.00,0.00,1.00}{This \emph{is} the land that yet remaineth: all the borders of the Philistines, and all Geshuri,}\index[AWIP]{This!Joshua!Jsh 13:002}\index[AWIP]{\emph{is}!Joshua!Jsh 13:002}\index[AWIP]{the!Joshua!Jsh 13:002}\index[AWIP]{land!Joshua!Jsh 13:002}\index[AWIP]{that!Joshua!Jsh 13:002}\index[AWIP]{yet!Joshua!Jsh 13:002}\index[AWIP]{remaineth!Joshua!Jsh 13:002}\index[AWIP]{all!Joshua!Jsh 13:002}\index[AWIP]{the!Joshua!Jsh 13:002 (2)}\index[AWIP]{borders!Joshua!Jsh 13:002}\index[AWIP]{of!Joshua!Jsh 13:002}\index[AWIP]{the!Joshua!Jsh 13:002 (3)}\index[AWIP]{Philistines!Joshua!Jsh 13:002}\index[AWIP]{and!Joshua!Jsh 13:002}\index[AWIP]{all!Joshua!Jsh 13:002 (2)}\index[AWIP]{Geshuri!Joshua!Jsh 13:002}\index[NWIV]{16!Joshua!Jsh 13:002}\index[PNIP]{Philistines!Joshua!Jsh 13:002}\index[PNIP]{Geshuri!Joshua!Jsh 13:002}
[3] \textcolor[rgb]{0.00,0.00,1.00}{From Sihor, which \emph{is} before Egypt, even unto the borders of Ekron northward, \emph{which} is counted to the Canaanite: five lords of the Philistines; the Gazathites, and the Ashdothites, the Eshkalonites, the Gittites, and the Ekronites; also the Avites:}\index[AWIP]{From!Joshua!Jsh 13:003}\index[AWIP]{Sihor!Joshua!Jsh 13:003}\index[AWIP]{which!Joshua!Jsh 13:003}\index[AWIP]{\emph{is}!Joshua!Jsh 13:003}\index[AWIP]{before!Joshua!Jsh 13:003}\index[AWIP]{Egypt!Joshua!Jsh 13:003}\index[AWIP]{even!Joshua!Jsh 13:003}\index[AWIP]{unto!Joshua!Jsh 13:003}\index[AWIP]{the!Joshua!Jsh 13:003}\index[AWIP]{borders!Joshua!Jsh 13:003}\index[AWIP]{of!Joshua!Jsh 13:003}\index[AWIP]{Ekron!Joshua!Jsh 13:003}\index[AWIP]{northward!Joshua!Jsh 13:003}\index[AWIP]{\emph{which}!Joshua!Jsh 13:003}\index[AWIP]{is!Joshua!Jsh 13:003}\index[AWIP]{counted!Joshua!Jsh 13:003}\index[AWIP]{to!Joshua!Jsh 13:003}\index[AWIP]{the!Joshua!Jsh 13:003 (2)}\index[AWIP]{Canaanite!Joshua!Jsh 13:003}\index[AWIP]{five!Joshua!Jsh 13:003}\index[AWIP]{lords!Joshua!Jsh 13:003}\index[AWIP]{of!Joshua!Jsh 13:003 (2)}\index[AWIP]{the!Joshua!Jsh 13:003 (3)}\index[AWIP]{Philistines!Joshua!Jsh 13:003}\index[AWIP]{the!Joshua!Jsh 13:003 (4)}\index[AWIP]{Gazathites!Joshua!Jsh 13:003}\index[AWIP]{and!Joshua!Jsh 13:003}\index[AWIP]{the!Joshua!Jsh 13:003 (5)}\index[AWIP]{Ashdothites!Joshua!Jsh 13:003}\index[AWIP]{the!Joshua!Jsh 13:003 (6)}\index[AWIP]{Eshkalonites!Joshua!Jsh 13:003}\index[AWIP]{the!Joshua!Jsh 13:003 (7)}\index[AWIP]{Gittites!Joshua!Jsh 13:003}\index[AWIP]{and!Joshua!Jsh 13:003 (2)}\index[AWIP]{the!Joshua!Jsh 13:003 (8)}\index[AWIP]{Ekronites!Joshua!Jsh 13:003}\index[AWIP]{also!Joshua!Jsh 13:003}\index[AWIP]{the!Joshua!Jsh 13:003 (9)}\index[AWIP]{Avites!Joshua!Jsh 13:003}\index[NWIV]{39!Joshua!Jsh 13:003}\index[PNIP]{Canaanite!Joshua!Jsh 13:003}\index[PNIP]{Egypt!Joshua!Jsh 13:003}\index[PNIP]{Philistines!Joshua!Jsh 13:003}\index[PNIP]{Sihor!Joshua!Jsh 13:003}\index[PNIP]{Ekron!Joshua!Jsh 13:003}\index[PNIP]{Gazathites!Joshua!Jsh 13:003}\index[PNIP]{Ashdothites!Joshua!Jsh 13:003}\index[PNIP]{Eshkalonites!Joshua!Jsh 13:003}\index[PNIP]{Gittites!Joshua!Jsh 13:003}\index[PNIP]{Ekronites!Joshua!Jsh 13:003}\index[PNIP]{Avites!Joshua!Jsh 13:003}\footnote{“Sihor, which is before Egypt” (vs. 3) is a river “in the way of Egypt” (Jer. 2:18; Isa. 23:3); that is, the “Via Maris,” what Exodus 13:17 calls “the way of the land of the Philistines.” It was the ancient trade route that connected Egypt with Canaan through northern Sinai. Some expositors have identified it with the Nile, others with the so-called “river (or brook) of Egypt” in the north of the Sinai Peninsula. The International Standard Bible Encyclopedia (under the article on “Shihor”) states that in ancient times a branch of the Nile Delta flowed all the way to “the river of Egypt.” Whatever the case, Sihor was the southernmost border of the Philistines that separated it from Egypt. “Ekron” was the northernmost major city of the Philistines (vs. 3). “The Avites” of verse 3 are “the Avims” of Deuteronomy 2:23. According to that passage, they dwelt in “Hazerim” (the Hebrew word means “villages”), associated with what would later be the Philistine city of “Azzah” (Jer. 25:20), until they were dispossessed by “the Caphtorims,” cousins to the Philistines (see our comments on Gen. 10:14 in that Commentary). The Avim were a race of giants that still remained among the Philistines (see our comments on 11:22).}
[4] \textcolor[rgb]{0.00,0.00,1.00}{From the south, all the land of the Canaanites, and Mearah that \emph{is} beside the Sidonians, unto Aphek, to the borders of the Amorites:}\index[AWIP]{From!Joshua!Jsh 13:004}\index[AWIP]{the!Joshua!Jsh 13:004}\index[AWIP]{south!Joshua!Jsh 13:004}\index[AWIP]{all!Joshua!Jsh 13:004}\index[AWIP]{the!Joshua!Jsh 13:004 (2)}\index[AWIP]{land!Joshua!Jsh 13:004}\index[AWIP]{of!Joshua!Jsh 13:004}\index[AWIP]{the!Joshua!Jsh 13:004 (3)}\index[AWIP]{Canaanites!Joshua!Jsh 13:004}\index[AWIP]{and!Joshua!Jsh 13:004}\index[AWIP]{Mearah!Joshua!Jsh 13:004}\index[AWIP]{that!Joshua!Jsh 13:004}\index[AWIP]{\emph{is}!Joshua!Jsh 13:004}\index[AWIP]{beside!Joshua!Jsh 13:004}\index[AWIP]{the!Joshua!Jsh 13:004 (4)}\index[AWIP]{Sidonians!Joshua!Jsh 13:004}\index[AWIP]{unto!Joshua!Jsh 13:004}\index[AWIP]{Aphek!Joshua!Jsh 13:004}\index[AWIP]{to!Joshua!Jsh 13:004}\index[AWIP]{the!Joshua!Jsh 13:004 (5)}\index[AWIP]{borders!Joshua!Jsh 13:004}\index[AWIP]{of!Joshua!Jsh 13:004 (2)}\index[AWIP]{the!Joshua!Jsh 13:004 (6)}\index[AWIP]{Amorites!Joshua!Jsh 13:004}\index[NWIV]{24!Joshua!Jsh 13:004}\index[PNIP]{Amorites!Joshua!Jsh 13:004}\index[PNIP]{Canaanites!Joshua!Jsh 13:004}\index[PNIP]{Mearah!Joshua!Jsh 13:004}\index[PNIP]{Sidonians!Joshua!Jsh 13:004}\index[PNIP]{Aphek!Joshua!Jsh 13:004}
[5] \textcolor[rgb]{0.00,0.00,1.00}{And the land of the Giblites, and all Lebanon, toward the sunrising, from Baal-gad under mount Hermon unto the entering into Hamath.}\index[AWIP]{And!Joshua!Jsh 13:005}\index[AWIP]{the!Joshua!Jsh 13:005}\index[AWIP]{land!Joshua!Jsh 13:005}\index[AWIP]{of!Joshua!Jsh 13:005}\index[AWIP]{the!Joshua!Jsh 13:005 (2)}\index[AWIP]{Giblites!Joshua!Jsh 13:005}\index[AWIP]{and!Joshua!Jsh 13:005}\index[AWIP]{all!Joshua!Jsh 13:005}\index[AWIP]{Lebanon!Joshua!Jsh 13:005}\index[AWIP]{toward!Joshua!Jsh 13:005}\index[AWIP]{the!Joshua!Jsh 13:005 (3)}\index[AWIP]{sunrising!Joshua!Jsh 13:005}\index[AWIP]{from!Joshua!Jsh 13:005}\index[AWIP]{Baal-gad!Joshua!Jsh 13:005}\index[AWIP]{under!Joshua!Jsh 13:005}\index[AWIP]{mount!Joshua!Jsh 13:005}\index[AWIP]{Hermon!Joshua!Jsh 13:005}\index[AWIP]{unto!Joshua!Jsh 13:005}\index[AWIP]{the!Joshua!Jsh 13:005 (4)}\index[AWIP]{entering!Joshua!Jsh 13:005}\index[AWIP]{into!Joshua!Jsh 13:005}\index[AWIP]{Hamath!Joshua!Jsh 13:005}\index[NWIV]{22!Joshua!Jsh 13:005}\index[PNIP]{Lebanon!Joshua!Jsh 13:005}\index[PNIP]{Giblites!Joshua!Jsh 13:005}\index[PNIP]{Baal-gad!Joshua!Jsh 13:005}\index[PNIP]{Hermon!Joshua!Jsh 13:005}\index[PNIP]{Hamath!Joshua!Jsh 13:005}\footnote{“The entering into Hamath” was a road that led to the ancient Syrian city of Hamath. The road lay in the valley of Lebanon between Mt. Lebanon and Mt. Hermon. It was the northernmost border of Israel (Num. 34:8), but it was not “possessed” (see our remarks under vs. 1) until the times of David and Solomon (1 Kings 8:65). At some point after the kingdom split up under Rehoboam, Israel lost it, but it was retaken under Jeroboam II (2 Kings 14:25). It remains the northern border of Israel’s inheritance all the way into the Millennium (Ezek. 48:1). Now notice in verse 6 that Joshua is commanded to divide the land even though it has not all been possessed yet. There are several accounts like that in the Bible. The Lord tells Abraham the land grant his seed is going to get (Gen. 15:18), and those Jews don’t get it until the time of David nearly a thousand years later (2 Sam. 8:3), and they won’t get it again until Jesus Christ reigns on the throne of David in Jerusalem during the Millennium. God told those Jews the boundaries of the twelve tribes during the Millennial kingdom back in Ezekiel 48, and those tribes haven’t gotten that land as it is laid out there in Ezekiel as yet. But they are going to! You see, God has staked His integrity on a Book. You can find doctrines like the Virgin Birth, the Resurrection, and the Trinity in pagan religions before Christ; but that Book makes statements that are impossible to fulfill from a human standpoint, yet they all come through right on the money. Mohammed will base his religion on a book (the Koran), but then is afraid to tell you anything that will happen in history until the Last Judgment when time is no more (Rev. 10:6). The Bible doesn’t hesitate to tell you what will happen to a land beyond the year A.D. 2014. You can sit back and watch it come to pass, just like the disciples did to 48 prophecies made 400–1,500 years before they were born (see our remarks on prophecy under 6:26).}
[6] \textcolor[rgb]{0.00,0.00,1.00}{All the inhabitants of the hill country from Lebanon unto Misrephoth-maim, \emph{and} all the Sidonians, them will I drive out from before the children of Israel: only divide thou it by lot unto the Israelites for an inheritance, as I have commanded thee.}\index[AWIP]{All!Joshua!Jsh 13:006}\index[AWIP]{the!Joshua!Jsh 13:006}\index[AWIP]{inhabitants!Joshua!Jsh 13:006}\index[AWIP]{of!Joshua!Jsh 13:006}\index[AWIP]{the!Joshua!Jsh 13:006 (2)}\index[AWIP]{hill!Joshua!Jsh 13:006}\index[AWIP]{country!Joshua!Jsh 13:006}\index[AWIP]{from!Joshua!Jsh 13:006}\index[AWIP]{Lebanon!Joshua!Jsh 13:006}\index[AWIP]{unto!Joshua!Jsh 13:006}\index[AWIP]{Misrephoth-maim!Joshua!Jsh 13:006}\index[AWIP]{\emph{and}!Joshua!Jsh 13:006}\index[AWIP]{all!Joshua!Jsh 13:006}\index[AWIP]{the!Joshua!Jsh 13:006 (3)}\index[AWIP]{Sidonians!Joshua!Jsh 13:006}\index[AWIP]{them!Joshua!Jsh 13:006}\index[AWIP]{will!Joshua!Jsh 13:006}\index[AWIP]{I!Joshua!Jsh 13:006}\index[AWIP]{drive!Joshua!Jsh 13:006}\index[AWIP]{out!Joshua!Jsh 13:006}\index[AWIP]{from!Joshua!Jsh 13:006 (2)}\index[AWIP]{before!Joshua!Jsh 13:006}\index[AWIP]{the!Joshua!Jsh 13:006 (4)}\index[AWIP]{children!Joshua!Jsh 13:006}\index[AWIP]{of!Joshua!Jsh 13:006 (2)}\index[AWIP]{Israel!Joshua!Jsh 13:006}\index[AWIP]{only!Joshua!Jsh 13:006}\index[AWIP]{divide!Joshua!Jsh 13:006}\index[AWIP]{thou!Joshua!Jsh 13:006}\index[AWIP]{it!Joshua!Jsh 13:006}\index[AWIP]{by!Joshua!Jsh 13:006}\index[AWIP]{lot!Joshua!Jsh 13:006}\index[AWIP]{unto!Joshua!Jsh 13:006 (2)}\index[AWIP]{the!Joshua!Jsh 13:006 (5)}\index[AWIP]{Israelites!Joshua!Jsh 13:006}\index[AWIP]{for!Joshua!Jsh 13:006}\index[AWIP]{an!Joshua!Jsh 13:006}\index[AWIP]{inheritance!Joshua!Jsh 13:006}\index[AWIP]{as!Joshua!Jsh 13:006}\index[AWIP]{I!Joshua!Jsh 13:006 (2)}\index[AWIP]{have!Joshua!Jsh 13:006}\index[AWIP]{commanded!Joshua!Jsh 13:006}\index[AWIP]{thee!Joshua!Jsh 13:006}\index[NWIV]{43!Joshua!Jsh 13:006}\index[PNIP]{I!Joshua!Jsh 13:006}\index[PNIP]{Israel!Joshua!Jsh 13:006}\index[PNIP]{Lebanon!Joshua!Jsh 13:006}\index[PNIP]{Sidonians!Joshua!Jsh 13:006}\index[PNIP]{Misrephoth-maim!Joshua!Jsh 13:006}\index[PNIP]{Israelites!Joshua!Jsh 13:006}
[7] \textcolor[rgb]{0.00,0.00,1.00}{Now therefore divide this land for an inheritance unto the nine tribes, and the half tribe of Manasseh,}\index[AWIP]{Now!Joshua!Jsh 13:007}\index[AWIP]{therefore!Joshua!Jsh 13:007}\index[AWIP]{divide!Joshua!Jsh 13:007}\index[AWIP]{this!Joshua!Jsh 13:007}\index[AWIP]{land!Joshua!Jsh 13:007}\index[AWIP]{for!Joshua!Jsh 13:007}\index[AWIP]{an!Joshua!Jsh 13:007}\index[AWIP]{inheritance!Joshua!Jsh 13:007}\index[AWIP]{unto!Joshua!Jsh 13:007}\index[AWIP]{the!Joshua!Jsh 13:007}\index[AWIP]{nine!Joshua!Jsh 13:007}\index[AWIP]{tribes!Joshua!Jsh 13:007}\index[AWIP]{and!Joshua!Jsh 13:007}\index[AWIP]{the!Joshua!Jsh 13:007 (2)}\index[AWIP]{half!Joshua!Jsh 13:007}\index[AWIP]{tribe!Joshua!Jsh 13:007}\index[AWIP]{of!Joshua!Jsh 13:007}\index[AWIP]{Manasseh!Joshua!Jsh 13:007}\index[NWIV]{18!Joshua!Jsh 13:007}\index[PNIP]{Manasseh!Joshua!Jsh 13:007}
[8] \textcolor[rgb]{0.00,0.00,1.00}{With whom the Reubenites and the Gadites have received their inheritance, which Moses gave them, beyond Jordan eastward, \emph{even} as Moses the servant of the LORD gave them;}\index[AWIP]{With!Joshua!Jsh 13:008}\index[AWIP]{whom!Joshua!Jsh 13:008}\index[AWIP]{the!Joshua!Jsh 13:008}\index[AWIP]{Reubenites!Joshua!Jsh 13:008}\index[AWIP]{and!Joshua!Jsh 13:008}\index[AWIP]{the!Joshua!Jsh 13:008 (2)}\index[AWIP]{Gadites!Joshua!Jsh 13:008}\index[AWIP]{have!Joshua!Jsh 13:008}\index[AWIP]{received!Joshua!Jsh 13:008}\index[AWIP]{their!Joshua!Jsh 13:008}\index[AWIP]{inheritance!Joshua!Jsh 13:008}\index[AWIP]{which!Joshua!Jsh 13:008}\index[AWIP]{Moses!Joshua!Jsh 13:008}\index[AWIP]{gave!Joshua!Jsh 13:008}\index[AWIP]{them!Joshua!Jsh 13:008}\index[AWIP]{beyond!Joshua!Jsh 13:008}\index[AWIP]{Jordan!Joshua!Jsh 13:008}\index[AWIP]{eastward!Joshua!Jsh 13:008}\index[AWIP]{\emph{even}!Joshua!Jsh 13:008}\index[AWIP]{as!Joshua!Jsh 13:008}\index[AWIP]{Moses!Joshua!Jsh 13:008 (2)}\index[AWIP]{the!Joshua!Jsh 13:008 (3)}\index[AWIP]{servant!Joshua!Jsh 13:008}\index[AWIP]{of!Joshua!Jsh 13:008}\index[AWIP]{the!Joshua!Jsh 13:008 (4)}\index[AWIP]{LORD!Joshua!Jsh 13:008}\index[AWIP]{gave!Joshua!Jsh 13:008 (2)}\index[AWIP]{them!Joshua!Jsh 13:008 (2)}\index[NWIV]{28!Joshua!Jsh 13:008}\index[PNIP]{Jordan!Joshua!Jsh 13:008}\index[PNIP]{LORD!Joshua!Jsh 13:008}\index[PNIP]{Moses!Joshua!Jsh 13:008}\index[PNIP]{Reubenites!Joshua!Jsh 13:008}\index[PNIP]{Gadites!Joshua!Jsh 13:008}
[9] \textcolor[rgb]{0.00,0.00,1.00}{From Aroer, that \emph{is} upon the bank of the river Arnon, and the city that \emph{is} in the midst of the river, and all the plain of Medeba unto Dibon;}\index[AWIP]{From!Joshua!Jsh 13:009}\index[AWIP]{Aroer!Joshua!Jsh 13:009}\index[AWIP]{that!Joshua!Jsh 13:009}\index[AWIP]{\emph{is}!Joshua!Jsh 13:009}\index[AWIP]{upon!Joshua!Jsh 13:009}\index[AWIP]{the!Joshua!Jsh 13:009}\index[AWIP]{bank!Joshua!Jsh 13:009}\index[AWIP]{of!Joshua!Jsh 13:009}\index[AWIP]{the!Joshua!Jsh 13:009 (2)}\index[AWIP]{river!Joshua!Jsh 13:009}\index[AWIP]{Arnon!Joshua!Jsh 13:009}\index[AWIP]{and!Joshua!Jsh 13:009}\index[AWIP]{the!Joshua!Jsh 13:009 (3)}\index[AWIP]{city!Joshua!Jsh 13:009}\index[AWIP]{that!Joshua!Jsh 13:009 (2)}\index[AWIP]{\emph{is}!Joshua!Jsh 13:009 (2)}\index[AWIP]{in!Joshua!Jsh 13:009}\index[AWIP]{the!Joshua!Jsh 13:009 (4)}\index[AWIP]{midst!Joshua!Jsh 13:009}\index[AWIP]{of!Joshua!Jsh 13:009 (2)}\index[AWIP]{the!Joshua!Jsh 13:009 (5)}\index[AWIP]{river!Joshua!Jsh 13:009 (2)}\index[AWIP]{and!Joshua!Jsh 13:009 (2)}\index[AWIP]{all!Joshua!Jsh 13:009}\index[AWIP]{the!Joshua!Jsh 13:009 (6)}\index[AWIP]{plain!Joshua!Jsh 13:009}\index[AWIP]{of!Joshua!Jsh 13:009 (3)}\index[AWIP]{Medeba!Joshua!Jsh 13:009}\index[AWIP]{unto!Joshua!Jsh 13:009}\index[AWIP]{Dibon!Joshua!Jsh 13:009}\index[NWIV]{30!Joshua!Jsh 13:009}\index[PNIP]{Arnon!Joshua!Jsh 13:009}\index[PNIP]{Dibon!Joshua!Jsh 13:009}\index[PNIP]{Aroer!Joshua!Jsh 13:009}\index[PNIP]{Medeba!Joshua!Jsh 13:009}
[10] \textcolor[rgb]{0.00,0.00,1.00}{And all the cities of Sihon king of the Amorites, which reigned in Heshbon, unto the border of the children of Ammon;}\index[AWIP]{And!Joshua!Jsh 13:010}\index[AWIP]{all!Joshua!Jsh 13:010}\index[AWIP]{the!Joshua!Jsh 13:010}\index[AWIP]{cities!Joshua!Jsh 13:010}\index[AWIP]{of!Joshua!Jsh 13:010}\index[AWIP]{Sihon!Joshua!Jsh 13:010}\index[AWIP]{king!Joshua!Jsh 13:010}\index[AWIP]{of!Joshua!Jsh 13:010 (2)}\index[AWIP]{the!Joshua!Jsh 13:010 (2)}\index[AWIP]{Amorites!Joshua!Jsh 13:010}\index[AWIP]{which!Joshua!Jsh 13:010}\index[AWIP]{reigned!Joshua!Jsh 13:010}\index[AWIP]{in!Joshua!Jsh 13:010}\index[AWIP]{Heshbon!Joshua!Jsh 13:010}\index[AWIP]{unto!Joshua!Jsh 13:010}\index[AWIP]{the!Joshua!Jsh 13:010 (3)}\index[AWIP]{border!Joshua!Jsh 13:010}\index[AWIP]{of!Joshua!Jsh 13:010 (3)}\index[AWIP]{the!Joshua!Jsh 13:010 (4)}\index[AWIP]{children!Joshua!Jsh 13:010}\index[AWIP]{of!Joshua!Jsh 13:010 (4)}\index[AWIP]{Ammon!Joshua!Jsh 13:010}\index[NWIV]{22!Joshua!Jsh 13:010}\index[PNIP]{Amorites!Joshua!Jsh 13:010}\index[PNIP]{Heshbon!Joshua!Jsh 13:010}\index[PNIP]{Sihon!Joshua!Jsh 13:010}\index[PNIP]{Ammon!Joshua!Jsh 13:010}
[11] \textcolor[rgb]{0.00,0.00,1.00}{And Gilead, and the border of the Geshurites and Maachathites, and all mount Hermon, and all Bashan unto Salcah;}\index[AWIP]{And!Joshua!Jsh 13:011}\index[AWIP]{Gilead!Joshua!Jsh 13:011}\index[AWIP]{and!Joshua!Jsh 13:011}\index[AWIP]{the!Joshua!Jsh 13:011}\index[AWIP]{border!Joshua!Jsh 13:011}\index[AWIP]{of!Joshua!Jsh 13:011}\index[AWIP]{the!Joshua!Jsh 13:011 (2)}\index[AWIP]{Geshurites!Joshua!Jsh 13:011}\index[AWIP]{and!Joshua!Jsh 13:011 (2)}\index[AWIP]{Maachathites!Joshua!Jsh 13:011}\index[AWIP]{and!Joshua!Jsh 13:011 (3)}\index[AWIP]{all!Joshua!Jsh 13:011}\index[AWIP]{mount!Joshua!Jsh 13:011}\index[AWIP]{Hermon!Joshua!Jsh 13:011}\index[AWIP]{and!Joshua!Jsh 13:011 (4)}\index[AWIP]{all!Joshua!Jsh 13:011 (2)}\index[AWIP]{Bashan!Joshua!Jsh 13:011}\index[AWIP]{unto!Joshua!Jsh 13:011}\index[AWIP]{Salcah!Joshua!Jsh 13:011}\index[NWIV]{19!Joshua!Jsh 13:011}\index[PNIP]{Gilead!Joshua!Jsh 13:011}\index[PNIP]{Hermon!Joshua!Jsh 13:011}\index[PNIP]{Geshurites!Joshua!Jsh 13:011}\index[PNIP]{Maachathites!Joshua!Jsh 13:011}\index[PNIP]{Bashan!Joshua!Jsh 13:011}\index[PNIP]{Salcah!Joshua!Jsh 13:011}
[12] \textcolor[rgb]{0.00,0.00,1.00}{All the kingdom of Og in Bashan, which reigned in Ashtaroth and in Edrei, who remained of the remnant of the giants: for these did Moses smite, and cast them out.}\index[AWIP]{All!Joshua!Jsh 13:012}\index[AWIP]{the!Joshua!Jsh 13:012}\index[AWIP]{kingdom!Joshua!Jsh 13:012}\index[AWIP]{of!Joshua!Jsh 13:012}\index[AWIP]{Og!Joshua!Jsh 13:012}\index[AWIP]{in!Joshua!Jsh 13:012}\index[AWIP]{Bashan!Joshua!Jsh 13:012}\index[AWIP]{which!Joshua!Jsh 13:012}\index[AWIP]{reigned!Joshua!Jsh 13:012}\index[AWIP]{in!Joshua!Jsh 13:012 (2)}\index[AWIP]{Ashtaroth!Joshua!Jsh 13:012}\index[AWIP]{and!Joshua!Jsh 13:012}\index[AWIP]{in!Joshua!Jsh 13:012 (3)}\index[AWIP]{Edrei!Joshua!Jsh 13:012}\index[AWIP]{who!Joshua!Jsh 13:012}\index[AWIP]{remained!Joshua!Jsh 13:012}\index[AWIP]{of!Joshua!Jsh 13:012 (2)}\index[AWIP]{the!Joshua!Jsh 13:012 (2)}\index[AWIP]{remnant!Joshua!Jsh 13:012}\index[AWIP]{of!Joshua!Jsh 13:012 (3)}\index[AWIP]{the!Joshua!Jsh 13:012 (3)}\index[AWIP]{giants!Joshua!Jsh 13:012}\index[AWIP]{for!Joshua!Jsh 13:012}\index[AWIP]{these!Joshua!Jsh 13:012}\index[AWIP]{did!Joshua!Jsh 13:012}\index[AWIP]{Moses!Joshua!Jsh 13:012}\index[AWIP]{smite!Joshua!Jsh 13:012}\index[AWIP]{and!Joshua!Jsh 13:012 (2)}\index[AWIP]{cast!Joshua!Jsh 13:012}\index[AWIP]{them!Joshua!Jsh 13:012}\index[AWIP]{out!Joshua!Jsh 13:012}\index[NWIV]{31!Joshua!Jsh 13:012}\index[PNIP]{Moses!Joshua!Jsh 13:012}\index[PNIP]{Bashan!Joshua!Jsh 13:012}\index[PNIP]{Og!Joshua!Jsh 13:012}\index[PNIP]{Ashtaroth!Joshua!Jsh 13:012}\index[PNIP]{Edrei!Joshua!Jsh 13:012}
[13] \textcolor[rgb]{0.00,0.00,1.00}{Nevertheless the children of Israel expelled not the Geshurites, nor the Maachathites: but the Geshurites and the Maachathites dwell among the Israelites until this day.}\index[AWIP]{Nevertheless!Joshua!Jsh 13:013}\index[AWIP]{the!Joshua!Jsh 13:013}\index[AWIP]{children!Joshua!Jsh 13:013}\index[AWIP]{of!Joshua!Jsh 13:013}\index[AWIP]{Israel!Joshua!Jsh 13:013}\index[AWIP]{expelled!Joshua!Jsh 13:013}\index[AWIP]{not!Joshua!Jsh 13:013}\index[AWIP]{the!Joshua!Jsh 13:013 (2)}\index[AWIP]{Geshurites!Joshua!Jsh 13:013}\index[AWIP]{nor!Joshua!Jsh 13:013}\index[AWIP]{the!Joshua!Jsh 13:013 (3)}\index[AWIP]{Maachathites!Joshua!Jsh 13:013}\index[AWIP]{but!Joshua!Jsh 13:013}\index[AWIP]{the!Joshua!Jsh 13:013 (4)}\index[AWIP]{Geshurites!Joshua!Jsh 13:013 (2)}\index[AWIP]{and!Joshua!Jsh 13:013}\index[AWIP]{the!Joshua!Jsh 13:013 (5)}\index[AWIP]{Maachathites!Joshua!Jsh 13:013 (2)}\index[AWIP]{dwell!Joshua!Jsh 13:013}\index[AWIP]{among!Joshua!Jsh 13:013}\index[AWIP]{the!Joshua!Jsh 13:013 (6)}\index[AWIP]{Israelites!Joshua!Jsh 13:013}\index[AWIP]{until!Joshua!Jsh 13:013}\index[AWIP]{this!Joshua!Jsh 13:013}\index[AWIP]{day!Joshua!Jsh 13:013}\index[NWIV]{25!Joshua!Jsh 13:013}\index[PNIP]{Israel!Joshua!Jsh 13:013}\index[PNIP]{Israelites!Joshua!Jsh 13:013}\index[PNIP]{Geshurites!Joshua!Jsh 13:013}\index[PNIP]{Maachathites!Joshua!Jsh 13:013}\footnote{Something important to note in verse 2 here in chapter 13 is that there are two “Geshuri.” There is one in the south of Canaan on the west side of Jordan, and that is the one here in verse 2 (see also 1 Sam. 27:8). The other land of the “Geshurites” is located up in Bashan in the north, on the east side of Jordan, and that is the one in verse 13 (see also 12:5; Deut. 3:14). It was from this second “Geshur” that one of the greatest types of the Antichrist came--Absalom (2 Samuel 3:3, 13:37, 15:8). Notice that the first mention of incomplete obedience on the part of the twelve tribes in driving out the heathen inhabitants of the land shows up in 13:13 (watch out for the thirteens). Those Geshurites and Maachathites lived in territory given to the half tribe of Manasseh, and the Manassites didn’t drive them out. That’s the first sign that something is going wrong. The Lord told those Jews to “drive out all the inhabitants of the land . . . destroy all their pictures . . . destroy all their molten images . . . pluck down all their high places: And . . . dispossess the inhabitants of the land” (Num. 33:52–53). What started out as one tribe on the east side of Jordan letting two heathen peoples “slide” resulted in seven tribes in Canaan proper, as well as the rest of the Manassites, not expelling the pagan peoples in their inheritances (Judg. 1:19–36). That incomplete obedience eventuated in the idolatry of the people in Judges and the continuous cycle of apostasy, enslavement, repentance, deliverance, and apostasy again (Judg. 2:11–19). Because of this incomplete obedience by one tribe, Israel winds up in total anarchy—“In those days there was no king in Israel: every man did that which was right in his own eyes” (Judg. 21:25). Literally thousands of people ended up suffering because one tribe allowed the heathen idolaters in its territory to live. The trouble David had in his kingdom was from a son whose mother was from the Geshurites and Maachathites (2 Sam. 3:3). The spiritual lesson for the Christian is not to compromise with heathen practices. When a Christian refuses to deal with sin in his life, it grows and torments him until it controls him like those pagan peoples controlled those
Israelites in the book of Judges.}
[14] \textcolor[rgb]{0.00,0.00,1.00}{Only unto the tribe of Levi he gave none inheritance; the sacrifices of the LORD God of Israel made by fire \emph{are} their inheritance, as he said unto them.}\index[AWIP]{Only!Joshua!Jsh 13:014}\index[AWIP]{unto!Joshua!Jsh 13:014}\index[AWIP]{the!Joshua!Jsh 13:014}\index[AWIP]{tribe!Joshua!Jsh 13:014}\index[AWIP]{of!Joshua!Jsh 13:014}\index[AWIP]{Levi!Joshua!Jsh 13:014}\index[AWIP]{he!Joshua!Jsh 13:014}\index[AWIP]{gave!Joshua!Jsh 13:014}\index[AWIP]{none!Joshua!Jsh 13:014}\index[AWIP]{inheritance!Joshua!Jsh 13:014}\index[AWIP]{the!Joshua!Jsh 13:014 (2)}\index[AWIP]{sacrifices!Joshua!Jsh 13:014}\index[AWIP]{of!Joshua!Jsh 13:014 (2)}\index[AWIP]{the!Joshua!Jsh 13:014 (3)}\index[AWIP]{LORD!Joshua!Jsh 13:014}\index[AWIP]{God!Joshua!Jsh 13:014}\index[AWIP]{of!Joshua!Jsh 13:014 (3)}\index[AWIP]{Israel!Joshua!Jsh 13:014}\index[AWIP]{made!Joshua!Jsh 13:014}\index[AWIP]{by!Joshua!Jsh 13:014}\index[AWIP]{fire!Joshua!Jsh 13:014}\index[AWIP]{\emph{are}!Joshua!Jsh 13:014}\index[AWIP]{their!Joshua!Jsh 13:014}\index[AWIP]{inheritance!Joshua!Jsh 13:014 (2)}\index[AWIP]{as!Joshua!Jsh 13:014}\index[AWIP]{he!Joshua!Jsh 13:014 (2)}\index[AWIP]{said!Joshua!Jsh 13:014}\index[AWIP]{unto!Joshua!Jsh 13:014 (2)}\index[AWIP]{them!Joshua!Jsh 13:014}\index[NWIV]{29!Joshua!Jsh 13:014}\index[PNIP]{God!Joshua!Jsh 13:014}\index[PNIP]{Israel!Joshua!Jsh 13:014}\index[PNIP]{LORD!Joshua!Jsh 13:014}\index[PNIP]{Levi!Joshua!Jsh 13:014}\footnote{Now before Joshua starts “divvying up” the
land, you are told this: “Only unto the tribe of
Levi he gave none inheritance; the sacrifices
of the LORD God of Israel made by fire are
their inheritance, as he said unto them” (vs.
14). Verse 33 says the Lord didn’t give the
Levites any inheritance because “the LORD
God of Israel was their inheritance.” Instead
of being given a tract of land for their whole
tribe to live and raise crops and cattle, the
Levites were scattered all throughout Israel in
individual cities. Their “living” was the
sacrifices and tithes brought to the Tabernacle
(Num. 18:20–24; Deut. 18:1–5). The Lord was
the Levities’ “inheritance” because they were
the ones who stood up for the Lord during that
mess with the golden calf back there in Exodus
32; because of that, God chose the tribe of Levi
as His priests (Deut. 10:8–9). Because they had
to be available for the service of the Tabernacle,
they couldn’t be preoccupied with land and
crops and cattle and the like.
Those Levites typify the Christian in this
age, as far as earthly, permanent possessions go.
The born-again child of God is part of a holy,
royal priesthood that is called “to offer up
spiritual sacrifices, acceptable to God by
Jesus Christ” (1 Pet. 2:5, 9). As such, he has
no permanent possessions down here on this
earth right now; everything he has is on loan to
him for service to God.
The old song goes: “This world is not my
home. I’m just a’ passin’ through.” If you are
saved, your home is up in Glory with your
Saviour (Phil. 3:20; 2 Cor. 5:8; John 14:3). All
you are on this earth is a pilgrim who is a
foreigner (1 Pet. 2:11) to “this present evil
world” (Gal. 1:4).
Like the Levites, the Christian's inheritance is the Lord. When he dies, he has ``an inheritance $\hdots$ reserved in heaven'' (1 Peter 1:4) because that's where his Lord is bodily right now (2 Cor. 5:8; Phil. 1:23). That's why the believer is to set his ``affection on things above, not on things on the earth'' (Col. 3:1--2), because that is where his ``treasure'' is supposed to be laid up (Matt. 6:19--20). When his Lord returns to this earth to rule and reign a thousand years, he comes with Him (Revelation 19:14; 2 Thessalonians 1:7--10), and then he will have an inheritance in His kingdom, if he has faithfully served Him now (1 Cor. 6:9--10; Galatians 5:19--21).}\\
\\
\P  \textcolor[rgb]{0.00,0.00,1.00}{And Moses gave unto the tribe of the children of Reuben \emph{inheritance} according to their families.}\index[AWIP]{And!Joshua!Jsh 13:015}\index[AWIP]{Moses!Joshua!Jsh 13:015}\index[AWIP]{gave!Joshua!Jsh 13:015}\index[AWIP]{unto!Joshua!Jsh 13:015}\index[AWIP]{the!Joshua!Jsh 13:015}\index[AWIP]{tribe!Joshua!Jsh 13:015}\index[AWIP]{of!Joshua!Jsh 13:015}\index[AWIP]{the!Joshua!Jsh 13:015 (2)}\index[AWIP]{children!Joshua!Jsh 13:015}\index[AWIP]{of!Joshua!Jsh 13:015 (2)}\index[AWIP]{Reuben!Joshua!Jsh 13:015}\index[AWIP]{\emph{inheritance}!Joshua!Jsh 13:015}\index[AWIP]{according!Joshua!Jsh 13:015}\index[AWIP]{to!Joshua!Jsh 13:015}\index[AWIP]{their!Joshua!Jsh 13:015}\index[AWIP]{families!Joshua!Jsh 13:015}\index[NWIV]{16!Joshua!Jsh 13:015}\index[PNIP]{Moses!Joshua!Jsh 13:015}\index[PNIP]{Reuben!Joshua!Jsh 13:015}\footnote{Reuben is the first tribe to receive its inheritance because Reuben was Jacob’s firstborn (1 Chron. 5:1). The tribe of Reuben got the lower half of Sihon’s kingdom from Aroer at the southern end to Sihon’s capital city of Heshbon.}
[16] \textcolor[rgb]{0.00,0.00,1.00}{And their coast was from Aroer, that \emph{is} on the bank of the river Arnon, and the city that \emph{is} in the midst of the river, and all the plain by Medeba;}\index[AWIP]{And!Joshua!Jsh 13:016}\index[AWIP]{their!Joshua!Jsh 13:016}\index[AWIP]{coast!Joshua!Jsh 13:016}\index[AWIP]{was!Joshua!Jsh 13:016}\index[AWIP]{from!Joshua!Jsh 13:016}\index[AWIP]{Aroer!Joshua!Jsh 13:016}\index[AWIP]{that!Joshua!Jsh 13:016}\index[AWIP]{\emph{is}!Joshua!Jsh 13:016}\index[AWIP]{on!Joshua!Jsh 13:016}\index[AWIP]{the!Joshua!Jsh 13:016}\index[AWIP]{bank!Joshua!Jsh 13:016}\index[AWIP]{of!Joshua!Jsh 13:016}\index[AWIP]{the!Joshua!Jsh 13:016 (2)}\index[AWIP]{river!Joshua!Jsh 13:016}\index[AWIP]{Arnon!Joshua!Jsh 13:016}\index[AWIP]{and!Joshua!Jsh 13:016}\index[AWIP]{the!Joshua!Jsh 13:016 (3)}\index[AWIP]{city!Joshua!Jsh 13:016}\index[AWIP]{that!Joshua!Jsh 13:016 (2)}\index[AWIP]{\emph{is}!Joshua!Jsh 13:016 (2)}\index[AWIP]{in!Joshua!Jsh 13:016}\index[AWIP]{the!Joshua!Jsh 13:016 (4)}\index[AWIP]{midst!Joshua!Jsh 13:016}\index[AWIP]{of!Joshua!Jsh 13:016 (2)}\index[AWIP]{the!Joshua!Jsh 13:016 (5)}\index[AWIP]{river!Joshua!Jsh 13:016 (2)}\index[AWIP]{and!Joshua!Jsh 13:016 (2)}\index[AWIP]{all!Joshua!Jsh 13:016}\index[AWIP]{the!Joshua!Jsh 13:016 (6)}\index[AWIP]{plain!Joshua!Jsh 13:016}\index[AWIP]{by!Joshua!Jsh 13:016}\index[AWIP]{Medeba!Joshua!Jsh 13:016}\index[NWIV]{32!Joshua!Jsh 13:016}\index[PNIP]{Arnon!Joshua!Jsh 13:016}\index[PNIP]{Aroer!Joshua!Jsh 13:016}\index[PNIP]{Medeba!Joshua!Jsh 13:016}
[17] \textcolor[rgb]{0.00,0.00,1.00}{Heshbon, and all her cities that \emph{are} in the plain; Dibon, and Bamoth-baal, and Beth-baal-meon,}\index[AWIP]{Heshbon!Joshua!Jsh 13:017}\index[AWIP]{and!Joshua!Jsh 13:017}\index[AWIP]{all!Joshua!Jsh 13:017}\index[AWIP]{her!Joshua!Jsh 13:017}\index[AWIP]{cities!Joshua!Jsh 13:017}\index[AWIP]{that!Joshua!Jsh 13:017}\index[AWIP]{\emph{are}!Joshua!Jsh 13:017}\index[AWIP]{in!Joshua!Jsh 13:017}\index[AWIP]{the!Joshua!Jsh 13:017}\index[AWIP]{plain!Joshua!Jsh 13:017}\index[AWIP]{Dibon!Joshua!Jsh 13:017}\index[AWIP]{and!Joshua!Jsh 13:017 (2)}\index[AWIP]{Bamoth-baal!Joshua!Jsh 13:017}\index[AWIP]{and!Joshua!Jsh 13:017 (3)}\index[AWIP]{Beth-baal-meon!Joshua!Jsh 13:017}\index[NWIV]{15!Joshua!Jsh 13:017}\index[PNIP]{Dibon!Joshua!Jsh 13:017}\index[PNIP]{Heshbon!Joshua!Jsh 13:017}\index[PNIP]{Bamoth-baal!Joshua!Jsh 13:017}\index[PNIP]{Beth-baal-meon!Joshua!Jsh 13:017}
[18] \textcolor[rgb]{0.00,0.00,1.00}{And Jahazah, and Kedemoth, and Mephaath,}\index[AWIP]{And!Joshua!Jsh 13:018}\index[AWIP]{Jahazah!Joshua!Jsh 13:018}\index[AWIP]{and!Joshua!Jsh 13:018}\index[AWIP]{Kedemoth!Joshua!Jsh 13:018}\index[AWIP]{and!Joshua!Jsh 13:018 (2)}\index[AWIP]{Mephaath!Joshua!Jsh 13:018}\index[NWIV]{6!Joshua!Jsh 13:018}\index[PNIP]{Jahazah!Joshua!Jsh 13:018}\index[PNIP]{Kedemoth!Joshua!Jsh 13:018}\index[PNIP]{Mephaath!Joshua!Jsh 13:018}
[19] \textcolor[rgb]{0.00,0.00,1.00}{And Kirjathaim, and Sibmah, and Zareth-shahar in the mount of the valley,}\index[AWIP]{And!Joshua!Jsh 13:019}\index[AWIP]{Kirjathaim!Joshua!Jsh 13:019}\index[AWIP]{and!Joshua!Jsh 13:019}\index[AWIP]{Sibmah!Joshua!Jsh 13:019}\index[AWIP]{and!Joshua!Jsh 13:019 (2)}\index[AWIP]{Zareth-shahar!Joshua!Jsh 13:019}\index[AWIP]{in!Joshua!Jsh 13:019}\index[AWIP]{the!Joshua!Jsh 13:019}\index[AWIP]{mount!Joshua!Jsh 13:019}\index[AWIP]{of!Joshua!Jsh 13:019}\index[AWIP]{the!Joshua!Jsh 13:019 (2)}\index[AWIP]{valley!Joshua!Jsh 13:019}\index[NWIV]{12!Joshua!Jsh 13:019}\index[PNIP]{Kirjathaim!Joshua!Jsh 13:019}\index[PNIP]{Sibmah!Joshua!Jsh 13:019}\index[PNIP]{Zareth-shahar!Joshua!Jsh 13:019}
[20] \textcolor[rgb]{0.00,0.00,1.00}{And Beth-peor, and Ashdoth-pisgah, and Beth-jeshimoth,}\index[AWIP]{And!Joshua!Jsh 13:020}\index[AWIP]{Beth-peor!Joshua!Jsh 13:020}\index[AWIP]{and!Joshua!Jsh 13:020}\index[AWIP]{Ashdoth-pisgah!Joshua!Jsh 13:020}\index[AWIP]{and!Joshua!Jsh 13:020 (2)}\index[AWIP]{Beth-jeshimoth!Joshua!Jsh 13:020}\index[NWIV]{6!Joshua!Jsh 13:020}\index[PNIP]{Beth-peor!Joshua!Jsh 13:020}\index[PNIP]{Ashdoth-pisgah!Joshua!Jsh 13:020}\index[PNIP]{Beth-jeshimoth!Joshua!Jsh 13:020}\footnote{Notice “Beth-peor, and Ashdoth-pisgah”
in verse 20. That’s Peor where the Israelites
went into idolatry and fornication with the
women of Moab and Midian (see Num. 25),
and Mt. Pisgah where Moses went up to view
the Promised Land and die (Deut. 34). So in
verse 21 here in Joshua 13, you are told about
“the princes of Midian” “whom Moses
smote” in response to the sin of Baal-peor
(Num. 25:16–18).}
[21] \textcolor[rgb]{0.00,0.00,1.00}{And all the cities of the plain, and all the kingdom of Sihon king of the Amorites, which reigned in Heshbon, whom Moses smote with the princes of Midian, Evi, and Rekem, and Zur, and Hur, and Reba, \emph{which} \emph{were} dukes of Sihon, dwelling in the country.}\index[AWIP]{And!Joshua!Jsh 13:021}\index[AWIP]{all!Joshua!Jsh 13:021}\index[AWIP]{the!Joshua!Jsh 13:021}\index[AWIP]{cities!Joshua!Jsh 13:021}\index[AWIP]{of!Joshua!Jsh 13:021}\index[AWIP]{the!Joshua!Jsh 13:021 (2)}\index[AWIP]{plain!Joshua!Jsh 13:021}\index[AWIP]{and!Joshua!Jsh 13:021}\index[AWIP]{all!Joshua!Jsh 13:021 (2)}\index[AWIP]{the!Joshua!Jsh 13:021 (3)}\index[AWIP]{kingdom!Joshua!Jsh 13:021}\index[AWIP]{of!Joshua!Jsh 13:021 (2)}\index[AWIP]{Sihon!Joshua!Jsh 13:021}\index[AWIP]{king!Joshua!Jsh 13:021}\index[AWIP]{of!Joshua!Jsh 13:021 (3)}\index[AWIP]{the!Joshua!Jsh 13:021 (4)}\index[AWIP]{Amorites!Joshua!Jsh 13:021}\index[AWIP]{which!Joshua!Jsh 13:021}\index[AWIP]{reigned!Joshua!Jsh 13:021}\index[AWIP]{in!Joshua!Jsh 13:021}\index[AWIP]{Heshbon!Joshua!Jsh 13:021}\index[AWIP]{whom!Joshua!Jsh 13:021}\index[AWIP]{Moses!Joshua!Jsh 13:021}\index[AWIP]{smote!Joshua!Jsh 13:021}\index[AWIP]{with!Joshua!Jsh 13:021}\index[AWIP]{the!Joshua!Jsh 13:021 (5)}\index[AWIP]{princes!Joshua!Jsh 13:021}\index[AWIP]{of!Joshua!Jsh 13:021 (4)}\index[AWIP]{Midian!Joshua!Jsh 13:021}\index[AWIP]{Evi!Joshua!Jsh 13:021}\index[AWIP]{and!Joshua!Jsh 13:021 (2)}\index[AWIP]{Rekem!Joshua!Jsh 13:021}\index[AWIP]{and!Joshua!Jsh 13:021 (3)}\index[AWIP]{Zur!Joshua!Jsh 13:021}\index[AWIP]{and!Joshua!Jsh 13:021 (4)}\index[AWIP]{Hur!Joshua!Jsh 13:021}\index[AWIP]{and!Joshua!Jsh 13:021 (5)}\index[AWIP]{Reba!Joshua!Jsh 13:021}\index[AWIP]{\emph{which}!Joshua!Jsh 13:021}\index[AWIP]{\emph{were}!Joshua!Jsh 13:021}\index[AWIP]{dukes!Joshua!Jsh 13:021}\index[AWIP]{of!Joshua!Jsh 13:021 (5)}\index[AWIP]{Sihon!Joshua!Jsh 13:021 (2)}\index[AWIP]{dwelling!Joshua!Jsh 13:021}\index[AWIP]{in!Joshua!Jsh 13:021 (2)}\index[AWIP]{the!Joshua!Jsh 13:021 (6)}\index[AWIP]{country!Joshua!Jsh 13:021}\index[NWIV]{47!Joshua!Jsh 13:021}\index[PNIP]{Amorites!Joshua!Jsh 13:021}\index[PNIP]{Heshbon!Joshua!Jsh 13:021}\index[PNIP]{Hur!Joshua!Jsh 13:021}\index[PNIP]{Midian!Joshua!Jsh 13:021}\index[PNIP]{Moses!Joshua!Jsh 13:021}\index[PNIP]{Sihon!Joshua!Jsh 13:021}\index[PNIP]{Evi!Joshua!Jsh 13:021}\index[PNIP]{Rekem!Joshua!Jsh 13:021}\index[PNIP]{Zur!Joshua!Jsh 13:021}\index[PNIP]{Reba!Joshua!Jsh 13:021}\\
\\
\P  \textcolor[rgb]{0.00,0.00,1.00}{Balaam also the son of Beor, the soothsayer, did the children of Israel slay with the sword among them that were slain by them.}\index[AWIP]{Balaam!Joshua!Jsh 13:022}\index[AWIP]{also!Joshua!Jsh 13:022}\index[AWIP]{the!Joshua!Jsh 13:022}\index[AWIP]{son!Joshua!Jsh 13:022}\index[AWIP]{of!Joshua!Jsh 13:022}\index[AWIP]{Beor!Joshua!Jsh 13:022}\index[AWIP]{the!Joshua!Jsh 13:022 (2)}\index[AWIP]{soothsayer!Joshua!Jsh 13:022}\index[AWIP]{did!Joshua!Jsh 13:022}\index[AWIP]{the!Joshua!Jsh 13:022 (3)}\index[AWIP]{children!Joshua!Jsh 13:022}\index[AWIP]{of!Joshua!Jsh 13:022 (2)}\index[AWIP]{Israel!Joshua!Jsh 13:022}\index[AWIP]{slay!Joshua!Jsh 13:022}\index[AWIP]{with!Joshua!Jsh 13:022}\index[AWIP]{the!Joshua!Jsh 13:022 (4)}\index[AWIP]{sword!Joshua!Jsh 13:022}\index[AWIP]{among!Joshua!Jsh 13:022}\index[AWIP]{them!Joshua!Jsh 13:022}\index[AWIP]{that!Joshua!Jsh 13:022}\index[AWIP]{were!Joshua!Jsh 13:022}\index[AWIP]{slain!Joshua!Jsh 13:022}\index[AWIP]{by!Joshua!Jsh 13:022}\index[AWIP]{them!Joshua!Jsh 13:022 (2)}\index[NWIV]{24!Joshua!Jsh 13:022}\index[PNIP]{Balaam!Joshua!Jsh 13:022}\index[PNIP]{Beor!Joshua!Jsh 13:022}\index[PNIP]{Israel!Joshua!Jsh 13:022}\footnote{Notice also the reference to the death of
Balaam here in verse 22. Balaam was the
prophet King Balak of Moab hired to curse the
children of Israel. Those familiar with the
account given in Numbers 23–24 will
remember that Balaam tried three times to curse
the children of Israel, and three times the Holy
Spirit took control of Balaam’s mouth and
made him bless the nation. When Balaam got
through, he prophesied to Balak about the First
and Second Comings of the Lord Jesus Christ.
Now there are all kinds of things we could
go into here with the story of Balaam that we
don’t have the space into which to go. For a
study on that material, we refer you to our work
on The Unknown Bible (1996), chapter one,
and our comments on 2 Peter 2:15; Jude 11;
and Revelation 2:14 in those Commentaries.
Briefly here, Balaam had a “way” (2 Pet. 2:15),
an “error” (Jude 11), and a “doctrine” (Rev.
2:14). His “way” was loving the money Balak
offered him to curse the children of Israel.
Balaam was so anxious to get his hands on “the
wages of unrighteousness” that he deleted
two-thirds of the message God gave him to
preach to Balak (see Num. 22:12–14).
“The error of Balaam” was, having been
dismissed by Balak for being unable to curse
the children of Israel (Num. 24:10–11, 25),
Balaam ended up returning to Balak with a way
to get God Himself to curse Israel. That is why
he was with the Midianites when the Israelites
slew them in battle (Num. 31:8). Balaam
returned to Balak and taught him to get the
children of Israel involved in idolatry and
fornicating with the Midianite women so that
God would plague the people, and that was
“the doctrine of Balaam” (Rev. 2:14 cf. Num.
25:1–9, 31:16).
Do you know what Balaam was? He was a
prophet; he was able to prophesy accurately
about the First and Second Comings of Jesus
Christ. He preached God’s imputed
righteousness (Num. 23:21 cf. Mic. 6:5). He
communed with God. But he ended going to
Hell.
Now what was the problem? Well, first of
all, he loved money. That Bible says, “For the
love of money is the root of all evil: which
while some coveted after, they have erred
from the faith, and pierced themselves
through with many sorrows” (1 Tim. 6:10).
Balaam got “pierced . . . through” literally.
Second, Balaam taught that the “clergy”
was over the “laity” (Rev. 2:15). The common
man had to come to him to get to God. Now
before I give you the third thing wrong with
Balaam, let me ask you a question: Do you
know of any church involved in “venerating
images,” that grows its church membership
through sex (only marry within the church; if
you marry outside of the church you agree to
raise the children “in the faith”; and no abortion
to guarantee your membership grows by
procreation instead of evangelism), and that has
a special “Priest class” through whom laymen
have to come to receive absolution from their
sins? Do you know of a church like that? Can
you name it? I’m not going to tell you; I want to
see if you know any church like that.
That church has special “vestments” for its
clergy (2 Kings 10:22). Its Priests insist on
being called “Father” (Judg. 18:19; Matt. 23:9);
in fact, the head of that church usurps the title
of God the Father for himself (John 17:11). Do you know any church like that? Well, that is the church of a prophet who could subscribe to every point in the ``Apostles' Creed'' and was just as lost as a goose in a horse race. Finally, Balaam had a problem with his companions. Balaam hung around a bunch of fornicating idolaters, and he ended up suffering the same fate they did. The Bible says, ``a companion of fools shall be destroyed'' (Proverbs 13:20). The warning to the saved, bornagain believer in the Body of Christ is: “Be not deceived: evil communications [i.e., those with whom you commune—your friends, your buddies] corrupt good manners [the manner in which you try to live for God]” (1 Cor. 15:33).}
[23] \textcolor[rgb]{0.00,0.00,1.00}{And the border of the children of Reuben was Jordan, and the border \emph{thereof}. This \emph{was} the inheritance of the children of Reuben after their families, the cities and the villages thereof.}\index[AWIP]{And!Joshua!Jsh 13:023}\index[AWIP]{the!Joshua!Jsh 13:023}\index[AWIP]{border!Joshua!Jsh 13:023}\index[AWIP]{of!Joshua!Jsh 13:023}\index[AWIP]{the!Joshua!Jsh 13:023 (2)}\index[AWIP]{children!Joshua!Jsh 13:023}\index[AWIP]{of!Joshua!Jsh 13:023 (2)}\index[AWIP]{Reuben!Joshua!Jsh 13:023}\index[AWIP]{was!Joshua!Jsh 13:023}\index[AWIP]{Jordan!Joshua!Jsh 13:023}\index[AWIP]{and!Joshua!Jsh 13:023}\index[AWIP]{the!Joshua!Jsh 13:023 (3)}\index[AWIP]{border!Joshua!Jsh 13:023 (2)}\index[AWIP]{\emph{thereof}!Joshua!Jsh 13:023}\index[AWIP]{This!Joshua!Jsh 13:023}\index[AWIP]{\emph{was}!Joshua!Jsh 13:023}\index[AWIP]{the!Joshua!Jsh 13:023 (4)}\index[AWIP]{inheritance!Joshua!Jsh 13:023}\index[AWIP]{of!Joshua!Jsh 13:023 (3)}\index[AWIP]{the!Joshua!Jsh 13:023 (5)}\index[AWIP]{children!Joshua!Jsh 13:023 (2)}\index[AWIP]{of!Joshua!Jsh 13:023 (4)}\index[AWIP]{Reuben!Joshua!Jsh 13:023 (2)}\index[AWIP]{after!Joshua!Jsh 13:023}\index[AWIP]{their!Joshua!Jsh 13:023}\index[AWIP]{families!Joshua!Jsh 13:023}\index[AWIP]{the!Joshua!Jsh 13:023 (6)}\index[AWIP]{cities!Joshua!Jsh 13:023}\index[AWIP]{and!Joshua!Jsh 13:023 (2)}\index[AWIP]{the!Joshua!Jsh 13:023 (7)}\index[AWIP]{villages!Joshua!Jsh 13:023}\index[AWIP]{thereof!Joshua!Jsh 13:023}\index[NWIV]{32!Joshua!Jsh 13:023}\index[PNIP]{Jordan!Joshua!Jsh 13:023}\index[PNIP]{Reuben!Joshua!Jsh 13:023}
[24] \textcolor[rgb]{0.00,0.00,1.00}{And Moses gave \emph{inheritance} unto the tribe of Gad, \emph{even} unto the children of Gad according to their families.}\index[AWIP]{And!Joshua!Jsh 13:024}\index[AWIP]{Moses!Joshua!Jsh 13:024}\index[AWIP]{gave!Joshua!Jsh 13:024}\index[AWIP]{\emph{inheritance}!Joshua!Jsh 13:024}\index[AWIP]{unto!Joshua!Jsh 13:024}\index[AWIP]{the!Joshua!Jsh 13:024}\index[AWIP]{tribe!Joshua!Jsh 13:024}\index[AWIP]{of!Joshua!Jsh 13:024}\index[AWIP]{Gad!Joshua!Jsh 13:024}\index[AWIP]{\emph{even}!Joshua!Jsh 13:024}\index[AWIP]{unto!Joshua!Jsh 13:024 (2)}\index[AWIP]{the!Joshua!Jsh 13:024 (2)}\index[AWIP]{children!Joshua!Jsh 13:024}\index[AWIP]{of!Joshua!Jsh 13:024 (2)}\index[AWIP]{Gad!Joshua!Jsh 13:024 (2)}\index[AWIP]{according!Joshua!Jsh 13:024}\index[AWIP]{to!Joshua!Jsh 13:024}\index[AWIP]{their!Joshua!Jsh 13:024}\index[AWIP]{families!Joshua!Jsh 13:024}\index[NWIV]{19!Joshua!Jsh 13:024}\index[PNIP]{Gad!Joshua!Jsh 13:024}\index[PNIP]{Moses!Joshua!Jsh 13:024}\footnote{The next tribe to receive an inheritance on
the east side of Jordan was Gad. Gad’s territory
started where Reuben’s left off: at Heshbon.
This was the land of Gilead, and it went all the
way up to the Jabbok River (Deut. 3:16). In
verse 25, this is called ``half the land of the
children of Ammon.'' Now this territory becomes a point of contention in the book of Judges between the Ammonites and the Gileadites. When the children of Israel finished their wilderness wanderings and came into the land, the Lord told them to leave the land of the Moabites and Ammonites alone (Deuteronomy 2:9, 19). That being the case, what is Gad doing getting ``half the land of the children of Ammon''? That was Ammon’s complaint in Judges 11:13: “And the king of the children of Ammon answered unto the messengers of Jephthah, Because Israel took away my land, when they came up out of Egypt, from Arnon even unto Jabbok, and unto Jordan: now therefore restore those lands again peaceably.” Now that seems like a reasonable request,
doesn’t it? Especially in the light of
Deuteronomy 2:19, 37. But Israel didn’t take
one inch of the land of Ammon under Moses or
Joshua. That took place before Israel ever got
there. “For Heshbon was the city of Sihon the
king of the Amorites, who had fought against
the former king of Moab, and taken all his
land out of his hand, even unto Arnon”
(Num. 21:26).
Moab had taken all of Ammon’s territory
east of Rabbah (vs. 25 cf. Deut. 3:11) clear to
the Jordan River and between the Jabbok and
the Arnon. Sihon came along and took that
same land away from Moab. So when Israel
came through, they took Sihon’s land, not
Ammon’s.
Gad’s territory went from Heshbon to the
Jabbok and then the lowlands alongside the
Jordan River clear up to the Sea of Galilee (vs.
27 cf. Deut. 3:17). “Mahanaim” (vs. 26) is
where Jacob encountered “the angels of God”
upon his return from Haran (Gen. 32:1–2); it is
also where David fled from Absalom (2 Sam.
17:24).}
[25] \textcolor[rgb]{0.00,0.00,1.00}{And their coast was Jazer, and all the cities of Gilead, and half the land of the children of Ammon, unto Aroer that \emph{is} before Rabbah;}\index[AWIP]{And!Joshua!Jsh 13:025}\index[AWIP]{their!Joshua!Jsh 13:025}\index[AWIP]{coast!Joshua!Jsh 13:025}\index[AWIP]{was!Joshua!Jsh 13:025}\index[AWIP]{Jazer!Joshua!Jsh 13:025}\index[AWIP]{and!Joshua!Jsh 13:025}\index[AWIP]{all!Joshua!Jsh 13:025}\index[AWIP]{the!Joshua!Jsh 13:025}\index[AWIP]{cities!Joshua!Jsh 13:025}\index[AWIP]{of!Joshua!Jsh 13:025}\index[AWIP]{Gilead!Joshua!Jsh 13:025}\index[AWIP]{and!Joshua!Jsh 13:025 (2)}\index[AWIP]{half!Joshua!Jsh 13:025}\index[AWIP]{the!Joshua!Jsh 13:025 (2)}\index[AWIP]{land!Joshua!Jsh 13:025}\index[AWIP]{of!Joshua!Jsh 13:025 (2)}\index[AWIP]{the!Joshua!Jsh 13:025 (3)}\index[AWIP]{children!Joshua!Jsh 13:025}\index[AWIP]{of!Joshua!Jsh 13:025 (3)}\index[AWIP]{Ammon!Joshua!Jsh 13:025}\index[AWIP]{unto!Joshua!Jsh 13:025}\index[AWIP]{Aroer!Joshua!Jsh 13:025}\index[AWIP]{that!Joshua!Jsh 13:025}\index[AWIP]{\emph{is}!Joshua!Jsh 13:025}\index[AWIP]{before!Joshua!Jsh 13:025}\index[AWIP]{Rabbah!Joshua!Jsh 13:025}\index[NWIV]{26!Joshua!Jsh 13:025}\index[PNIP]{Gilead!Joshua!Jsh 13:025}\index[PNIP]{Aroer!Joshua!Jsh 13:025}\index[PNIP]{Ammon!Joshua!Jsh 13:025}\index[PNIP]{Jazer!Joshua!Jsh 13:025}\index[PNIP]{Rabbah!Joshua!Jsh 13:025}
[26] \textcolor[rgb]{0.00,0.00,1.00}{And from Heshbon unto Ramath-mizpeh, and Betonim; and from Mahanaim unto the border of Debir;}\index[AWIP]{And!Joshua!Jsh 13:026}\index[AWIP]{from!Joshua!Jsh 13:026}\index[AWIP]{Heshbon!Joshua!Jsh 13:026}\index[AWIP]{unto!Joshua!Jsh 13:026}\index[AWIP]{Ramath-mizpeh!Joshua!Jsh 13:026}\index[AWIP]{and!Joshua!Jsh 13:026}\index[AWIP]{Betonim!Joshua!Jsh 13:026}\index[AWIP]{and!Joshua!Jsh 13:026 (2)}\index[AWIP]{from!Joshua!Jsh 13:026 (2)}\index[AWIP]{Mahanaim!Joshua!Jsh 13:026}\index[AWIP]{unto!Joshua!Jsh 13:026 (2)}\index[AWIP]{the!Joshua!Jsh 13:026}\index[AWIP]{border!Joshua!Jsh 13:026}\index[AWIP]{of!Joshua!Jsh 13:026}\index[AWIP]{Debir!Joshua!Jsh 13:026}\index[NWIV]{15!Joshua!Jsh 13:026}\index[PNIP]{Heshbon!Joshua!Jsh 13:026}\index[PNIP]{Ramath-mizpeh!Joshua!Jsh 13:026}\index[PNIP]{Betonim!Joshua!Jsh 13:026}\index[PNIP]{Mahanaim!Joshua!Jsh 13:026}\index[PNIP]{Debir!Joshua!Jsh 13:026}
[27] \textcolor[rgb]{0.00,0.00,1.00}{And in the valley, Beth-aram, and Beth-nimrah, and Succoth, and Zaphon, the rest of the kingdom of Sihon king of Heshbon, Jordan and \emph{his} border, \emph{even} unto the edge of the sea of Chinnereth on the other side Jordan eastward.}\index[AWIP]{And!Joshua!Jsh 13:027}\index[AWIP]{in!Joshua!Jsh 13:027}\index[AWIP]{the!Joshua!Jsh 13:027}\index[AWIP]{valley!Joshua!Jsh 13:027}\index[AWIP]{Beth-aram!Joshua!Jsh 13:027}\index[AWIP]{and!Joshua!Jsh 13:027}\index[AWIP]{Beth-nimrah!Joshua!Jsh 13:027}\index[AWIP]{and!Joshua!Jsh 13:027 (2)}\index[AWIP]{Succoth!Joshua!Jsh 13:027}\index[AWIP]{and!Joshua!Jsh 13:027 (3)}\index[AWIP]{Zaphon!Joshua!Jsh 13:027}\index[AWIP]{the!Joshua!Jsh 13:027 (2)}\index[AWIP]{rest!Joshua!Jsh 13:027}\index[AWIP]{of!Joshua!Jsh 13:027}\index[AWIP]{the!Joshua!Jsh 13:027 (3)}\index[AWIP]{kingdom!Joshua!Jsh 13:027}\index[AWIP]{of!Joshua!Jsh 13:027 (2)}\index[AWIP]{Sihon!Joshua!Jsh 13:027}\index[AWIP]{king!Joshua!Jsh 13:027}\index[AWIP]{of!Joshua!Jsh 13:027 (3)}\index[AWIP]{Heshbon!Joshua!Jsh 13:027}\index[AWIP]{Jordan!Joshua!Jsh 13:027}\index[AWIP]{and!Joshua!Jsh 13:027 (4)}\index[AWIP]{\emph{his}!Joshua!Jsh 13:027}\index[AWIP]{border!Joshua!Jsh 13:027}\index[AWIP]{\emph{even}!Joshua!Jsh 13:027}\index[AWIP]{unto!Joshua!Jsh 13:027}\index[AWIP]{the!Joshua!Jsh 13:027 (4)}\index[AWIP]{edge!Joshua!Jsh 13:027}\index[AWIP]{of!Joshua!Jsh 13:027 (4)}\index[AWIP]{the!Joshua!Jsh 13:027 (5)}\index[AWIP]{sea!Joshua!Jsh 13:027}\index[AWIP]{of!Joshua!Jsh 13:027 (5)}\index[AWIP]{Chinnereth!Joshua!Jsh 13:027}\index[AWIP]{on!Joshua!Jsh 13:027}\index[AWIP]{the!Joshua!Jsh 13:027 (6)}\index[AWIP]{other!Joshua!Jsh 13:027}\index[AWIP]{side!Joshua!Jsh 13:027}\index[AWIP]{Jordan!Joshua!Jsh 13:027 (2)}\index[AWIP]{eastward!Joshua!Jsh 13:027}\index[NWIV]{40!Joshua!Jsh 13:027}\index[PNIP]{Heshbon!Joshua!Jsh 13:027}\index[PNIP]{Jordan!Joshua!Jsh 13:027}\index[PNIP]{Sihon!Joshua!Jsh 13:027}\index[PNIP]{Succoth!Joshua!Jsh 13:027}\index[PNIP]{Beth-aram!Joshua!Jsh 13:027}\index[PNIP]{Beth-nimrah!Joshua!Jsh 13:027}\index[PNIP]{Zaphon!Joshua!Jsh 13:027}\index[PNIP]{Chinnereth!Joshua!Jsh 13:027}
[28] \textcolor[rgb]{0.00,0.00,1.00}{This \emph{is} the inheritance of the children of Gad after their families, the cities, and their villages.}\index[AWIP]{This!Joshua!Jsh 13:028}\index[AWIP]{\emph{is}!Joshua!Jsh 13:028}\index[AWIP]{the!Joshua!Jsh 13:028}\index[AWIP]{inheritance!Joshua!Jsh 13:028}\index[AWIP]{of!Joshua!Jsh 13:028}\index[AWIP]{the!Joshua!Jsh 13:028 (2)}\index[AWIP]{children!Joshua!Jsh 13:028}\index[AWIP]{of!Joshua!Jsh 13:028 (2)}\index[AWIP]{Gad!Joshua!Jsh 13:028}\index[AWIP]{after!Joshua!Jsh 13:028}\index[AWIP]{their!Joshua!Jsh 13:028}\index[AWIP]{families!Joshua!Jsh 13:028}\index[AWIP]{the!Joshua!Jsh 13:028 (3)}\index[AWIP]{cities!Joshua!Jsh 13:028}\index[AWIP]{and!Joshua!Jsh 13:028}\index[AWIP]{their!Joshua!Jsh 13:028 (2)}\index[AWIP]{villages!Joshua!Jsh 13:028}\index[NWIV]{17!Joshua!Jsh 13:028}\index[PNIP]{Gad!Joshua!Jsh 13:028}\\
\\
\P \textcolor[rgb]{0.00,0.00,1.00}{And Moses gave \emph{inheritance} unto the half tribe of Manasseh: and \emph{this} was \emph{the} \emph{possession} of the half tribe of the children of Manasseh by their families.}\index[AWIP]{And!Joshua!Jsh 13:029}\index[AWIP]{Moses!Joshua!Jsh 13:029}\index[AWIP]{gave!Joshua!Jsh 13:029}\index[AWIP]{\emph{inheritance}!Joshua!Jsh 13:029}\index[AWIP]{unto!Joshua!Jsh 13:029}\index[AWIP]{the!Joshua!Jsh 13:029}\index[AWIP]{half!Joshua!Jsh 13:029}\index[AWIP]{tribe!Joshua!Jsh 13:029}\index[AWIP]{of!Joshua!Jsh 13:029}\index[AWIP]{Manasseh!Joshua!Jsh 13:029}\index[AWIP]{and!Joshua!Jsh 13:029}\index[AWIP]{\emph{this}!Joshua!Jsh 13:029}\index[AWIP]{was!Joshua!Jsh 13:029}\index[AWIP]{\emph{the}!Joshua!Jsh 13:029}\index[AWIP]{\emph{possession}!Joshua!Jsh 13:029}\index[AWIP]{of!Joshua!Jsh 13:029 (2)}\index[AWIP]{the!Joshua!Jsh 13:029 (2)}\index[AWIP]{half!Joshua!Jsh 13:029 (2)}\index[AWIP]{tribe!Joshua!Jsh 13:029 (2)}\index[AWIP]{of!Joshua!Jsh 13:029 (3)}\index[AWIP]{the!Joshua!Jsh 13:029 (3)}\index[AWIP]{children!Joshua!Jsh 13:029}\index[AWIP]{of!Joshua!Jsh 13:029 (4)}\index[AWIP]{Manasseh!Joshua!Jsh 13:029 (2)}\index[AWIP]{by!Joshua!Jsh 13:029}\index[AWIP]{their!Joshua!Jsh 13:029}\index[AWIP]{families!Joshua!Jsh 13:029}\index[NWIV]{27!Joshua!Jsh 13:029}\index[PNIP]{Manasseh!Joshua!Jsh 13:029}\index[PNIP]{Moses!Joshua!Jsh 13:029}\footnote{The final inheritance on the east side of Jordan was given to half the tribe of Manasseh, specifically the sons of Machir (vs. 31 cf. Deut. 3:15), the eldest son of Manasseh (Josh. 17:1). (There is some question as to whether Machir was the only son of Manasseh [see 1 Chron. 7:14–15]; notice the wording in vs. 31—“even to the one half of the children of Machir by their families.”) The half tribe of Manasseh’s inheritance was from Mahanaim to the Yarmuk River, which was the rest of the land of Gilead (vs. 31); as well as the territory north of the Yarmuk to Mt. Hermon, which was the land of Bashan where Og ruled.}
[30] \textcolor[rgb]{0.00,0.00,1.00}{And their coast was from Mahanaim, all Bashan, all the kingdom of Og king of Bashan, and all the towns of Jair, which \emph{are} in Bashan, threescore cities:}\index[AWIP]{And!Joshua!Jsh 13:030}\index[AWIP]{their!Joshua!Jsh 13:030}\index[AWIP]{coast!Joshua!Jsh 13:030}\index[AWIP]{was!Joshua!Jsh 13:030}\index[AWIP]{from!Joshua!Jsh 13:030}\index[AWIP]{Mahanaim!Joshua!Jsh 13:030}\index[AWIP]{all!Joshua!Jsh 13:030}\index[AWIP]{Bashan!Joshua!Jsh 13:030}\index[AWIP]{all!Joshua!Jsh 13:030 (2)}\index[AWIP]{the!Joshua!Jsh 13:030}\index[AWIP]{kingdom!Joshua!Jsh 13:030}\index[AWIP]{of!Joshua!Jsh 13:030}\index[AWIP]{Og!Joshua!Jsh 13:030}\index[AWIP]{king!Joshua!Jsh 13:030}\index[AWIP]{of!Joshua!Jsh 13:030 (2)}\index[AWIP]{Bashan!Joshua!Jsh 13:030 (2)}\index[AWIP]{and!Joshua!Jsh 13:030}\index[AWIP]{all!Joshua!Jsh 13:030 (3)}\index[AWIP]{the!Joshua!Jsh 13:030 (2)}\index[AWIP]{towns!Joshua!Jsh 13:030}\index[AWIP]{of!Joshua!Jsh 13:030 (3)}\index[AWIP]{Jair!Joshua!Jsh 13:030}\index[AWIP]{which!Joshua!Jsh 13:030}\index[AWIP]{\emph{are}!Joshua!Jsh 13:030}\index[AWIP]{in!Joshua!Jsh 13:030}\index[AWIP]{Bashan!Joshua!Jsh 13:030 (3)}\index[AWIP]{threescore!Joshua!Jsh 13:030}\index[AWIP]{cities!Joshua!Jsh 13:030}\index[NWIV]{28!Joshua!Jsh 13:030}\index[PNIP]{Bashan!Joshua!Jsh 13:030}\index[PNIP]{Og!Joshua!Jsh 13:030}\index[PNIP]{Mahanaim!Joshua!Jsh 13:030}\index[PNIP]{Jair!Joshua!Jsh 13:030}\footnote{Notice “all the towns of Jair” in verse 30.
Jair was a grandson of Machir through his
daughter. He already had 23 small villages of
his own in Gilead which he took in battle (Num.
32:41 cf. 1 Chron. 2:22), but he went up into
Bashan and took seventy more villages in the
region of Argob (vs. 30 cf. Deut. 3:14). Jair’s
“investment in real estate” paid off. By the time
of Solomon, the villages Jair took in Bashan
(havoth is another Hebrew word for “villages”)
grew into “great cities with walls and brasen
bars” (1 Kings 4:13).
It reminds one of the desolation and
barrenness of Palestine under Moslem control,
and the fertility and productivity of the land of
Israel under the Jewish State. A Russian and a
Jew were traveling together on a train once, and
they got to talking. The Russian said to the Jew,
“We don’t have any Jews in my village.” To
which the Jew replied, “That’s vhy it’s still a
village.”
Those Israelis came back to Palestine and
rebuilt the cities of ancient Israel where only
some small Turkish villages or Bedouin tents
stood before. They reclaimed hundreds of acres
of arid land by irrigation, and where once was
what Mark Twain described as “desolate
country . . . given over wholly to weeds” and
“worthless soil” (The Innocents Abroad, 1881)
are now cultivated fields bearing all kinds of
crops. The Turks never did with the land what
those Jews did when they returned.
Where once was “hardly a tree or shrub
anywhere” (Ibid.), now stand forests of trees
planted there by the Israelis, not the
“Palestinians.” One of the reasons for the
cultivation and reforestation is due to the
increase of rainfall. When God sent those Jews
back to the land He gave them in Joshua, He
increased the rainfall from 2.1 inches annually
(1931–1960) to 21.1 inches annually (1960–
1980).
That land has changed so much since “the
fig tree” budded in 1948 (Jer. 8:11–13 cf.
Mark 13:28–29; Luke 21:29–32) that beasts
and birds that were missing for centuries have
returned. Vultures are now nesting in Israel
after fifteen centuries. Wild leopards have
shown back up around Engedi, and herds of
wild goats (ibex) are showing back up.
Now do you know what all that shows? It’s
more than just that the Jews are more
industrious than the Moslems (which they are),
or that God blessed those Jews when they
returned to the land (which He did). Those are
all signs pointing to Christ’s soon return. Of
course, since the signs are for Israel (1 Cor.
1:22), those are signs for the Second Advent,
not the Rapture. But the Christian can read the
signs for the Advent and know that the Rapture
is near (1 Thess. 5:1–11), since the Rapture
precedes the Advent.}
[31] \textcolor[rgb]{0.00,0.00,1.00}{And half Gilead, and Ashtaroth, and Edrei, cities of the kingdom of Og in Bashan, \emph{were} \emph{pertaining} unto the children of Machir the son of Manasseh, \emph{even} to the one half of the children of Machir by their families.}\index[AWIP]{And!Joshua!Jsh 13:031}\index[AWIP]{half!Joshua!Jsh 13:031}\index[AWIP]{Gilead!Joshua!Jsh 13:031}\index[AWIP]{and!Joshua!Jsh 13:031}\index[AWIP]{Ashtaroth!Joshua!Jsh 13:031}\index[AWIP]{and!Joshua!Jsh 13:031 (2)}\index[AWIP]{Edrei!Joshua!Jsh 13:031}\index[AWIP]{cities!Joshua!Jsh 13:031}\index[AWIP]{of!Joshua!Jsh 13:031}\index[AWIP]{the!Joshua!Jsh 13:031}\index[AWIP]{kingdom!Joshua!Jsh 13:031}\index[AWIP]{of!Joshua!Jsh 13:031 (2)}\index[AWIP]{Og!Joshua!Jsh 13:031}\index[AWIP]{in!Joshua!Jsh 13:031}\index[AWIP]{Bashan!Joshua!Jsh 13:031}\index[AWIP]{\emph{were}!Joshua!Jsh 13:031}\index[AWIP]{\emph{pertaining}!Joshua!Jsh 13:031}\index[AWIP]{unto!Joshua!Jsh 13:031}\index[AWIP]{the!Joshua!Jsh 13:031 (2)}\index[AWIP]{children!Joshua!Jsh 13:031}\index[AWIP]{of!Joshua!Jsh 13:031 (3)}\index[AWIP]{Machir!Joshua!Jsh 13:031}\index[AWIP]{the!Joshua!Jsh 13:031 (3)}\index[AWIP]{son!Joshua!Jsh 13:031}\index[AWIP]{of!Joshua!Jsh 13:031 (4)}\index[AWIP]{Manasseh!Joshua!Jsh 13:031}\index[AWIP]{\emph{even}!Joshua!Jsh 13:031}\index[AWIP]{to!Joshua!Jsh 13:031}\index[AWIP]{the!Joshua!Jsh 13:031 (4)}\index[AWIP]{one!Joshua!Jsh 13:031}\index[AWIP]{half!Joshua!Jsh 13:031 (2)}\index[AWIP]{of!Joshua!Jsh 13:031 (5)}\index[AWIP]{the!Joshua!Jsh 13:031 (5)}\index[AWIP]{children!Joshua!Jsh 13:031 (2)}\index[AWIP]{of!Joshua!Jsh 13:031 (6)}\index[AWIP]{Machir!Joshua!Jsh 13:031 (2)}\index[AWIP]{by!Joshua!Jsh 13:031}\index[AWIP]{their!Joshua!Jsh 13:031}\index[AWIP]{families!Joshua!Jsh 13:031}\index[NWIV]{39!Joshua!Jsh 13:031}\index[PNIP]{Gilead!Joshua!Jsh 13:031}\index[PNIP]{Machir!Joshua!Jsh 13:031}\index[PNIP]{Manasseh!Joshua!Jsh 13:031}\index[PNIP]{Bashan!Joshua!Jsh 13:031}\index[PNIP]{Og!Joshua!Jsh 13:031}\index[PNIP]{Ashtaroth!Joshua!Jsh 13:031}\index[PNIP]{Edrei!Joshua!Jsh 13:031}
[32] \textcolor[rgb]{0.00,0.00,1.00}{These \emph{are} \emph{the} \emph{countries} which Moses did distribute for inheritance in the plains of Moab, on the other side Jordan, by Jericho, eastward.}\index[AWIP]{These!Joshua!Jsh 13:032}\index[AWIP]{\emph{are}!Joshua!Jsh 13:032}\index[AWIP]{\emph{the}!Joshua!Jsh 13:032}\index[AWIP]{\emph{countries}!Joshua!Jsh 13:032}\index[AWIP]{which!Joshua!Jsh 13:032}\index[AWIP]{Moses!Joshua!Jsh 13:032}\index[AWIP]{did!Joshua!Jsh 13:032}\index[AWIP]{distribute!Joshua!Jsh 13:032}\index[AWIP]{for!Joshua!Jsh 13:032}\index[AWIP]{inheritance!Joshua!Jsh 13:032}\index[AWIP]{in!Joshua!Jsh 13:032}\index[AWIP]{the!Joshua!Jsh 13:032}\index[AWIP]{plains!Joshua!Jsh 13:032}\index[AWIP]{of!Joshua!Jsh 13:032}\index[AWIP]{Moab!Joshua!Jsh 13:032}\index[AWIP]{on!Joshua!Jsh 13:032}\index[AWIP]{the!Joshua!Jsh 13:032 (2)}\index[AWIP]{other!Joshua!Jsh 13:032}\index[AWIP]{side!Joshua!Jsh 13:032}\index[AWIP]{Jordan!Joshua!Jsh 13:032}\index[AWIP]{by!Joshua!Jsh 13:032}\index[AWIP]{Jericho!Joshua!Jsh 13:032}\index[AWIP]{eastward!Joshua!Jsh 13:032}\index[NWIV]{23!Joshua!Jsh 13:032}\index[PNIP]{Jordan!Joshua!Jsh 13:032}\index[PNIP]{Moab!Joshua!Jsh 13:032}\index[PNIP]{Moses!Joshua!Jsh 13:032}\index[PNIP]{Jericho!Joshua!Jsh 13:032}
[33] \textcolor[rgb]{0.00,0.00,1.00}{But unto the tribe of Levi Moses gave not \emph{any} inheritance: the LORD God of Israel \emph{was} their inheritance, as he said unto them.}\index[AWIP]{But!Joshua!Jsh 13:033}\index[AWIP]{unto!Joshua!Jsh 13:033}\index[AWIP]{the!Joshua!Jsh 13:033}\index[AWIP]{tribe!Joshua!Jsh 13:033}\index[AWIP]{of!Joshua!Jsh 13:033}\index[AWIP]{Levi!Joshua!Jsh 13:033}\index[AWIP]{Moses!Joshua!Jsh 13:033}\index[AWIP]{gave!Joshua!Jsh 13:033}\index[AWIP]{not!Joshua!Jsh 13:033}\index[AWIP]{\emph{any}!Joshua!Jsh 13:033}\index[AWIP]{inheritance!Joshua!Jsh 13:033}\index[AWIP]{the!Joshua!Jsh 13:033 (2)}\index[AWIP]{LORD!Joshua!Jsh 13:033}\index[AWIP]{God!Joshua!Jsh 13:033}\index[AWIP]{of!Joshua!Jsh 13:033 (2)}\index[AWIP]{Israel!Joshua!Jsh 13:033}\index[AWIP]{\emph{was}!Joshua!Jsh 13:033}\index[AWIP]{their!Joshua!Jsh 13:033}\index[AWIP]{inheritance!Joshua!Jsh 13:033 (2)}\index[AWIP]{as!Joshua!Jsh 13:033}\index[AWIP]{he!Joshua!Jsh 13:033}\index[AWIP]{said!Joshua!Jsh 13:033}\index[AWIP]{unto!Joshua!Jsh 13:033 (2)}\index[AWIP]{them!Joshua!Jsh 13:033}\index[NWIV]{24!Joshua!Jsh 13:033}\index[PNIP]{God!Joshua!Jsh 13:033}\index[PNIP]{Israel!Joshua!Jsh 13:033}\index[PNIP]{LORD!Joshua!Jsh 13:033}\index[PNIP]{Levi!Joshua!Jsh 13:033}\index[PNIP]{Moses!Joshua!Jsh 13:033}

