\chapter{Joshua 16}

\marginpar{\scriptsize \centering \fcolorbox{bone}{lime}{{\textbf{LAND FOR THE SONS OF JOSEPH}}}\\ (Joshua 16:1-10) \begin{compactenum}[I.][8]
	\item The \textbf{Sons}  \index[scripture]{Joshua!Jsh 16:01}  (Jsh 16:1) 
	\item \textbf{Sides of the Compass}  \index[scripture]{Joshua!Jsh 16:02--03}  (Jsh 16:2--3) 
	\item \textbf{Shiloh}  \index[scripture]{Joshua!Jsh 16:06}  (Jsh 16:6) 
	\item The \textbf{Sea} as a Border \index[scripture]{Joshua!Jsh 16:08}  (Jsh 16:8) 
	\item The \textbf{Separate Cities}  \index[scripture]{Joshua!Jsh 16:09}  (Jsh 16:9) 
	\item \textbf{Squatters}  \index[scripture]{Joshua!Jsh 16:10}  (Jsh 16:10) 
	\item Like \textbf{Sin}  \index[scripture]{Joshua!Jsh 16:10}  (Jsh 16:10) 
\end{compactenum}}



\footnote{\textcolor[cmyk]{0.99998,1,0,0}{\hyperlink{TOC}{Return to end of Table of Contents.}}}\footnote{\href{https://audiobible.com/bible/joshua_16.html}{\textcolor[cmyk]{0.99998,1,0,0}{Joshua 16 Audio}}}\textcolor[cmyk]{0.99998,1,0,0}{And the lot of the \fcolorbox{bone}{lime}{children of Joseph} fell from Jordan by Jericho, unto the water of Jericho on the east, to the wilderness that goeth up from Jericho throughout mount Beth-el,} %\footnote{The passage introduces the inheritance of Joseph’s two sons, Manasseh and Ephraim. We have already discussed the double portion of Joseph in our comments on 14:4, which see. The eastern border of Ephraim went from the spring of Jericho over to Jordan (vs. 1; Jericho proper belonging to Benjamin--18:21) and up to “Ataroth” (vs. 2; called “Atarothaddar” in vs. 5). The southern border went from the spring of Jericho up to “Bethel” (vs. 2) and westward over to “Bethhoron” (the lower and the upper—vss. 3, 5), and then down to “Gezer” (vs. 3). “The water of Jericho” (vs. 1) was the bitter spring of the city that was later healed by Elisha (2 Kings 2:19–21). “Bethel to Luz” (vs. 2) is peculiar wording since Bethel was Luz (Judg. 1:23). But in Genesis 28:19, the distinction is made between the place where Jacob slept on the stone and had the vision of the ladder up to Heaven (Bethel) and the city near the place (Luz). \cite{Ruckman2014Joshua} }
[2] \textcolor[cmyk]{0.99998,1,0,0}{And goeth out from Beth-el to Luz, and passeth along unto the borders of Archi to Ataroth,} %\footnote{“Archi” is the town from which David’s counselor Hushai, who acted as a “double agent” during Absalom’s rebellion, came (2 Sam. 16:16). According to 1 Chronicles 7:22--24, the two Bethhorons (vss. 3, 5) belonged to the tribe of Ephraim because it was Ephraim’s daughter (“Sherah”) who founded the cities. \cite{ruckman2014Joshua,}}
[3] \textcolor[cmyk]{0.99998,1,0,0}{And goeth down westward to the coast of Japhleti, unto the coast of Beth-horon the nether, and to Gezer: and the goings out thereof are at the sea.}
[4] \textcolor[cmyk]{0.99998,1,0,0}{So the children of Joseph, Manasseh and Ephraim, took their inheritance.}\\
\\
\P \textcolor[cmyk]{0.99998,1,0,0}{And the border of the children of Ephraim according to their families was \emph{thus}: even the border of their inheritance on the east side was Ataroth-addar, unto Beth-horon the upper;}
[6] \textcolor[cmyk]{0.99998,1,0,0}{And the border went out toward the sea to Michmethah on the north side; and the border went about eastward unto \fcolorbox{bone}{lime}{{Taanath-shiloh}, and passed by it on the east to Janohah;} %\footnote{“Michmethah” (vs. 6) was a place on the northern boundary of Ephraim that bordered Manasseh (17:7). “Taanathshiloh” (vs. 6) is a fuller name for Shiloh where the Tabernacle was set up (18:1).  \cite{Ruckman2014Joshua} }
[7] \textcolor[cmyk]{0.99998,1,0,0}{And it went down from Janohah to Ataroth, and to Naarath, and came to Jericho, and went out at Jordan.}
[8] \textcolor[cmyk]{0.99998,1,0,0}{The border went out from Tappuah westward unto the river Kanah; and the goings out thereof were at the \fcolorbox{bone}{lime}{{sea}. This \emph{is} the inheritance of the tribe of the children of Ephraim by their families.}
[9] \textcolor[cmyk]{0.99998,1,0,0}{And the \fcolorbox{bone}{lime}{{separate cities} for the children of Ephraim \emph{were} among the inheritance of the children of Manasseh, all the cities with their villages.} %\footnote{“Tappuah” (vs. 8) was both the name of a region and a city within that region. “The land of Tappuah” belonged to Manasseh, but the city of Tappuah belonged to Ephraim (17:8). At Tappuah, the northern border of Ephraim ran along the Kanah River all the way to the Mediterranean (vs. 8). According to 17:9, all the cities south of the Kanah belonged to Ephraim, all the cities north of the river belonged to Manasseh. And as we saw with Tappuah above, there were certain cities within the boundaries of Manasseh that actually belonged to Ephraim (vs. 9).}
[10] \textcolor[cmyk]{0.99998,1,0,0}{And they drave not out the Canaanites that dwelt in Gezer: but the \fcolorbox{bone}{lime}{{Canaanites dwell} among the Ephraimites unto this day, and serve under tribute.} %\footnote{Verse 10 gives you the incomplete obedience of Ephraim in not driving out the Canaanites in Gezer (see our comments on 13:13 and 15:63). In fact, when God finally gets rid of the Canaanites in Gezer, He has to use a heathen King to do it (1 Kings 9:16). Ephraim doesn’t do it (Judg. 1:29); none of the judges do it; Saul doesn’t do it; David doesn’t do it; and Solomon doesn’t do it. It is Pharaoh who does it, and he does it as a “wedding gift” for his daughter who married Solomon. The spiritual lesson for the Christian’s warfare is to deal with sin and worldliness. Fight against it. Don’t give it “aid and comfort”; don’t allow it a place in your life. When the Holy Spirit exposes an idol in your life, break it down (2 Cor. 10:5). If you don’t, it’s going to keep popping up and bothering you, and eventually get control of you if you let it. There isn’t any way of dealing with it other than fighting. I wish every Christian could read some of the anecdotes I have read from military history. There is account after account, and story after story, of brave men who fought, sometimes against impossible odds, and for what? Absolutely nothing. Those Germans in World War II fought for a demon-possessed madman. The average German solder wasn’t a Nazi; he was just fighting for the “Fatherland.” But there were a lot of brave soldiers who fought for Hitler at places like Stalingrad, where only one in ten made it back alive. Our boys in Vietnam fought for a fornicating Catholic (JFK). You have young Americans dying and being maimed right now (as of this writing) for a Marxist Moslem (Obama). Well, they’re not Marxist or Moslem, for the greater part; a lot of them are brave young men just following orders. But they fight. Now if someone can do that for a lost, Hell-bound sinner, what’s your excuse, Christian? Haven’t you got a better Commander? I think so! Haven’t you got a better cause! I think so! THEN FIGHT!  \cite{Ruckman2014Joshua} }

