\chapter{Joshua 14}
\footnote{\textcolor[rgb]{0.00,0.25,0.00}{\hyperlink{TOC}{Return to end of Table of Contents.}}}\textcolor[rgb]{0.00,0.00,1.00}{And these \emph{are} \emph{the} \emph{countries} which the children of Israel inherited in the land of Canaan, which Eleazar the priest, and Joshua the son of Nun, and the heads of the fathers of the tribes of the children of Israel, distributed for inheritance to them.}\index[AWIP]{And!Joshua!Jsh 14:001}\index[AWIP]{these!Joshua!Jsh 14:001}\index[AWIP]{\emph{are}!Joshua!Jsh 14:001}\index[AWIP]{\emph{the}!Joshua!Jsh 14:001}\index[AWIP]{\emph{countries}!Joshua!Jsh 14:001}\index[AWIP]{which!Joshua!Jsh 14:001}\index[AWIP]{the!Joshua!Jsh 14:001}\index[AWIP]{children!Joshua!Jsh 14:001}\index[AWIP]{of!Joshua!Jsh 14:001}\index[AWIP]{Israel!Joshua!Jsh 14:001}\index[AWIP]{inherited!Joshua!Jsh 14:001}\index[AWIP]{in!Joshua!Jsh 14:001}\index[AWIP]{the!Joshua!Jsh 14:001 (2)}\index[AWIP]{land!Joshua!Jsh 14:001}\index[AWIP]{of!Joshua!Jsh 14:001 (2)}\index[AWIP]{Canaan!Joshua!Jsh 14:001}\index[AWIP]{which!Joshua!Jsh 14:001 (2)}\index[AWIP]{Eleazar!Joshua!Jsh 14:001}\index[AWIP]{the!Joshua!Jsh 14:001 (3)}\index[AWIP]{priest!Joshua!Jsh 14:001}\index[AWIP]{and!Joshua!Jsh 14:001}\index[AWIP]{Joshua!Joshua!Jsh 14:001}\index[AWIP]{the!Joshua!Jsh 14:001 (4)}\index[AWIP]{son!Joshua!Jsh 14:001}\index[AWIP]{of!Joshua!Jsh 14:001 (3)}\index[AWIP]{Nun!Joshua!Jsh 14:001}\index[AWIP]{and!Joshua!Jsh 14:001 (2)}\index[AWIP]{the!Joshua!Jsh 14:001 (5)}\index[AWIP]{heads!Joshua!Jsh 14:001}\index[AWIP]{of!Joshua!Jsh 14:001 (4)}\index[AWIP]{the!Joshua!Jsh 14:001 (6)}\index[AWIP]{fathers!Joshua!Jsh 14:001}\index[AWIP]{of!Joshua!Jsh 14:001 (5)}\index[AWIP]{the!Joshua!Jsh 14:001 (7)}\index[AWIP]{tribes!Joshua!Jsh 14:001}\index[AWIP]{of!Joshua!Jsh 14:001 (6)}\index[AWIP]{the!Joshua!Jsh 14:001 (8)}\index[AWIP]{children!Joshua!Jsh 14:001 (2)}\index[AWIP]{of!Joshua!Jsh 14:001 (7)}\index[AWIP]{Israel!Joshua!Jsh 14:001 (2)}\index[AWIP]{distributed!Joshua!Jsh 14:001}\index[AWIP]{for!Joshua!Jsh 14:001}\index[AWIP]{inheritance!Joshua!Jsh 14:001}\index[AWIP]{to!Joshua!Jsh 14:001}\index[AWIP]{them!Joshua!Jsh 14:001}\index[NWIV]{45!Joshua!Jsh 14:001}\index[PNIP]{Canaan!Joshua!Jsh 14:001}\index[PNIP]{Eleazar!Joshua!Jsh 14:001}\index[PNIP]{Israel!Joshua!Jsh 14:001}\index[PNIP]{Joshua!Joshua!Jsh 14:001}\index[PNIP]{Nun!Joshua!Jsh 14:001}
[2] \textcolor[rgb]{0.00,0.00,1.00}{By lot \emph{was} their inheritance, as the LORD commanded by the hand of Moses, for the nine tribes, and \emph{for} the half tribe.}\index[AWIP]{By!Joshua!Jsh 14:002}\index[AWIP]{lot!Joshua!Jsh 14:002}\index[AWIP]{\emph{was}!Joshua!Jsh 14:002}\index[AWIP]{their!Joshua!Jsh 14:002}\index[AWIP]{inheritance!Joshua!Jsh 14:002}\index[AWIP]{as!Joshua!Jsh 14:002}\index[AWIP]{the!Joshua!Jsh 14:002}\index[AWIP]{LORD!Joshua!Jsh 14:002}\index[AWIP]{commanded!Joshua!Jsh 14:002}\index[AWIP]{by!Joshua!Jsh 14:002}\index[AWIP]{the!Joshua!Jsh 14:002 (2)}\index[AWIP]{hand!Joshua!Jsh 14:002}\index[AWIP]{of!Joshua!Jsh 14:002}\index[AWIP]{Moses!Joshua!Jsh 14:002}\index[AWIP]{for!Joshua!Jsh 14:002}\index[AWIP]{the!Joshua!Jsh 14:002 (3)}\index[AWIP]{nine!Joshua!Jsh 14:002}\index[AWIP]{tribes!Joshua!Jsh 14:002}\index[AWIP]{and!Joshua!Jsh 14:002}\index[AWIP]{\emph{for}!Joshua!Jsh 14:002}\index[AWIP]{the!Joshua!Jsh 14:002 (4)}\index[AWIP]{half!Joshua!Jsh 14:002}\index[AWIP]{tribe!Joshua!Jsh 14:002}\index[NWIV]{23!Joshua!Jsh 14:002}\index[PNIP]{LORD!Joshua!Jsh 14:002}\index[PNIP]{Moses!Joshua!Jsh 14:002}\footnote{Now chapter 13 dealt with the division of the land on the east side of Jordan to Reuben, Gad, and the half tribe of Manasseh (vs. 3). Chapter 14 introduces the division of Canaan proper to the other nine and a half tribes (vss. 1–2). Verse 2 says, “By lot was their inheritance.” We have run into a form of the lot before back under 7:14 (which see). In Bible times, lots were either drawn like straws or cast like dice. The procedure was to place several white stones in a vessel along with a black stone. Then the parties involved would step forward and draw a stone out of the vessel. Whoever got the black stone was the chosen party. That was the method by which lots were drawn. When the lot was cast, a fellow would sit down and stretch out his robe so that it was taut. The parties involved would gather round that robe, and that fellow would shake up those stones in that vessel. He would then cast them into the robe on his lap, and towards whatever person the black stone fell, that person was chosen. Now that was one of the ways serious matters were decided in Bible times. Proverbs 18:18 says, “The lot causeth contentions to cease, and parteth between the mighty.” That might seem a frivolous way of making important decisions, but it is still practiced today. In survival situations where someone must die that others might live, drawing straws is often used to decide who “gets it”; that way the method of selection is impartial. In the Old Testament, lots were used because those Jews believed, and rightly so, that God had a direct hand in the outcome. The Bible says, “The lot is cast into the lap; but the whole disposing thereof is of the LORD” (Prov. 16:33). That is why the apostles used the lot to choose a replacement for Judas in Acts 1; they were operating under an Old Testament system used in their Scripture. There was no New Testament yet in Acts 1 or, for that matter, in Acts 2, 3, 4, 5, 6, 7, 8. The Pauline revelation to the Body of Christ doesn’t show up until after Acts 9 (Gal. 1:11–24), but even then, the first New Testament Scriptures to the churches don’t show up until Acts 17 (1 Thessalonians). There are no “Christians” until Acts 11:26. That is why a man is a fool to go to Acts 2 for doctrine for the New Testament Church. There are no “Christians” in Acts 2. It’s Jews and Jewish proselytes through the entire chapter. Don’t believe me? GO READ IT! See if you can find the word Christian anywhere in Acts 2. It isn’t there. Then why would you go to Acts 2 to get the Holy Spirit by water baptism when the chapter isn’t even addressed to you? Self-righteous religionists would rather go to Hell on Acts 2:38 than go to Heaven on Acts 15:9–11 or Acts 13:38–39 or Acts 16:31 or Romans 4:5 or Ephesians 2:8–9 or Titus 3:5. Back to this matter of the lot. “The lot is cast into the lap; but the whole disposing thereof is of the LORD.” That’s why gambling is wrong: it’s tempting the providence of God. Jesus said, “Thou shalt not tempt the Lord thy God” (Matt. 4:7; Luke 4:12 cf. Deut. 6:16). If the disposing of the lot is of the Lord, then when you sit down and deal the cards or roll the dice or draw the numbers, it is the Lord who determines the outcome. You are tempting God to give you a good hand or give you the lucky numbers. For the believer, there is no such thing as “chance.” I mean, any Being who is omniscient and knows the number of hairs on your head (Matt. 10:30), do you honestly think He doesn’t know how the dice are going to roll or how the cards will be dealt? The old saying is: “Your hap is God’s map.” That means that God directs what we call “happenstance.” When the Bible says it was Ruth’s “hap. . . to light on a part of the field belonging unto Boaz” (Ruth 2:3), that was God working behind the scenes to direct Ruth to a man who would be her kinsman redeemer. So when you gamble or play games of chance, you are tempting God to providentially give you a good hand or a good roll; quite often He’ll give you “snake eyes” or a “four flush” to teach you a lesson. There’s an old joke about a Charismatic who believed in the “word of knowledge” and the “prosperity gospel.” He decided to go to Las Vegas and bet his life savings in the casinos because he was convinced that God wanted him rich, so He would give him all the winning hands and numbers. Well, you know what happened: he lost everything he had except the clothes he had on. As he was walking down the road out of town, he cried out to God and said, “Lord, how could you let this happen to me?! I’m penniless!” A voice came from Heaven and told him, “Look in your pocket.” The fellow emptied his pockets, and there was a lone twenty dollar bill he didn’t realize he had. The voice said, ``Go back and bet it on red 21 at the roulette wheel.'' So he turned around and went back into town to the nearest casino. He went to the roulette wheel and put his money down on red 21, and it came up black 13. As that poor nut walked down the road broke, he said, ``Lord, I thought you said to bet on red 21.'' The voice came back, “Well how about that.” Let that be a lesson to you. Don’t gamble, or the Lord may let you lose your shirt and say to you, “Well how about that.” Now what is about to take place is that plots of land are going to be laid out, and then representatives of the tribes are going to present themselves to Joshua as each plot is doled out. As each plot is “up for grabs,” the lots are cast until one of the tribes is chosen for that territory.}
[3] \textcolor[rgb]{0.00,0.00,1.00}{For Moses had given the inheritance of two tribes and an half tribe on the other side Jordan: but unto the Levites he gave none inheritance among them.}\index[AWIP]{For!Joshua!Jsh 14:003}\index[AWIP]{Moses!Joshua!Jsh 14:003}\index[AWIP]{had!Joshua!Jsh 14:003}\index[AWIP]{given!Joshua!Jsh 14:003}\index[AWIP]{the!Joshua!Jsh 14:003}\index[AWIP]{inheritance!Joshua!Jsh 14:003}\index[AWIP]{of!Joshua!Jsh 14:003}\index[AWIP]{two!Joshua!Jsh 14:003}\index[AWIP]{tribes!Joshua!Jsh 14:003}\index[AWIP]{and!Joshua!Jsh 14:003}\index[AWIP]{an!Joshua!Jsh 14:003}\index[AWIP]{half!Joshua!Jsh 14:003}\index[AWIP]{tribe!Joshua!Jsh 14:003}\index[AWIP]{on!Joshua!Jsh 14:003}\index[AWIP]{the!Joshua!Jsh 14:003 (2)}\index[AWIP]{other!Joshua!Jsh 14:003}\index[AWIP]{side!Joshua!Jsh 14:003}\index[AWIP]{Jordan!Joshua!Jsh 14:003}\index[AWIP]{but!Joshua!Jsh 14:003}\index[AWIP]{unto!Joshua!Jsh 14:003}\index[AWIP]{the!Joshua!Jsh 14:003 (3)}\index[AWIP]{Levites!Joshua!Jsh 14:003}\index[AWIP]{he!Joshua!Jsh 14:003}\index[AWIP]{gave!Joshua!Jsh 14:003}\index[AWIP]{none!Joshua!Jsh 14:003}\index[AWIP]{inheritance!Joshua!Jsh 14:003 (2)}\index[AWIP]{among!Joshua!Jsh 14:003}\index[AWIP]{them!Joshua!Jsh 14:003}\index[NWIV]{28!Joshua!Jsh 14:003}\index[PNIP]{Jordan!Joshua!Jsh 14:003}\index[PNIP]{Moses!Joshua!Jsh 14:003}\index[PNIP]{Levites!Joshua!Jsh 14:003}\footnote{Now we covered the reason for this back in 12:14. The Lord and the sacrifices and tithes of the Tabernacle were to be their inheritance. But here in verse 4, you are given another reason why they didn’t have a land inheritance like the other tribes.}
[4] \textcolor[rgb]{0.00,0.00,1.00}{For the children of Joseph were two tribes, Manasseh and Ephraim: therefore they gave no part unto the Levites in the land, save cities to dwell \emph{in}, with their suburbs for their cattle and for their substance.}\index[AWIP]{For!Joshua!Jsh 14:004}\index[AWIP]{the!Joshua!Jsh 14:004}\index[AWIP]{children!Joshua!Jsh 14:004}\index[AWIP]{of!Joshua!Jsh 14:004}\index[AWIP]{Joseph!Joshua!Jsh 14:004}\index[AWIP]{were!Joshua!Jsh 14:004}\index[AWIP]{two!Joshua!Jsh 14:004}\index[AWIP]{tribes!Joshua!Jsh 14:004}\index[AWIP]{Manasseh!Joshua!Jsh 14:004}\index[AWIP]{and!Joshua!Jsh 14:004}\index[AWIP]{Ephraim!Joshua!Jsh 14:004}\index[AWIP]{therefore!Joshua!Jsh 14:004}\index[AWIP]{they!Joshua!Jsh 14:004}\index[AWIP]{gave!Joshua!Jsh 14:004}\index[AWIP]{no!Joshua!Jsh 14:004}\index[AWIP]{part!Joshua!Jsh 14:004}\index[AWIP]{unto!Joshua!Jsh 14:004}\index[AWIP]{the!Joshua!Jsh 14:004 (2)}\index[AWIP]{Levites!Joshua!Jsh 14:004}\index[AWIP]{in!Joshua!Jsh 14:004}\index[AWIP]{the!Joshua!Jsh 14:004 (3)}\index[AWIP]{land!Joshua!Jsh 14:004}\index[AWIP]{save!Joshua!Jsh 14:004}\index[AWIP]{cities!Joshua!Jsh 14:004}\index[AWIP]{to!Joshua!Jsh 14:004}\index[AWIP]{dwell!Joshua!Jsh 14:004}\index[AWIP]{\emph{in}!Joshua!Jsh 14:004}\index[AWIP]{with!Joshua!Jsh 14:004}\index[AWIP]{their!Joshua!Jsh 14:004}\index[AWIP]{suburbs!Joshua!Jsh 14:004}\index[AWIP]{for!Joshua!Jsh 14:004}\index[AWIP]{their!Joshua!Jsh 14:004 (2)}\index[AWIP]{cattle!Joshua!Jsh 14:004}\index[AWIP]{and!Joshua!Jsh 14:004 (2)}\index[AWIP]{for!Joshua!Jsh 14:004 (2)}\index[AWIP]{their!Joshua!Jsh 14:004 (3)}\index[AWIP]{substance!Joshua!Jsh 14:004}\index[NWIV]{37!Joshua!Jsh 14:004}\index[PNIP]{Ephraim!Joshua!Jsh 14:004}\index[PNIP]{Joseph!Joshua!Jsh 14:004}\index[PNIP]{Manasseh!Joshua!Jsh 14:004}\index[PNIP]{Levites!Joshua!Jsh 14:004}\footnote{Instead of Joseph only having one tribe, he had two: “Manasseh and Ephraim.” Joseph gets two tribes because he gets the “double portion” of the birthright (1 Chron. 5:1 cf. Deut. 21:17; Gen. 48:22). Notice in Genesis 48, where Jacob gives Joseph the extra portion, that Jacob owns Joseph’s two sons in the place of his own first two sons, Reuben and Simeon (Gen. 48:5). Now we have covered why Reuben lost the birthright (see 1 Chron. 5:1), but what did Jacob have against Simeon? To get the answer, turn to Genesis 49:5–7, and take special note of Simeon’s connection to Levi. Jacob is referring to Simeon’s and Levi’s slaying of Shechem and his city in revenge for the rape of their sister. You can read the whole account in Genesis 34. What they did, they did out of hatred, vengeance, and cruel anger, so God scattered them among the tribes of Israel. Simeon got no real inheritance like the other tribes; his territory was some cities within the inheritance of Judah. Levi fared a little better because of what happened at the golden calf and Baal-peor (Exod. 32:26–28; Num. 25:7– 13), but all he got were cities he shared with the other tribes.}\footnote{\textbf{Genesis 49:5--7} - Simeon and Levi are brethren; instruments of cruelty are in their habitations. [6]  O my soul, come not thou into their secret; unto their assembly, mine honour, be not thou united: for in their anger they slew a man, and in their selfwill they digged down a wall. [7]  Cursed be their anger, for it was fierce; and their wrath, for it was cruel: I will divide them in Jacob, and scatter them in Israel.}\footnote{\textbf{Romans 12:10} - Dearly beloved, avenge not yourselves, but rather give place unto wrath: for it is written, Vengeance is mine, I will repay, saith the Lord.}
[5] \textcolor[rgb]{0.00,0.00,1.00}{As the LORD commanded Moses, so the children of Israel did, and they divided the land.}\index[AWIP]{As!Joshua!Jsh 14:005}\index[AWIP]{the!Joshua!Jsh 14:005}\index[AWIP]{LORD!Joshua!Jsh 14:005}\index[AWIP]{commanded!Joshua!Jsh 14:005}\index[AWIP]{Moses!Joshua!Jsh 14:005}\index[AWIP]{so!Joshua!Jsh 14:005}\index[AWIP]{the!Joshua!Jsh 14:005 (2)}\index[AWIP]{children!Joshua!Jsh 14:005}\index[AWIP]{of!Joshua!Jsh 14:005}\index[AWIP]{Israel!Joshua!Jsh 14:005}\index[AWIP]{did!Joshua!Jsh 14:005}\index[AWIP]{and!Joshua!Jsh 14:005}\index[AWIP]{they!Joshua!Jsh 14:005}\index[AWIP]{divided!Joshua!Jsh 14:005}\index[AWIP]{the!Joshua!Jsh 14:005 (3)}\index[AWIP]{land!Joshua!Jsh 14:005}\index[NWIV]{16!Joshua!Jsh 14:005}\index[PNIP]{Israel!Joshua!Jsh 14:005}\index[PNIP]{LORD!Joshua!Jsh 14:005}\index[PNIP]{Moses!Joshua!Jsh 14:005}\\
\\
\P \textcolor[rgb]{0.00,0.00,1.00}{Then the children of Judah came unto Joshua in Gilgal: and Caleb the son of Jephunneh the Kenezite said unto him, Thou knowest the thing that the LORD said unto Moses the man of God concerning me and thee in Kadesh-barnea.}\index[AWIP]{Then!Joshua!Jsh 14:006}\index[AWIP]{the!Joshua!Jsh 14:006}\index[AWIP]{children!Joshua!Jsh 14:006}\index[AWIP]{of!Joshua!Jsh 14:006}\index[AWIP]{Judah!Joshua!Jsh 14:006}\index[AWIP]{came!Joshua!Jsh 14:006}\index[AWIP]{unto!Joshua!Jsh 14:006}\index[AWIP]{Joshua!Joshua!Jsh 14:006}\index[AWIP]{in!Joshua!Jsh 14:006}\index[AWIP]{Gilgal!Joshua!Jsh 14:006}\index[AWIP]{and!Joshua!Jsh 14:006}\index[AWIP]{Caleb!Joshua!Jsh 14:006}\index[AWIP]{the!Joshua!Jsh 14:006 (2)}\index[AWIP]{son!Joshua!Jsh 14:006}\index[AWIP]{of!Joshua!Jsh 14:006 (2)}\index[AWIP]{Jephunneh!Joshua!Jsh 14:006}\index[AWIP]{the!Joshua!Jsh 14:006 (3)}\index[AWIP]{Kenezite!Joshua!Jsh 14:006}\index[AWIP]{said!Joshua!Jsh 14:006}\index[AWIP]{unto!Joshua!Jsh 14:006 (2)}\index[AWIP]{him!Joshua!Jsh 14:006}\index[AWIP]{Thou!Joshua!Jsh 14:006}\index[AWIP]{knowest!Joshua!Jsh 14:006}\index[AWIP]{the!Joshua!Jsh 14:006 (4)}\index[AWIP]{thing!Joshua!Jsh 14:006}\index[AWIP]{that!Joshua!Jsh 14:006}\index[AWIP]{the!Joshua!Jsh 14:006 (5)}\index[AWIP]{LORD!Joshua!Jsh 14:006}\index[AWIP]{said!Joshua!Jsh 14:006 (2)}\index[AWIP]{unto!Joshua!Jsh 14:006 (3)}\index[AWIP]{Moses!Joshua!Jsh 14:006}\index[AWIP]{the!Joshua!Jsh 14:006 (6)}\index[AWIP]{man!Joshua!Jsh 14:006}\index[AWIP]{of!Joshua!Jsh 14:006 (3)}\index[AWIP]{God!Joshua!Jsh 14:006}\index[AWIP]{concerning!Joshua!Jsh 14:006}\index[AWIP]{me!Joshua!Jsh 14:006}\index[AWIP]{and!Joshua!Jsh 14:006 (2)}\index[AWIP]{thee!Joshua!Jsh 14:006}\index[AWIP]{in!Joshua!Jsh 14:006 (2)}\index[AWIP]{Kadesh-barnea!Joshua!Jsh 14:006}\index[NWIV]{41!Joshua!Jsh 14:006}\index[PNIP]{Gilgal!Joshua!Jsh 14:006}\index[PNIP]{God!Joshua!Jsh 14:006}\index[PNIP]{Judah!Joshua!Jsh 14:006}\index[PNIP]{LORD!Joshua!Jsh 14:006}\index[PNIP]{Moses!Joshua!Jsh 14:006}\index[PNIP]{Joshua!Joshua!Jsh 14:006}\index[PNIP]{Caleb!Joshua!Jsh 14:006}\index[PNIP]{Jephunneh!Joshua!Jsh 14:006}\index[PNIP]{Kenezite!Joshua!Jsh 14:006}\index[PNIP]{Kadesh-barnea!Joshua!Jsh 14:006}\footnote{This is the account of Caleb coming to Joshua to ask for the land God promised him in Numbers 14:24 and Deuteronomy 1:36. This is one of those instances in Scripture where the narrative is not chronological. Notice in verses 12–13 that Caleb wants to drive out the Anakim in Hebron, but that took place back in 12:21--22. That passage says, “There was none of the Anakims left in the land,” with the exception of the cities of the Philistines. So the rest of the chapter here is something that takes place before what you read in 12:21–22. There is nothing unusual about that in the Bible. In the next chapter, you are going to get into the account of Caleb taking Debir (which also took place back in 12:21–22). But it is recorded again in Judges 1, which opens up with these words: “Now after the death of Joshua it came to pass” (Judg. 1:1). Things like that aren’t “contradictions” in the Scriptures; it’s just the Holy Spirit interjecting parenthetical material where He wants it. You say, “But why does He do that and make it so confusing?” To trap a bunch of puffed-up stuffshirts educated beyond their intelligence who want an excuse to reject what God said. Brethren, that Book is a steel bear trap. If you come to it with the wrong heart attitude or the wrong motive, the Lord won’t just “allow” you to be deceived, He’ll take an active roll in the deception (read Ezek. 14:1–11 carefully and note vs. 9 in that passage). Notice the expression “the man of God” in verse 6. This is the second time it shows up in your Bible. The first time is in Deuteronomy 33:1, where it is also a reference to Moses. The next time it shows up is in Judges 13 where it is a reference to the Angel of the Lord sent to Manoah and his wife with news of the birth of Samson (Judg. 13:6, 8). David is called “the man of God” in 2 Chronicles 8:14 and Nehemiah 12:24, 36. The expression appears the most in 1 \& 2 Kings: 1 Kings has it nineteen times; 2 Kings has it 36 times. Throughout the Old Testament, it is used of a prophet sent to deliver a message from God. In writing of the “prophecy of the scripture,” Simon Peter said, “holy men of God spake as they were moved by the Holy Ghost” (2 Pet. 1:20–21). The phrase only shows up twice in the New Testament: once in reference to Timothy as the pastor of the church at Ephesus (1 Tim. 6:11) and once as a general reference to any minister of the Scriptures (2 Tim. 3:17). Although John the Baptist was said to be “a man sent from God” (John 1:6), he was never given the title man of God; neither were any of the apostles. But given the two New Testament references above, the term could be applied to any God-called preacher involved in preaching the Gospel and the whole counsel of God. For certain, the expression never refers to any black-robed Baalite priest or wino over in Rome trying to get you into Heaven by African black magic and cannibalism (transubstantiation and the “Mass”), or a Campbellite “Water Dog” trying to get you into the bathtub to wash your sins away, or some Charismatic nut trying to get you to blabber incoherent nonsense, or a Liberal (now called “emergents”) with honey dripping from his tongue talking about how God would never send anyone to Hell because He is “love.” Those are all “false prophets” and “false teachers” whom Peter contrasts with “holy men of God” (2 Pet. 2:1). They “bring in damnable heresies,” deny the Lord, and damn themselves and their congregations (2 Pet. 2:1–2). They are not “men of God”; they are men of Satan (2 Cor. 11:14–15). You can recognize them by their message. It is marked by “great swelling words” that are as devoid of spiritual life as a corpse (2 Pet. 2:18; Jude 16); they speak “good words and fair speeches” for one purpose—to “deceive the hearts of the simple” (Rom. 16:18). Their message is a positive one (Jer. 23:16–17): “All is well; there is no Hell”; “You can make it to Heaven by your own good works”; “I’m OK; you’re OK.” Those are the messages of the ministers of the Devil.}
[7] \textcolor[rgb]{0.00,0.00,1.00}{Forty years old \emph{was} I when Moses the servant of the LORD sent me from Kadesh-barnea to espy out the land; and I brought him word again as \emph{it} \emph{was} in mine heart.}\index[AWIP]{Forty!Joshua!Jsh 14:007}\index[AWIP]{years!Joshua!Jsh 14:007}\index[AWIP]{old!Joshua!Jsh 14:007}\index[AWIP]{\emph{was}!Joshua!Jsh 14:007}\index[AWIP]{I!Joshua!Jsh 14:007}\index[AWIP]{when!Joshua!Jsh 14:007}\index[AWIP]{Moses!Joshua!Jsh 14:007}\index[AWIP]{the!Joshua!Jsh 14:007}\index[AWIP]{servant!Joshua!Jsh 14:007}\index[AWIP]{of!Joshua!Jsh 14:007}\index[AWIP]{the!Joshua!Jsh 14:007 (2)}\index[AWIP]{LORD!Joshua!Jsh 14:007}\index[AWIP]{sent!Joshua!Jsh 14:007}\index[AWIP]{me!Joshua!Jsh 14:007}\index[AWIP]{from!Joshua!Jsh 14:007}\index[AWIP]{Kadesh-barnea!Joshua!Jsh 14:007}\index[AWIP]{to!Joshua!Jsh 14:007}\index[AWIP]{espy!Joshua!Jsh 14:007}\index[AWIP]{out!Joshua!Jsh 14:007}\index[AWIP]{the!Joshua!Jsh 14:007 (3)}\index[AWIP]{land!Joshua!Jsh 14:007}\index[AWIP]{and!Joshua!Jsh 14:007}\index[AWIP]{I!Joshua!Jsh 14:007 (2)}\index[AWIP]{brought!Joshua!Jsh 14:007}\index[AWIP]{him!Joshua!Jsh 14:007}\index[AWIP]{word!Joshua!Jsh 14:007}\index[AWIP]{again!Joshua!Jsh 14:007}\index[AWIP]{as!Joshua!Jsh 14:007}\index[AWIP]{\emph{it}!Joshua!Jsh 14:007}\index[AWIP]{\emph{was}!Joshua!Jsh 14:007 (2)}\index[AWIP]{in!Joshua!Jsh 14:007}\index[AWIP]{mine!Joshua!Jsh 14:007}\index[AWIP]{heart!Joshua!Jsh 14:007}\index[NWIV]{33!Joshua!Jsh 14:007}\index[PNIP]{I!Joshua!Jsh 14:007}\index[PNIP]{LORD!Joshua!Jsh 14:007}\index[PNIP]{Moses!Joshua!Jsh 14:007}\index[PNIP]{Kadesh-barnea!Joshua!Jsh 14:007}\index[PNIP]{Forty!Joshua!Jsh 14:007}\footnote{Verses 7--8 are Caleb' synopsis of what took place in Numbers 13--14, which see. The word ``espy''in verse 7 is not only another way of saying ``spy''; it is a word meaning ``to examine and make discoveries.'' The word in the military is a scout. ``I brought him word again as it was in mine heart'' (vs. 7). Jesus said, ``out of the abundance of the heart the mouth speaketh'' (Matt. 12:34). Caleb's heart was in the right place. His heart was full of faith: he believed he could take the land just as God promised.}
[8] \textcolor[rgb]{0.00,0.00,1.00}{Nevertheless my brethren that went up with me made the heart of the people melt: but I wholly followed the LORD my God.}\index[AWIP]{Nevertheless!Joshua!Jsh 14:008}\index[AWIP]{my!Joshua!Jsh 14:008}\index[AWIP]{brethren!Joshua!Jsh 14:008}\index[AWIP]{that!Joshua!Jsh 14:008}\index[AWIP]{went!Joshua!Jsh 14:008}\index[AWIP]{up!Joshua!Jsh 14:008}\index[AWIP]{with!Joshua!Jsh 14:008}\index[AWIP]{me!Joshua!Jsh 14:008}\index[AWIP]{made!Joshua!Jsh 14:008}\index[AWIP]{the!Joshua!Jsh 14:008}\index[AWIP]{heart!Joshua!Jsh 14:008}\index[AWIP]{of!Joshua!Jsh 14:008}\index[AWIP]{the!Joshua!Jsh 14:008 (2)}\index[AWIP]{people!Joshua!Jsh 14:008}\index[AWIP]{melt!Joshua!Jsh 14:008}\index[AWIP]{but!Joshua!Jsh 14:008}\index[AWIP]{I!Joshua!Jsh 14:008}\index[AWIP]{wholly!Joshua!Jsh 14:008}\index[AWIP]{followed!Joshua!Jsh 14:008}\index[AWIP]{the!Joshua!Jsh 14:008 (3)}\index[AWIP]{LORD!Joshua!Jsh 14:008}\index[AWIP]{my!Joshua!Jsh 14:008 (2)}\index[AWIP]{God!Joshua!Jsh 14:008}\index[NWIV]{23!Joshua!Jsh 14:008}\index[PNIP]{God!Joshua!Jsh 14:008}\index[PNIP]{I!Joshua!Jsh 14:008}\index[PNIP]{LORD!Joshua!Jsh 14:008}\footnote{That sounds like bragging, but it wasn’t. Caleb was just quoting Scripture (vs. 9 cf. Deut. 1:36). Lost religionists think we bornagain Christians are being arrogant and presumptuous when we say we know we are going to Heaven. We aren’t presuming anything; we are simply quoting the promise of God (1 John 5:13).}
[9] \textcolor[rgb]{0.00,0.00,1.00}{And Moses sware on that day, saying, Surely the land whereon thy feet have trodden shall be thine inheritance, and thy children's for ever, because thou hast wholly followed the LORD my God.}\index[AWIP]{And!Joshua!Jsh 14:009}\index[AWIP]{Moses!Joshua!Jsh 14:009}\index[AWIP]{sware!Joshua!Jsh 14:009}\index[AWIP]{on!Joshua!Jsh 14:009}\index[AWIP]{that!Joshua!Jsh 14:009}\index[AWIP]{day!Joshua!Jsh 14:009}\index[AWIP]{saying!Joshua!Jsh 14:009}\index[AWIP]{Surely!Joshua!Jsh 14:009}\index[AWIP]{the!Joshua!Jsh 14:009}\index[AWIP]{land!Joshua!Jsh 14:009}\index[AWIP]{whereon!Joshua!Jsh 14:009}\index[AWIP]{thy!Joshua!Jsh 14:009}\index[AWIP]{feet!Joshua!Jsh 14:009}\index[AWIP]{have!Joshua!Jsh 14:009}\index[AWIP]{trodden!Joshua!Jsh 14:009}\index[AWIP]{shall!Joshua!Jsh 14:009}\index[AWIP]{be!Joshua!Jsh 14:009}\index[AWIP]{thine!Joshua!Jsh 14:009}\index[AWIP]{inheritance!Joshua!Jsh 14:009}\index[AWIP]{and!Joshua!Jsh 14:009}\index[AWIP]{thy!Joshua!Jsh 14:009 (2)}\index[AWIP]{children's!Joshua!Jsh 14:009}\index[AWIP]{for!Joshua!Jsh 14:009}\index[AWIP]{ever!Joshua!Jsh 14:009}\index[AWIP]{because!Joshua!Jsh 14:009}\index[AWIP]{thou!Joshua!Jsh 14:009}\index[AWIP]{hast!Joshua!Jsh 14:009}\index[AWIP]{wholly!Joshua!Jsh 14:009}\index[AWIP]{followed!Joshua!Jsh 14:009}\index[AWIP]{the!Joshua!Jsh 14:009 (2)}\index[AWIP]{LORD!Joshua!Jsh 14:009}\index[AWIP]{my!Joshua!Jsh 14:009}\index[AWIP]{God!Joshua!Jsh 14:009}\index[NWIV]{33!Joshua!Jsh 14:009}\index[PNIP]{God!Joshua!Jsh 14:009}\index[PNIP]{LORD!Joshua!Jsh 14:009}\index[PNIP]{Moses!Joshua!Jsh 14:009}
[10] \textcolor[rgb]{0.00,0.00,1.00}{And now, behold, the LORD hath kept me alive, as he said, these forty and five years, even since the LORD spake this word unto Moses, while \emph{the} \emph{children} \emph{of} Israel wandered in the wilderness: and now, lo, I \emph{am} this day fourscore and five years old.}\index[AWIP]{And!Joshua!Jsh 14:010}\index[AWIP]{now!Joshua!Jsh 14:010}\index[AWIP]{behold!Joshua!Jsh 14:010}\index[AWIP]{the!Joshua!Jsh 14:010}\index[AWIP]{LORD!Joshua!Jsh 14:010}\index[AWIP]{hath!Joshua!Jsh 14:010}\index[AWIP]{kept!Joshua!Jsh 14:010}\index[AWIP]{me!Joshua!Jsh 14:010}\index[AWIP]{alive!Joshua!Jsh 14:010}\index[AWIP]{as!Joshua!Jsh 14:010}\index[AWIP]{he!Joshua!Jsh 14:010}\index[AWIP]{said!Joshua!Jsh 14:010}\index[AWIP]{these!Joshua!Jsh 14:010}\index[AWIP]{forty!Joshua!Jsh 14:010}\index[AWIP]{and!Joshua!Jsh 14:010}\index[AWIP]{five!Joshua!Jsh 14:010}\index[AWIP]{years!Joshua!Jsh 14:010}\index[AWIP]{even!Joshua!Jsh 14:010}\index[AWIP]{since!Joshua!Jsh 14:010}\index[AWIP]{the!Joshua!Jsh 14:010 (2)}\index[AWIP]{LORD!Joshua!Jsh 14:010 (2)}\index[AWIP]{spake!Joshua!Jsh 14:010}\index[AWIP]{this!Joshua!Jsh 14:010}\index[AWIP]{word!Joshua!Jsh 14:010}\index[AWIP]{unto!Joshua!Jsh 14:010}\index[AWIP]{Moses!Joshua!Jsh 14:010}\index[AWIP]{while!Joshua!Jsh 14:010}\index[AWIP]{\emph{the}!Joshua!Jsh 14:010}\index[AWIP]{\emph{children}!Joshua!Jsh 14:010}\index[AWIP]{\emph{of}!Joshua!Jsh 14:010}\index[AWIP]{Israel!Joshua!Jsh 14:010}\index[AWIP]{wandered!Joshua!Jsh 14:010}\index[AWIP]{in!Joshua!Jsh 14:010}\index[AWIP]{the!Joshua!Jsh 14:010 (3)}\index[AWIP]{wilderness!Joshua!Jsh 14:010}\index[AWIP]{and!Joshua!Jsh 14:010 (2)}\index[AWIP]{now!Joshua!Jsh 14:010 (2)}\index[AWIP]{lo!Joshua!Jsh 14:010}\index[AWIP]{I!Joshua!Jsh 14:010}\index[AWIP]{\emph{am}!Joshua!Jsh 14:010}\index[AWIP]{this!Joshua!Jsh 14:010 (2)}\index[AWIP]{day!Joshua!Jsh 14:010}\index[AWIP]{fourscore!Joshua!Jsh 14:010}\index[AWIP]{and!Joshua!Jsh 14:010 (3)}\index[AWIP]{five!Joshua!Jsh 14:010 (2)}\index[AWIP]{years!Joshua!Jsh 14:010 (2)}\index[AWIP]{old!Joshua!Jsh 14:010}\index[NWIV]{47!Joshua!Jsh 14:010}\index[PNIP]{I!Joshua!Jsh 14:010}\index[PNIP]{Israel!Joshua!Jsh 14:010}\index[PNIP]{LORD!Joshua!Jsh 14:010}\index[PNIP]{Moses!Joshua!Jsh 14:010}\footnote{Caleb was forty years old when he was called, and he was just fulfilling the Lord’s purpose for his life at 85. Moses was called when he was eighty and served until he was 120. I know they lived longer in those days (Joshua lived to be 110—see 24:29), but the equivalent would be someone in his sixties or seventies still going strong for the Lord. Something I have never understood about older folk is why, if they are in good health, they just sit around and do nothing. I am 92 now, but as long as the Lord gives me the strength and the eyesight, I am going to keep on doing something for Him. The eyesight is going out now, but I will still serve the Lord with whatever eyesight I still have left. Just because you are up in years doesn’t mean the Lord can’t still use you. He was still using the Apostle John into his nineties; He used Daniel into his eighties. Billy Sunday had a convert named Lucius Eddy who got saved when he was 75; Eddy went into evangelism and led 4,000 men to Christ before he died at 87. “And now, behold, the LORD hath kept me alive.” My, my, my, how true that is. Do you know why I’m still alive and kicking at 92? Because “the LORD hath kept me alive,” and if He hadn’t, I’d be “pushing up daisies.” I attribute my age, and years and years of good health to two things: the mercy of God and the prayers of God’s people. I’m not trying to be pious; that’s the truth. If a bunch of Christians hadn’t been praying for me and the Lord hadn’t been merciful to me, I’d have been “six feet under” long ago. “The LORD hath kept me alive.” There are a lot of people He didn’t. My generation fought in World War II. Out of a congregation of about 300, at this writing, there are only two of my generation still left: myself and Robert Mitchell, one of my trustees. Where are the rest of my generation? They’re gone; they’re dead. A lot of them never made it back from the Belgium bulge, Iwo Jima, Okinawa, etc.; they were buried before they were 25 years old. Where are all my buddies in the ministry? They’ve gone home to be with the Lord. “Pappy” Reveal is gone. He was the superintendent of the Evansville Rescue Mission in Indiana. He “ordained” me the day I graduated from Bob Jones University. I came out of the Rodeheaver Auditorium with my degree, and “Pappy” was sitting in his car. He called me over and told me to kneel. He put his hands on me and prayed for me. He thought so much of me that when he died, he left me two notebooks full of his sermon outlines. The Lord called him home in 1959. The man who got me into the ministry, Glenn Schunk, died in 1978. My buddy in the ministry, Edmund Dinant, the converted gangster who pastored up in Flint, Michigan, passed on in 1989. The man who led me to Christ, Hugh Pyle, went home to Glory in 2010. Why am I still here when so many of my buddies are gone? “The LORD hath kept me alive,” and as long as I am alive and health permits, I am going to do something for Him. “As my strength was then, even so is my strength now.” There is a fellow who, at the end, was just as tough as when he began. That old boy was just as strong physically at 85 as he was at forty. If you don’t think that’s something, just try it. There’s a reason why young men fight wars: they have the stamina, strength, speed, and agility to do what is necessary in battle. As you get older, you get winded easier, you slow down, and your joints stiffen up. At some point, you don’t go to the gym for “body-building”; it’s just “care and maintenance.” Not Caleb: he was ready to take on the giants at 85. That was God taking care of that bird and preserving him like He did Moses (Deut. 34:7). “For war, both to go out, and to come in.” Caleb is talking about going out to battle and coming back victorious, and safe and sound. Combatants don’t always come back from the battle like they go out. A lot of our boys came back from Vietnam and Iraq and Afghanistan with legs blown off and limbs blown off. They are confined to wheelchairs or have to go around on artificial legs and use artificial limbs. The French went off to the Battle of the Marne (1914) in taxis and came back in coffins. Those Yankees marched out to the Battle of Bull Run (1861) in bright yellow, red, and green uniforms, and it was a “run” too.  “Johnny Reb” sent them running all the way back to Washington, D.C. Napoleon marched into Russia with a grand army all in ranks, and what was left of them came hobbling out on crutches and being pulled in sleds (the wounded). Hitler charged into Russia like lightning (blitzkrieg) and came limping out with two million casualties. You don’t always “come in” like you “go out.”}
[11] \textcolor[rgb]{0.00,0.00,1.00}{As yet I \emph{am} \emph{as} strong this day as \emph{I} \emph{was} in the day that Moses sent me: as my strength \emph{was} then, even so \emph{is} my strength now, for war, both to go out, and to come in.}\index[AWIP]{As!Joshua!Jsh 14:011}\index[AWIP]{yet!Joshua!Jsh 14:011}\index[AWIP]{I!Joshua!Jsh 14:011}\index[AWIP]{\emph{am}!Joshua!Jsh 14:011}\index[AWIP]{\emph{as}!Joshua!Jsh 14:011}\index[AWIP]{strong!Joshua!Jsh 14:011}\index[AWIP]{this!Joshua!Jsh 14:011}\index[AWIP]{day!Joshua!Jsh 14:011}\index[AWIP]{as!Joshua!Jsh 14:011}\index[AWIP]{\emph{I}!Joshua!Jsh 14:011}\index[AWIP]{\emph{was}!Joshua!Jsh 14:011}\index[AWIP]{in!Joshua!Jsh 14:011}\index[AWIP]{the!Joshua!Jsh 14:011}\index[AWIP]{day!Joshua!Jsh 14:011 (2)}\index[AWIP]{that!Joshua!Jsh 14:011}\index[AWIP]{Moses!Joshua!Jsh 14:011}\index[AWIP]{sent!Joshua!Jsh 14:011}\index[AWIP]{me!Joshua!Jsh 14:011}\index[AWIP]{as!Joshua!Jsh 14:011 (2)}\index[AWIP]{my!Joshua!Jsh 14:011}\index[AWIP]{strength!Joshua!Jsh 14:011}\index[AWIP]{\emph{was}!Joshua!Jsh 14:011 (2)}\index[AWIP]{then!Joshua!Jsh 14:011}\index[AWIP]{even!Joshua!Jsh 14:011}\index[AWIP]{so!Joshua!Jsh 14:011}\index[AWIP]{\emph{is}!Joshua!Jsh 14:011}\index[AWIP]{my!Joshua!Jsh 14:011 (2)}\index[AWIP]{strength!Joshua!Jsh 14:011 (2)}\index[AWIP]{now!Joshua!Jsh 14:011}\index[AWIP]{for!Joshua!Jsh 14:011}\index[AWIP]{war!Joshua!Jsh 14:011}\index[AWIP]{both!Joshua!Jsh 14:011}\index[AWIP]{to!Joshua!Jsh 14:011}\index[AWIP]{go!Joshua!Jsh 14:011}\index[AWIP]{out!Joshua!Jsh 14:011}\index[AWIP]{and!Joshua!Jsh 14:011}\index[AWIP]{to!Joshua!Jsh 14:011 (2)}\index[AWIP]{come!Joshua!Jsh 14:011}\index[AWIP]{in!Joshua!Jsh 14:011 (2)}\index[NWIV]{39!Joshua!Jsh 14:011}\index[PNIP]{I!Joshua!Jsh 14:011}\index[PNIP]{Moses!Joshua!Jsh 14:011}
[12] \textcolor[rgb]{0.00,0.00,1.00}{Now therefore give me this mountain, whereof the LORD spake in that day; for thou heardest in that day how the Anakims \emph{were} there, and \emph{that} the cities \emph{were} great \emph{and} fenced: if so be the LORD \emph{will} \emph{be} with me, then I shall be able to drive them out, as the LORD said.}\index[AWIP]{Now!Joshua!Jsh 14:012}\index[AWIP]{therefore!Joshua!Jsh 14:012}\index[AWIP]{give!Joshua!Jsh 14:012}\index[AWIP]{me!Joshua!Jsh 14:012}\index[AWIP]{this!Joshua!Jsh 14:012}\index[AWIP]{mountain!Joshua!Jsh 14:012}\index[AWIP]{whereof!Joshua!Jsh 14:012}\index[AWIP]{the!Joshua!Jsh 14:012}\index[AWIP]{LORD!Joshua!Jsh 14:012}\index[AWIP]{spake!Joshua!Jsh 14:012}\index[AWIP]{in!Joshua!Jsh 14:012}\index[AWIP]{that!Joshua!Jsh 14:012}\index[AWIP]{day!Joshua!Jsh 14:012}\index[AWIP]{for!Joshua!Jsh 14:012}\index[AWIP]{thou!Joshua!Jsh 14:012}\index[AWIP]{heardest!Joshua!Jsh 14:012}\index[AWIP]{in!Joshua!Jsh 14:012 (2)}\index[AWIP]{that!Joshua!Jsh 14:012 (2)}\index[AWIP]{day!Joshua!Jsh 14:012 (2)}\index[AWIP]{how!Joshua!Jsh 14:012}\index[AWIP]{the!Joshua!Jsh 14:012 (2)}\index[AWIP]{Anakims!Joshua!Jsh 14:012}\index[AWIP]{\emph{were}!Joshua!Jsh 14:012}\index[AWIP]{there!Joshua!Jsh 14:012}\index[AWIP]{and!Joshua!Jsh 14:012}\index[AWIP]{\emph{that}!Joshua!Jsh 14:012}\index[AWIP]{the!Joshua!Jsh 14:012 (3)}\index[AWIP]{cities!Joshua!Jsh 14:012}\index[AWIP]{\emph{were}!Joshua!Jsh 14:012 (2)}\index[AWIP]{great!Joshua!Jsh 14:012}\index[AWIP]{\emph{and}!Joshua!Jsh 14:012}\index[AWIP]{fenced!Joshua!Jsh 14:012}\index[AWIP]{if!Joshua!Jsh 14:012}\index[AWIP]{so!Joshua!Jsh 14:012}\index[AWIP]{be!Joshua!Jsh 14:012}\index[AWIP]{the!Joshua!Jsh 14:012 (4)}\index[AWIP]{LORD!Joshua!Jsh 14:012 (2)}\index[AWIP]{\emph{will}!Joshua!Jsh 14:012}\index[AWIP]{\emph{be}!Joshua!Jsh 14:012}\index[AWIP]{with!Joshua!Jsh 14:012}\index[AWIP]{me!Joshua!Jsh 14:012 (2)}\index[AWIP]{then!Joshua!Jsh 14:012}\index[AWIP]{I!Joshua!Jsh 14:012}\index[AWIP]{shall!Joshua!Jsh 14:012}\index[AWIP]{be!Joshua!Jsh 14:012 (2)}\index[AWIP]{able!Joshua!Jsh 14:012}\index[AWIP]{to!Joshua!Jsh 14:012}\index[AWIP]{drive!Joshua!Jsh 14:012}\index[AWIP]{them!Joshua!Jsh 14:012}\index[AWIP]{out!Joshua!Jsh 14:012}\index[AWIP]{as!Joshua!Jsh 14:012}\index[AWIP]{the!Joshua!Jsh 14:012 (5)}\index[AWIP]{LORD!Joshua!Jsh 14:012 (3)}\index[AWIP]{said!Joshua!Jsh 14:012}\index[NWIV]{54!Joshua!Jsh 14:012}\index[PNIP]{I!Joshua!Jsh 14:012}\index[PNIP]{LORD!Joshua!Jsh 14:012}\index[PNIP]{Anakims!Joshua!Jsh 14:012}\footnote{The Bible says, “But covet earnestly the
best gifts” (1 Cor. 12:31). In other words, if
you’re going to say “give me,” ask for
something worthwhile. The prodigal son said,
“give me the portion of goods that falleth to
me” (Luke 15:12). Wrong thing for which to
ask. He was asking to consume it on his own
lusts (James 4:3). He was going to squander it
all in the “far country” and be left penniless in
a hog pen. He would have been better off
asking God for “godliness with contentment”
(1 Tim. 6:6).
Herodias’ daughter told Herod, “Give me
here John Baptist’s head in a charger” (Matt.
14:8) and ended up damning her soul. Judas
asked those chief priests, “What will ye give
me,” and sold his soul out for “thirty pieces of
silver” (Matt. 26:15), and wound up in Hell
(Acts 1:25). That rich man in Hell said, “Give
me water” (Luke 16:24). Too little too late. He
should have asked for the water of life while he
had a chance (John 4:10, 15).
If you are going to say “give me” to the
Lord, you had better ask for something that will
make Him happy. Solomon said, “Give me
now wisdom and knowledge, that I may go
out and come in before this people: for who
can judge this thy people, that is so great?”
(2 Chron. 1:10). The Bible says, “the speech
pleased the Lord, that Solomon had asked
this thing” (1 Kings 3:10).
David hit it closer than Solomon; he said,
“give me understanding ACCORDING TO
THY WORD” (Psa. 119:169). It’s not just
“wisdom and knowledge.” Knowledge is
knowing the facts of a thing, wisdom is
knowing how to use those facts properly, but
understanding is knowing how the facts and
their use relate to God. You want to know how
to use the wisdom and knowledge God gives
you properly, and the only way you will do that
is “according to thy WORD.”
Caleb asked for the right thing. He asked
for what God promised him in His words:
“whereof the LORD spake in that day.”
Look at the end of verse 12: “if so be the
LORD will be with me, then I shall be able
to drive them out, AS THE LORD SAID.”
Caleb isn’t bragging; he’s claiming the promise
of God.
That’s a good rule to follow in prayer. If
you ask for a million dollars, you may or may
not get it, depending on whether the Lord
thinks it will be a detriment to you. If you ask
for a new car or a new house or a plasma TV
or something you plan to consume on your own
lusts, the Lord’s not going to give it to you
(James 4:3). But if you find something in His
Book that the Lord has promised you as a
believer in Christ and you persistently claim that
promise in prayer, you will get your prayer
answered in the Lord’s time (see our comments
on Luke 11:5–8, 18:1–8 in that Commentary).}
[13] \textcolor[rgb]{0.00,0.00,1.00}{And Joshua blessed him, and gave unto Caleb the son of Jephunneh Hebron for an inheritance.}\index[AWIP]{And!Joshua!Jsh 14:013}\index[AWIP]{Joshua!Joshua!Jsh 14:013}\index[AWIP]{blessed!Joshua!Jsh 14:013}\index[AWIP]{him!Joshua!Jsh 14:013}\index[AWIP]{and!Joshua!Jsh 14:013}\index[AWIP]{gave!Joshua!Jsh 14:013}\index[AWIP]{unto!Joshua!Jsh 14:013}\index[AWIP]{Caleb!Joshua!Jsh 14:013}\index[AWIP]{the!Joshua!Jsh 14:013}\index[AWIP]{son!Joshua!Jsh 14:013}\index[AWIP]{of!Joshua!Jsh 14:013}\index[AWIP]{Jephunneh!Joshua!Jsh 14:013}\index[AWIP]{Hebron!Joshua!Jsh 14:013}\index[AWIP]{for!Joshua!Jsh 14:013}\index[AWIP]{an!Joshua!Jsh 14:013}\index[AWIP]{inheritance!Joshua!Jsh 14:013}\index[NWIV]{16!Joshua!Jsh 14:013}\index[PNIP]{Hebron!Joshua!Jsh 14:013}\index[PNIP]{Joshua!Joshua!Jsh 14:013}\index[PNIP]{Caleb!Joshua!Jsh 14:013}\index[PNIP]{Jephunneh!Joshua!Jsh 14:013}
[14] \textcolor[rgb]{0.00,0.00,1.00}{Hebron therefore became the inheritance of Caleb the son of Jephunneh the Kenezite unto this day, because that he wholly followed the LORD God of Israel.}\index[AWIP]{Hebron!Joshua!Jsh 14:014}\index[AWIP]{therefore!Joshua!Jsh 14:014}\index[AWIP]{became!Joshua!Jsh 14:014}\index[AWIP]{the!Joshua!Jsh 14:014}\index[AWIP]{inheritance!Joshua!Jsh 14:014}\index[AWIP]{of!Joshua!Jsh 14:014}\index[AWIP]{Caleb!Joshua!Jsh 14:014}\index[AWIP]{the!Joshua!Jsh 14:014 (2)}\index[AWIP]{son!Joshua!Jsh 14:014}\index[AWIP]{of!Joshua!Jsh 14:014 (2)}\index[AWIP]{Jephunneh!Joshua!Jsh 14:014}\index[AWIP]{the!Joshua!Jsh 14:014 (3)}\index[AWIP]{Kenezite!Joshua!Jsh 14:014}\index[AWIP]{unto!Joshua!Jsh 14:014}\index[AWIP]{this!Joshua!Jsh 14:014}\index[AWIP]{day!Joshua!Jsh 14:014}\index[AWIP]{because!Joshua!Jsh 14:014}\index[AWIP]{that!Joshua!Jsh 14:014}\index[AWIP]{he!Joshua!Jsh 14:014}\index[AWIP]{wholly!Joshua!Jsh 14:014}\index[AWIP]{followed!Joshua!Jsh 14:014}\index[AWIP]{the!Joshua!Jsh 14:014 (4)}\index[AWIP]{LORD!Joshua!Jsh 14:014}\index[AWIP]{God!Joshua!Jsh 14:014}\index[AWIP]{of!Joshua!Jsh 14:014 (3)}\index[AWIP]{Israel!Joshua!Jsh 14:014}\index[NWIV]{26!Joshua!Jsh 14:014}\index[PNIP]{God!Joshua!Jsh 14:014}\index[PNIP]{Hebron!Joshua!Jsh 14:014}\index[PNIP]{Israel!Joshua!Jsh 14:014}\index[PNIP]{LORD!Joshua!Jsh 14:014}\index[PNIP]{Caleb!Joshua!Jsh 14:014}\index[PNIP]{Jephunneh!Joshua!Jsh 14:014}\index[PNIP]{Kenezite!Joshua!Jsh 14:014}
[15] \textcolor[rgb]{0.00,0.00,1.00}{And the name of Hebron before \emph{was} Kirjath-arba; \emph{which} \emph{Arba} \emph{was} a great man among the Anakims. And the land had rest from war.}\index[AWIP]{And!Joshua!Jsh 14:015}\index[AWIP]{the!Joshua!Jsh 14:015}\index[AWIP]{name!Joshua!Jsh 14:015}\index[AWIP]{of!Joshua!Jsh 14:015}\index[AWIP]{Hebron!Joshua!Jsh 14:015}\index[AWIP]{before!Joshua!Jsh 14:015}\index[AWIP]{\emph{was}!Joshua!Jsh 14:015}\index[AWIP]{Kirjath-arba!Joshua!Jsh 14:015}\index[AWIP]{\emph{which}!Joshua!Jsh 14:015}\index[AWIP]{\emph{Arba}!Joshua!Jsh 14:015}\index[AWIP]{\emph{was}!Joshua!Jsh 14:015 (2)}\index[AWIP]{a!Joshua!Jsh 14:015}\index[AWIP]{great!Joshua!Jsh 14:015}\index[AWIP]{man!Joshua!Jsh 14:015}\index[AWIP]{among!Joshua!Jsh 14:015}\index[AWIP]{the!Joshua!Jsh 14:015 (2)}\index[AWIP]{Anakims!Joshua!Jsh 14:015}\index[AWIP]{And!Joshua!Jsh 14:015 (2)}\index[AWIP]{the!Joshua!Jsh 14:015 (3)}\index[AWIP]{land!Joshua!Jsh 14:015}\index[AWIP]{had!Joshua!Jsh 14:015}\index[AWIP]{rest!Joshua!Jsh 14:015}\index[AWIP]{from!Joshua!Jsh 14:015}\index[AWIP]{war!Joshua!Jsh 14:015}\index[NWIV]{24!Joshua!Jsh 14:015}\index[PNIP]{Hebron!Joshua!Jsh 14:015}\index[PNIP]{Anakims!Joshua!Jsh 14:015}\index[PNIP]{Kirjath-arba!Joshua!Jsh 14:015}\footnote{Back to Hebron. Today, Hebron has been
turned over to the Moslems, under extortion
and blackmail (i.e., give us the land we
demand, and we’ll stop shooting rockets at you,
which was a lie -- more rockets were fired at
Israel from the “West Bank” and “Gaza Strip”
after they were turned over exclusively to the
Moslems than when the Israelis dwelled there
with the Moslems), as part of the “West Bank”
(another piece of news media propaganda; no
bank of a river is fifteen to thirty miles inland).
This was done on the basis that since Moslems
trace their ancestry back to Abraham through
Ishmael, that Hebron is one of Islam’s four
“holy cities.” But it is not Abraham and Ishmael
who are buried in Hebron; it is Abraham,
Isaac, and Jacob -- the three Jewish patriarchs.
And it is not Hagar the mother of Ishmael who
is buried with Abraham; it is Sarah -- the
mother of the Hebrew race.
Don’t you find it peculiar that there are two
cities in “Palestine” connected with the King of
a Jewish State -- Bethlehem and Hebron -- and
both of them have been turned over to Moslem
control? If the Moslems had their way, they
would have control of the most important city
of the King of the Jews -- Jerusalem. Someone
doesn’t want the King to come back (Psa. 2:1--3); that’s why they cemented the Eastern Gate
of Jerusalem shut back in A.D. 1187 -- to
prevent the Jewish King from fulfilling Ezekiel
44:1--3, 46:1--2 in the Jewish Scriptures. Not
only that, but in front of that gate, they put a
Moslem cemetery, thinking that the Jewish
King would never defile Himself with dead
bodies.
Boy, does Jehovah God have a surprise for
them. Contrary to what the Koran says (Sura
23:50 note 1723, 3:54, 5:75 note 724), the
King of the Jews has been here already, He
died on a cross, He was buried in a borrowed
tomb, and He came up from the dead. He is up
in Heaven right now at His Father’s right hand
(and the Koran says that’s a lie too, because
“Allah has no Son” -- see Sura 3:58, 9:30,
17:39, 2:116, 72:3, 6:102, 37:152), waiting
until His Father sends Him back to His throne
on the hill in Jerusalem where the Moslem
“Dome of the Rock” (the Mosque of Omar)
now stands (Psa. 2:6). To add insult to injury,
He’ll come back on Mohammed’s flying horse
(Hadith, Bukhari, Vol. V, Book 58, No. 227).
When that happens, the Jewish King from
Judah (not the “West Bank”) will enter
Jerusalem through that Eastern Gate just as His
Book said He would back in 574 B.C. When he
does, He isn’t going to worry about a bunch of
dead Moslems or a cemented gate. You see, He
will have just gotten through stomping 200
million UN troops into a bloody pulp (Rev.
14:14--20; Isa. 63:1-4). You think He’s going
to worry about trampling over the decayed
corpses of Moslems who have been dead and in
Hell for centuries? As for that cemented wall,
He’ll go through it without breaking a sweat
(John 20:19). And when He gets to that hill
where that Moslem mosque sits, He’ll tear it
down (if it hasn’t already been torn down to
accommodate the Jewish temple -- see 2 Thess.
2:4), put His throne there, and pitch every
“Palestinian” off His property (Zech. 14:21).
Jerusalem, Hebron, Bethlehem, and the rest
of “Palestine” will belong to the King of the
Jews once more. He’ll divide it to His people
(Ezekiel 48); and I don’t mean any Moslems,
Arabs, or “Palestinians.” In that day, every
“Moslem” will submit to Him, not Allah, and if
he doesn’t, he’ll get smacked up side the head
with an iron rod (Psa. 2:9) and have the rain
shut off in his land until he does submit (Zech.
14:18). That’s what the future “Joshua” has
planned for Mohammed’s Moslems and
“Palestinians.”}
\