\section{Acts 1 Comments}


\subsection{Acts 1:1}
“The FORMER treatise...O Theophilus” (vs. 1) is a perfect match to a “former treatise”
which Luke wrote to Theophilus (Luke 1:3). Furthermore, Luke is with Paul to the end (2 Tim. 4:11)
of this ministry, and the author of the book of Acts is with Paul on board ship (Acts 27:2–4).
What the majority of scholars want you to do is think that THEY proved Lukan authorship by
years of investigation and arguing. They will not tolerate the Bible itself proving anything without
their approval. The believer can kick off the game by recognizing Luke as the author independently of
the entire library shelf of books on the book of Acts written by Felten, Knabenbauer, Wendt, Matthias,
Zockler, Zahn, Ramsay, Julicher, Weiss, Bethge, Gilbert, McGiffert, Nosgen, Askwith, Vogel, Nestle,
Bousset, Blass, Harnack, Dalman, Hort, and Curtius. If any one of these critics arrived at the
conclusion that Luke wrote the book of Acts we may congratulate him for his common sense, but we
are no more going to accept this as proof of his ability to understand the Authorized Text than we
would if he had never posited an opinion in the first place.
“Of all that Jesus began both to do and teach” (vs. 1). The beginning of Christ’s ministry,
therefore, is in the Gospel of Luke. The statement implies a future doing and teaching, and what
follows (see Acts 1:2–5) confirms the wording. In the person of the Holy Ghost, the Lord Jesus
continued “both to DO and TEACH” long after He ascended from Mt. Olivet. The opening words
are the equivalent of sending a man a telegram beginning with the word “and” and then putting “No.
2” at the top of the telegram.


\subsection{Acts 1:2}
“Commandments unto the apostles whom He had chosen” (vs. 2). These constitute a set of
additional commandments like those in John 15:12 and 1 John 2:3, which go beyond the “Ten” still
magnified by Seventh-day Adventists, whoremongers, menstealers, liars, etc., (see 1 Tim. 1:10).
Furthermore, these “commandments” pertain to “the kingdom of God” (vs. 3), which is always a
reference to a spiritual state of moral righteousness (Rom. 14:17). Although this kingdom is never
identical to the Kingdom of Heaven (see The Sure Word of Prophecy, 1973), they are present on earth
when the Lord Jesus Christ is on earth. Notice that the physical and spiritual kingdom are present in
Genesis 18:1–2 and in Luke 24:12–49. This foreshadows the Millennium when the Kingdom of God
will “appear” (Luke 19:11) alongside the visible, physical restoration of the Jewish theocracy in
Palestine (Luke 1:30–33). It is apparent that the Kingdom of God mentioned in Romans 14:17 is
connected with the Body of Christ in this age (Eph. 3:1–6), and since this Body is not revealed until after Acts 9—I said “revealed,” not “formed”—the Kingdom of God in Acts 1:3 has to have a
Millennial application.
All of this is obscured from the major commentators for reasons which we have mentioned on a
score of occasions. Oddly enough, Cornelius Stam (Acts Dispensationally Considered, p. 28) fails to
make application here although he knew the truth of the matter. Here, Brother Stam equates the
Kingdom of God with the Kingdom of Heaven (as G. T. Armstrong has always done!) by pointing out
that the “kingdom reign” (p. 29) is helpful in interpreting the text. But this is a half truth. The
“kingdom REIGN” is the visible, literal Kingdom of Heaven, as it appears in every passage in either
Testament. The Kingdom of God is a spiritual Kingdom of righteousness wherever it appears in either
Testament. [Williams—also a Hyper-Dispensationalist—muffs the ball neatly here and tells us that
the book of Acts does not “record” the formation of the Church, and therefore, we shouldn’t believe
any Church (which is Christ’s Body) is present (Williams Commentary, p. 820). We will speak more
of this later.]

\subsection{Acts 1:5}

But wait for the PROMISE OF THE FATHER, which, saith he, ye have heard of me” (vs. 4). Pentecost, then becomes illuminated by the words used to describe it. The “promise of the Father” at Pentecost is:
\begin{compactenum}
	\item A BAPTISM (vs. 5).
	\item An enduement of POWER (vs. 8).
	\item A FILLING of the Spirit (Acts 2:4).
	\item The Beginning of an ORGANISM (John 17:21, 23).
	\item The beginning of a DISPENSATION (John 14:16, 16:13).
\end{compactenum}
The Pentecostals believe 1, 2, and 5. The Baptists will allow 1, 2, 4, and 5. Stam and Bullinger
take only 1, 2, and 3. Scofield takes 1, 3, 4, and 5, and you can take as many as you like. (It’s a free
country!) But before we are through, we are going to believe all five of them without batting an eye or
going to a lexicon or looking at one Greek word or manuscript. We will not even “pray through at the
altar” to get our information. Belief will solve a lot of problems that “the originals” will never solve.
“The promise of the Father” is found in Luke 24:49, and the date for its fulfillment was fixed
before David begat Solomon (Lev. 23). The “promise” is the Holy Ghost coming UPON (Luke
24:49) the disciples (Acts 1:8) in the sense of “overhead.” This clearly matches the immersion in
water and, consequently, explains why the action is compared to that of John the Baptist (see
comments on Matt. 3:11). Notice that Mary experiences the same thing in giving conception to the
Saviour (Luke 1:35). For this reason the term “baptism” is retained throughout the Pauline
epistles even at the expense of someone’s confusing the word (Rom. 6:2–5; Col. 2:12) with water
baptism every time it occurs. The “ONE BAPTISM” of which Paul speaks (Eph. 4:5) is not only
Spirit baptism—Stam and Baker caught that—it is ONE baptism, not two, three, or four, as Stam,
Baker, Ballinger, Bullinger, and O’Hare would have you believe. The same baptism that put Paul into
the Body put the Corinthians into the Body, if we are to believe 1 Corinthians 12:13: This same
“ONE baptism” put people into Christ before Acts 9:1–6 if we are to believe Paul’s own confession
in Romans 16:7.
If the Body of Christ began AFTER Acts 9 (or after Acts 28), there are THREE baptisms: one for
Peter, James, and John in Acts 2; one for Paul in Acts 9; and one for the Gentile converts after Acts
13. To avoid this “most embarrassing situation” Stam and Baker run to the Greek (just like Wendt,
Schaff, Lightfoot, Ellicott, and you-know-who!) to prove that “by” and “with” are not the same
baptism. (But we shall also comment on this matter later. Until then, remember that if you get hit in the
street “by” a car, you probably got hit “with” one at the same time.)
“But ye shall be baptized with the Holy Ghost not many days hence” (vs. 5). What happened
to the “baptism of the Holy Ghost and FIRE”? This is what we keep hearing in Florida, Ohio,
Tennessee, Alabama, California, Kentucky, Georgia, Kansas, Oklahoma, South Carolina, Michigan,
Arizona, and Maine. How could Luke have forgotten the “fire” of Matthew 3:11? And how did Peter
forget it AGAIN in Acts 11:16? (See comments on Joe Smith and Moroni—Acts 2:47.)
Well, Luke and Peter forgot nothing; our Pentecostal friends forgot to read the context of Matthew
3:11 where the “fire” was a reference to Malachi 4:1, 3, and 2 Thessalonians 1:8–9. Both of these
contexts speak of the destruction of sinners at the Second Advent. If you have been praying for the
“baptism of the Holy Ghost and FIRE,” you had better rest content with the first half and drop the
second petition before you experience it! (The “fire” of Acts 2:3 is plainly “LIKE AS of fire”—not
“fire.”)
Before leaving our first main section, observe the peculiar twist given to this baptism by the Hyper-Dispensationalists who abhor Acts 2 like a Campbellite abhors Acts 10. As we have just mentioned, Stam and Co. have insisted that this baptism is “with” (Greek, en), while the baptism of 1
Corinthians 12:13 is “by” (Greek, en—same word). The ancient cliché for this funny business is that in Acts 2 the Lord is the Baptizer,” while in 1 Corinthians “the Holy Spirit is the Baptizer.” However, it is the Lord (NOT the Holy Spirit) who performs the baptism (Greek, en) of Colossians 2:12! That last reference is the proof text used by Bullinger and Stam for baptism into the One Body (or the “Mystery Body” or the “Church of the one Body”). Someone is so set on “dividing” the word they have made divisions which “wrongly divided the word of truth.” We shall say more about these matters when we get into the exposition of Acts 2:1–39.
(The Hyper-Dispensationalist flinches in Acts 2 because of his own inability to reconcile it with Ephesians doctrinally. What a Hyper-Dispensationalist cannot understand doctrinally he relegates to another dispensation in his own system of private interpretation.)