\subsection{Outlines from Eades}

\subsubsection{Life Out of Balance}

\index[speaker]{Brian Eades!Proverb 11:1 (Life Out of Balance)}
\index[series]{Proverbs (Brian Eades)!Pro 11:1 (Life Out of Balance)}
\index[date]{2017/09/17!Proverbs 11:1 (Life Out of Balance) (Brian Eades)}

\index[FACEBOOK]{SERMON HINTS!Brian Eades - Proverb 11:1!2017/09/17}


\textbf{Introduction:} All of us have the same number of hours each day, what matters is how we balance our time. People will always do what is most important to them no matter what. It's hard to make things balance, like time was our family, time for our self comma time with God, and not just time with each of these things but quality time. A balance was very important in managing expenses and cost in society in biblical times. the very reason that Jesus went into the temple and overthrew the tables of the money changers was because they used a false balance and took advantage of people's desire to worship God by making merchandise of them when buying sacrifices to offer the sale of sheeps and goats and doves was a common thing around the Temple Mount because people often could not make the long journey with livestock. Sometimes it was more convenient to take money and travel so you could buy your sacrifice when you arrive to Jerusalem. The problem in the matter was that people who sold the sacrifices were very dishonest. I wonder how much of our life is off balance and not pleasing to God?
\begin{compactenum}[I.]
    \item A FALSE BALANCE IS ONE-SIDED - The side that it leans to is always the wrong side and the side of dishonesty and Evil.\footnote{Brian Eades, 16 September 2017, \textbf{Sermon Hints} Facebook Group}
    \item A FALSE BALANCE IS OPPORTUNISTIC These are the people that live for God only when it's convenient, but when something better comes up to do they will drop their spiritual life first thing.
    \item A FALSE BALANCE IS OVERCHARGING - This would be the so-called Christian that expects too much of other people and not of their own self. This would be the person as Jesus described who puts upon people burdens that are heavy and grevious to be borne, yet will not lift that same weight with one of their fingers. it is the person who tries to help others but needs help themselves, the ones who strain at a gnat and swallow a camel, the ones who want to remove the speck out of somebody else's eye when they have a log in their own.
\end{compactenum}
\index[FACEBOOK]{SERMON HINTS!Brian Eades (Life Out of Balance) - Proverb 111!22017/09/17}

\subsubsection{A Jewel of Gold in a Swine's Snout}

\index[speaker]{Brian Eades!Proverb 11:22 (A Jewel of Gold in a Swine's Snout)}
\index[series]{Proverbs (Brian Eades)!Pro 11:22 (A Jewel of Gold in a Swine's Snout)}
\index[date]{2018/09/21!Proverbs 11:22 (A Jewel of Gold in a Swine's Snout) (Brian Eades)}

\index[FACEBOOK]{SERMON HINTS!Brian Eades (A Jewel of Gold in a Swine's Snout) - Proverb 11:22!2018/09/21}


\textbf{Introduction:} Swine were considered an unclean animal according to Mosaic law. Although we love to eat pork and we are not under judaic law, it is interesting that the symbol of being fat is a pig! Arteries are clogged by eating too much pork, so they were probably many underlying reasons for why the Lord put it on the unclean list. I will admit that bacon is my favorite protein. However we are not talking about pigs but we are talkin about people, women in particular. A woman without discretion is like a jewel of gold in a swine's snout.
\begin{compactenum}[I.]
    \item DECORATION WITHOUT DISCRETION \footnote{Brian Eades, 21 September 2018, \textbf{Sermon Hints} Facebook Group}
    \item GOOD NUANCES WITH A BAD NATURE
    \item FAIR ATTRACTION WITH A FLAWED APPETITE
    \item DELIGHTS IN MANIPULATION BUT DESIRES THE MUD
\end{compactenum}

\subsubsection{He that Watereth Shall Be Watered}

\index[speaker]{Brian Eades!Proverb 11:25 (He that Watereth Shall Be Watered)}
\index[series]{Proverbs (Brian Eades)!Pro 11:25 (He that Watereth Shall Be Watered)}
\index[date]{2018/09/21!Proverbs 11:25 (He that Watereth Shall Be Watered) (Brian Eades)}

\index[FACEBOOK]{SERMON HINTS!Brian Eades (He that Watereth Shall Be Watered) - Proverb 11:25!2018/09/21}


\textbf{Introduction:} Water is a very rare thing in the Middle East and therefore it is appreciated throughout scripture. As it stands today, the Jordan River is basically a dry gully from most of its length because of irrigation.  I have talked to Jews who live in Israel and they say that their biggest fight is not necessarily over Palestinian or Arabic issues, but mainly over water rights. In recent years the Syrians dammed up the Yarmouk river which feeds the Jordan as a primary source, which has caused instability in the region. On another note, as I've gotten older I have grown fond of flowers. I've learned the importance of watering to keep them vibrant and fragrant. Ministry is likened to watering flowers. Solomon says if you water somebody else, you'll get watered to!  If you look around you will find people wilting and drying up, so you need to get to watering as a child of God!
\begin{compactenum}[I.]
    \item THE FAITHFUL SERVANTS THAT WATER \footnote{Brian Eades, 21 September 2018, \textbf{Sermon Hints} Facebook Group}
    \item THE FLAWLESS SCRIPTURE THAT WATERS
    \item THE SERENDIPITY OF SAINTS THAT WATER
\end{compactenum}

\subsubsection{Every Christian's Job}

\index[speaker]{Brian Eades!Proverb 11:30 (Every Christian's Job)}
\index[series]{Proverbs (Brian Eades)!Pro 11:30 (Every Christian's Job)}
\index[event]{Mothers' Day (Brian Eades)!Proverbs 11:30 (Every Christian's Job)}
\index[date]{2019/05/17!Proverbs 11:30 (11:30) (Brian Eades)}

\index[FACEBOOK]{SERMON HINTS!Brian Eades (Every Christian's Job) - Proverb 11:30!2019/05/11}

\textbf{Introduction: }People are getting harder to reach for Christ than any other generation before us. LESS THAN 10\% OF THE CONVERTS PRODUCED ACTUALLY STICK AROUND A YEAR LATER.  After many people's profession of faith I have to go searching for them because they disappear.  People ask me "where is so-and-so? I thought they got saved?" Now this might be a hard pill to swallow but it is the truth, if you really believe in Jesus you will make your profession public and you will be baptized. If I have to pull that out of you then you are a lost sinner headed for Hell right now!  Believers, true believers do not have to be begged to love Jesus and to express it! In 1949, southern Baptist churches averaged one soul saved per year for every 20.2 members. Recently some statistics have determined that it takes nearly 200 members as of the year 2018 to win a single soul to Christ! For example, in the years I've pastored where I am at today, I have averaged yielding one soul saved for every 400 hours of Bible study and prayer (of course, I can save nobody, God alone saves). Basically one soul for every 40 sermons I have preached in my pulpit. That is a sad indictment on my ministry, and a wake-up call for all preachers!  I remember early in my ministry nearly 40 years ago when I would baptize over a dozen people at one gathering, but now it is just here and there.   At one time in my ministry, I set a personal goal of knocking on 12 doors per week (600+ doors a year); and after 6 solid years of doing so, I yielded just one soul.   It is true that His word will not return void, and there is no way that we can statistically track every convert from any part of our ministry, but it can be frustrating sometimes, BECAUSE THAT IS EVERY BELIEVER'S JOB! Jeremiah the prophet preached for 40 years with a little results in his time.... Nevertheless that is our mandate! Every believer should live with the motto "each one reach one"!  Technically every member of the church is a minister in some way, if they will just pray to find that ministry.  You are here on Sunday morning for two reasons, to worship God and to be equipped to win the lost when you go out those doors!  The only reason Jesus left the church on earth as recorded in John 17,  is that people may believe on our word.
\begin{compactenum}[I.]
    \item The \textbf{Inspiration} to Win Souls\footnote{Brian Eades, 17 May 2019, \textbf{Sermon Hints} Facebook Group}
    \begin{compactenum}[A.]
        \item Leading the way to salvation
        \item Learning the way of salvation
        \item Loving the way of salvation
    \end{compactenum}
    \item The \textbf{Incentives} to Win Souls
    \begin{compactenum}[A.]
        \item Salvage what's left!
        \item Save what's lost
        \item Serve with life
    \end{compactenum}
    \item The \textbf{Instruments} to Win Souls
    \begin{compactenum}[A.]
    \item The Truth of God
        \item The Touch of God
        \item The Trust in God
    \end{compactenum}
\end{compactenum}
