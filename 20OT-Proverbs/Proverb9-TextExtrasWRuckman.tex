\chapter{Proverbs 9}
\footnote{\textcolor[rgb]{0.00,0.25,0.00}{\hyperlink{TOC}{Return to end of Table of Contents.}}}\textcolor[rgb]{0.00,0.00,1.00}{Wisdom hath builded her house, she hath hewn out her seven pillars:}\footnote{[RUCKMAN] We can be certain that it is built on the Rock (Matt. 7:24) and has a sure foundation (1 Cor. 3:11) and is founded on nothing as unstable and transitory as the shifting sands of religion, science, and philosophy (see John 2:19).}\footnote{Wisdom is identifed as a female in the passage, so this would contrast with the ``strange woman'' also dscribed in Proverbs. In verses 1-3, wisdom does 7 things  \cite{Ruckman1972Proverbs}:
\begin{compactenum}
\item Wisdom hath builded her house (1),
\item Hewn out her seven pillars (1),
\item Killed her beasts (2),
\item Mingled her wine (2),
\item Furnished her table (2),
\item Sent forth her maidens (3), and
\item Crieth upon the highest places of the city (3).
\end{compactenum} 
}\footnote{The Seven Pillars of Wisdom - Assorted Explanations:
 }
[2] \textcolor[rgb]{0.00,0.00,1.00}{She hath killed her beasts; she hath mingled her wine; she hath also furnished her table.}\index[AWIP]{She!Proverbs!Pro 09:002}\index[AWIP]{hath!Proverbs!Pro 09:002}\index[AWIP]{killed!Proverbs!Pro 09:002}\index[AWIP]{her!Proverbs!Pro 09:002}\index[AWIP]{beasts!Proverbs!Pro 09:002}\index[AWIP]{she!Proverbs!Pro 09:002}\index[AWIP]{hath!Proverbs!Pro 09:002 (2)}\index[AWIP]{mingled!Proverbs!Pro 09:002}\index[AWIP]{her!Proverbs!Pro 09:002 (2)}\index[AWIP]{wine!Proverbs!Pro 09:002}\index[AWIP]{she!Proverbs!Pro 09:002 (2)}\index[AWIP]{hath!Proverbs!Pro 09:002 (3)}\index[AWIP]{also!Proverbs!Pro 09:002}\index[AWIP]{furnished!Proverbs!Pro 09:002}\index[AWIP]{her!Proverbs!Pro 09:002 (3)}\index[AWIP]{table!Proverbs!Pro 09:002}\index[NWIV]{16!Proverbs!Pro 09:002}\footnote{[RUCKMAN] The Holy Spirit has seven manifestations  (Isa. 11:2–3), and seven is plainly the number of prophecy completed (see The Bible Believer’s Commentary on Revelation, 1970). Solomon’s porch had two of these pillars, but Jesus has all seven for His house! (Heb. 3:3). \cite{Ruckman1972Proverbs}} \footnote{“She hath killed her ...She hath sent forth her maidens” (vss. 2--3). The scene nearly matches that of Matthew 22:1--3, except that, here, “maidens” are sent out instead of “servants.” The phony Septuagint, written hundreds of years after the death of Christ, picks up “servants” out of Matthew 22 and inserts it back into Proverbs 9:3, hoping that scholars will think that Matthew was quoting a pre Christian “Septuagint” instead of a Hebrew Old Testament. In spite of the obvious faux pas involved, the scholars do take the bait and believe it! The reader will observe that with this evidence, plus the dozen pieces of evidence just like it already mentioned, plus those mentioned in The Bible Believer’s Commentary on Genesis [1969], a mountain of evidence has been erected that no honest man could ignore. It is true that Machen, Robertson, Milligan, Gregory, Davis, Tregelles, Tischendorf, Lachmann, Griesbach, Weiss, Nestle, Westcott, Hort, and the faculties at fundamental schools reject the evidence, but this means nothing. It is merely indicative of the lateness of the hour, the darkness of the age, and the blindness of conservative scholarship. “She hath killed her beasts...she hath also furnished her table” (vs. 2, cf. 7:14, 18). Wisdom has spread a feast comparable to that of the foolish woman of 9:17, or better, and the hungry or thirsty soul (John 4:14) is invited to this banquet (Isa. 55:1) which is FREE and produces no heartburn, indigestion, stomach cramps, nausea, or “irregularity.” The feast of 9:17 winds up in a pile of worms where the fire is not quenched (Isa. 66:20–24). The reader will observe that the invitation is for fools as well as wise men (vs. 4, “Whoso is simple”), and that “the good and the bad” (Luke 14:21) are proper subjects for the invitation. To the young man who lacked understanding (7:7), Wisdom said, “Come, eat of my bread” (vs. 5). The bread and wine of wisdom (vs. 5) are not the worm eaten manna of the mass (Exod. 16:20) or the fermented “shinny” of the Roman priesthood (Deut. 32:32–33). \cite{Ruckman1972Proverbs}}
[3] \textcolor[rgb]{0.00,0.00,1.00}{She hath sent forth her maidens: she crieth upon the highest places of the city,}\index[AWIP]{She!Proverbs!Pro 09:003}\index[AWIP]{hath!Proverbs!Pro 09:003}\index[AWIP]{sent!Proverbs!Pro 09:003}\index[AWIP]{forth!Proverbs!Pro 09:003}\index[AWIP]{her!Proverbs!Pro 09:003}\index[AWIP]{maidens!Proverbs!Pro 09:003}\index[AWIP]{she!Proverbs!Pro 09:003}\index[AWIP]{crieth!Proverbs!Pro 09:003}\index[AWIP]{upon!Proverbs!Pro 09:003}\index[AWIP]{the!Proverbs!Pro 09:003}\index[AWIP]{highest!Proverbs!Pro 09:003}\index[AWIP]{places!Proverbs!Pro 09:003}\index[AWIP]{of!Proverbs!Pro 09:003}\index[AWIP]{the!Proverbs!Pro 09:003 (2)}\index[AWIP]{city!Proverbs!Pro 09:003}\index[NWIV]{15!Proverbs!Pro 09:003}
[4] \textcolor[rgb]{0.00,0.00,1.00}{Whoso \emph{is} simple, let him turn in hither: \emph{as} \emph{for} him that wanteth understanding, she saith to him,}\index[AWIP]{Whoso!Proverbs!Pro 09:004}\index[AWIP]{\emph{is}!Proverbs!Pro 09:004}\index[AWIP]{simple!Proverbs!Pro 09:004}\index[AWIP]{let!Proverbs!Pro 09:004}\index[AWIP]{him!Proverbs!Pro 09:004}\index[AWIP]{turn!Proverbs!Pro 09:004}\index[AWIP]{in!Proverbs!Pro 09:004}\index[AWIP]{hither!Proverbs!Pro 09:004}\index[AWIP]{\emph{as}!Proverbs!Pro 09:004}\index[AWIP]{\emph{for}!Proverbs!Pro 09:004}\index[AWIP]{him!Proverbs!Pro 09:004 (2)}\index[AWIP]{that!Proverbs!Pro 09:004}\index[AWIP]{wanteth!Proverbs!Pro 09:004}\index[AWIP]{understanding!Proverbs!Pro 09:004}\index[AWIP]{she!Proverbs!Pro 09:004}\index[AWIP]{saith!Proverbs!Pro 09:004}\index[AWIP]{to!Proverbs!Pro 09:004}\index[AWIP]{him!Proverbs!Pro 09:004 (3)}\index[NWIV]{18!Proverbs!Pro 09:004}
[5] \textcolor[rgb]{0.00,0.00,1.00}{Come, eat of my bread, and drink of the wine \emph{which} I have mingled.}\index[AWIP]{Come!Proverbs!Pro 09:005}\index[AWIP]{eat!Proverbs!Pro 09:005}\index[AWIP]{of!Proverbs!Pro 09:005}\index[AWIP]{my!Proverbs!Pro 09:005}\index[AWIP]{bread!Proverbs!Pro 09:005}\index[AWIP]{and!Proverbs!Pro 09:005}\index[AWIP]{drink!Proverbs!Pro 09:005}\index[AWIP]{of!Proverbs!Pro 09:005 (2)}\index[AWIP]{the!Proverbs!Pro 09:005}\index[AWIP]{wine!Proverbs!Pro 09:005}\index[AWIP]{\emph{which}!Proverbs!Pro 09:005}\index[AWIP]{I!Proverbs!Pro 09:005}\index[AWIP]{have!Proverbs!Pro 09:005}\index[AWIP]{mingled!Proverbs!Pro 09:005}\index[NWIV]{14!Proverbs!Pro 09:005}\index[PNIP]{I!Proverbs!Pro 09:005}
[6] \textcolor[rgb]{0.00,0.00,1.00}{Forsake the foolish, and live; and go in the way of understanding.}\index[AWIP]{Forsake!Proverbs!Pro 09:006}\index[AWIP]{the!Proverbs!Pro 09:006}\index[AWIP]{foolish!Proverbs!Pro 09:006}\index[AWIP]{and!Proverbs!Pro 09:006}\index[AWIP]{live!Proverbs!Pro 09:006}\index[AWIP]{and!Proverbs!Pro 09:006 (2)}\index[AWIP]{go!Proverbs!Pro 09:006}\index[AWIP]{in!Proverbs!Pro 09:006}\index[AWIP]{the!Proverbs!Pro 09:006 (2)}\index[AWIP]{way!Proverbs!Pro 09:006}\index[AWIP]{of!Proverbs!Pro 09:006}\index[AWIP]{understanding!Proverbs!Pro 09:006}\index[NWIV]{12!Proverbs!Pro 09:006}
[7] \textcolor[rgb]{0.00,0.00,1.00}{He that reproveth a scorner getteth to himself shame: and he that rebuketh a wicked \emph{man} \emph{getteth} himself a blot.}\index[AWIP]{He!Proverbs!Pro 09:007}\index[AWIP]{that!Proverbs!Pro 09:007}\index[AWIP]{reproveth!Proverbs!Pro 09:007}\index[AWIP]{a!Proverbs!Pro 09:007}\index[AWIP]{scorner!Proverbs!Pro 09:007}\index[AWIP]{getteth!Proverbs!Pro 09:007}\index[AWIP]{to!Proverbs!Pro 09:007}\index[AWIP]{himself!Proverbs!Pro 09:007}\index[AWIP]{shame!Proverbs!Pro 09:007}\index[AWIP]{and!Proverbs!Pro 09:007}\index[AWIP]{he!Proverbs!Pro 09:007}\index[AWIP]{that!Proverbs!Pro 09:007 (2)}\index[AWIP]{rebuketh!Proverbs!Pro 09:007}\index[AWIP]{a!Proverbs!Pro 09:007 (2)}\index[AWIP]{wicked!Proverbs!Pro 09:007}\index[AWIP]{\emph{man}!Proverbs!Pro 09:007}\index[AWIP]{\emph{getteth}!Proverbs!Pro 09:007}\index[AWIP]{himself!Proverbs!Pro 09:007 (2)}\index[AWIP]{a!Proverbs!Pro 09:007 (3)}\index[AWIP]{blot!Proverbs!Pro 09:007}\index[NWIV]{20!Proverbs!Pro 09:007}
[8] \textcolor[rgb]{0.00,0.00,1.00}{Reprove not a scorner, lest he hate thee: rebuke a wise man, and he will love thee.}\index[AWIP]{Reprove!Proverbs!Pro 09:008}\index[AWIP]{not!Proverbs!Pro 09:008}\index[AWIP]{a!Proverbs!Pro 09:008}\index[AWIP]{scorner!Proverbs!Pro 09:008}\index[AWIP]{lest!Proverbs!Pro 09:008}\index[AWIP]{he!Proverbs!Pro 09:008}\index[AWIP]{hate!Proverbs!Pro 09:008}\index[AWIP]{thee!Proverbs!Pro 09:008}\index[AWIP]{rebuke!Proverbs!Pro 09:008}\index[AWIP]{a!Proverbs!Pro 09:008 (2)}\index[AWIP]{wise!Proverbs!Pro 09:008}\index[AWIP]{man!Proverbs!Pro 09:008}\index[AWIP]{and!Proverbs!Pro 09:008}\index[AWIP]{he!Proverbs!Pro 09:008 (2)}\index[AWIP]{will!Proverbs!Pro 09:008}\index[AWIP]{love!Proverbs!Pro 09:008}\index[AWIP]{thee!Proverbs!Pro 09:008 (2)}\index[NWIV]{17!Proverbs!Pro 09:008}\footnote{[RUCKMAN] Verse 8 is to be interpreted by 1:22 and 3:34 (Matt. 24:9). Jesus often reproves the scorners and so does Paul (Titus 1:13), but, here, the Old Testament is speaking of common, down to earth rules for “getting along with people.” The New Testament demands for a heavenly walk (Col. 3:1--3), and spiritual warfare cannot be subjected to 9:8. (See Stephen, Acts 7; Peter, Acts 8; Paul, Acts 13.) John the Baptist reproves a scorner (Herod) and gets into plenty of hot water. The lesson, however, is an Old Testament Lesson. Stephen rebukes the scorners and loses his life; but Stephen in Greek is Stephanos (a crown!), and the Crown of Life is for those who are “faithful unto death” (Rev. 2:10). “Rebuke a wise man, and he will love thee” (vs. 8). The outstanding example in the Old Testament is David receiving rebuke at the mouth of Nathan; in the New Testament it is Peter receiving rebukes from Jesus (Matt. 16:23 and John 13:8). David words it excellently in Psalm 141:5. \cite{Ruckman1972Proverbs}}
[9] \textcolor[rgb]{0.00,0.00,1.00}{Give \emph{instruction} to a wise \emph{man}, and he will be yet wiser: teach a just \emph{man}, and he will increase in learning.}\index[AWIP]{Give!Proverbs!Pro 09:009}\index[AWIP]{\emph{instruction}!Proverbs!Pro 09:009}\index[AWIP]{to!Proverbs!Pro 09:009}\index[AWIP]{a!Proverbs!Pro 09:009}\index[AWIP]{wise!Proverbs!Pro 09:009}\index[AWIP]{\emph{man}!Proverbs!Pro 09:009}\index[AWIP]{and!Proverbs!Pro 09:009}\index[AWIP]{he!Proverbs!Pro 09:009}\index[AWIP]{will!Proverbs!Pro 09:009}\index[AWIP]{be!Proverbs!Pro 09:009}\index[AWIP]{yet!Proverbs!Pro 09:009}\index[AWIP]{wiser!Proverbs!Pro 09:009}\index[AWIP]{teach!Proverbs!Pro 09:009}\index[AWIP]{a!Proverbs!Pro 09:009 (2)}\index[AWIP]{just!Proverbs!Pro 09:009}\index[AWIP]{\emph{man}!Proverbs!Pro 09:009 (2)}\index[AWIP]{and!Proverbs!Pro 09:009 (2)}\index[AWIP]{he!Proverbs!Pro 09:009 (2)}\index[AWIP]{will!Proverbs!Pro 09:009 (2)}\index[AWIP]{increase!Proverbs!Pro 09:009}\index[AWIP]{in!Proverbs!Pro 09:009}\index[AWIP]{learning!Proverbs!Pro 09:009}\index[NWIV]{22!Proverbs!Pro 09:009}\footnote{[RUCKMAN] The reader should observe that the two prerequisites for learning are that a man be “wise” and “just,” not “talented” and “learned.” The soul winner is wise (Prov. 11:30), and when he is instructed, he will be yet wiser. The “just” man of verse 9 can take teaching and learn from it. By such standards (if they are true) the modern American system of education is shot through from the seat of its britches to its cap and gown. Of this system it can be truthfully said: “Ever learning, and never able to come to the knowledge of the truth” (2 Tim. 3:7). As a matter of fact, that is the profession of every secular university in the United States. None profess to be able to find THE truth, and most of them declare publicly and privately that there is no such thing as THE truth; there are only “truths” (see Pilate: John 18:38). \cite{Ruckman1972Proverbs}}
[10] \textcolor[rgb]{0.00,0.00,1.00}{The fear of the LORD \emph{is} the beginning of wisdom: and the knowledge of the holy \emph{is} understanding.}\index[AWIP]{The!Proverbs!Pro 09:010}\index[AWIP]{fear!Proverbs!Pro 09:010}\index[AWIP]{of!Proverbs!Pro 09:010}\index[AWIP]{the!Proverbs!Pro 09:010}\index[AWIP]{LORD!Proverbs!Pro 09:010}\index[AWIP]{\emph{is}!Proverbs!Pro 09:010}\index[AWIP]{the!Proverbs!Pro 09:010 (2)}\index[AWIP]{beginning!Proverbs!Pro 09:010}\index[AWIP]{of!Proverbs!Pro 09:010 (2)}\index[AWIP]{wisdom!Proverbs!Pro 09:010}\index[AWIP]{and!Proverbs!Pro 09:010}\index[AWIP]{the!Proverbs!Pro 09:010 (3)}\index[AWIP]{knowledge!Proverbs!Pro 09:010}\index[AWIP]{of!Proverbs!Pro 09:010 (3)}\index[AWIP]{the!Proverbs!Pro 09:010 (4)}\index[AWIP]{holy!Proverbs!Pro 09:010}\index[AWIP]{\emph{is}!Proverbs!Pro 09:010 (2)}\index[AWIP]{understanding!Proverbs!Pro 09:010}\index[NWIV]{18!Proverbs!Pro 09:010}\index[PNIP]{LORD!Proverbs!Pro 09:010}\footnote{Now Bullinger is in trouble. He has just stated that this proverb is out of date and represents a primitive error (see remarks under 8:13); yet, now it pops up in the same book with 8:13 and not a whole chapter from it! Bullinger wisely ducks the verse, makes no comment, and hopes the reader of the Companion Bible will not check on things too closely. “The knowledge of the holy is understanding” (vs. 10). Putting everything together (vss. 1–10), it would appear that wisdom and understanding are to be obtained in the following ways  \cite{Ruckman1972Proverbs}:
\begin{compactenum}
   \item Begin by fearing God.
   \item Accept His invitation to a free banquet.
   \item Live right (vs. 9) and study soulwinning.
   \item Learn about the things of God (vs. 10).
\item Receive instruction from the Divine Wisdom (vs. 9).
\end{compactenum}
How this lines up with the curriculum found in elementary schools, junior high schools, high schools, colleges, and universities is  a little hard to say. The Catholic private schools fare no better, teaching little ones all the pagan blasphemous nonsense accumulated through fifteen centuries: holy water, holy rosary, holy Mary, holy mass, holy candles, holy bells, holy mackerel (yes, they bless the fishing fleets!), etc., ad nauseam. What does this have to do with “knowledge of the holy” (vs. 10)? It only shows that Catholic priests and nuns are devoid of “understanding” in the Bible sense, for “knowledge of the holy IS understanding” (vs. 10). The private schools of the Protestants have little to contribute, for the requirement for wisdom and understanding was to be taught by God (see 1:7–8, 1:24–33, 2:1, 2:5–6, 3:5, 4:4, 8:8–9, etc.), not by philosophy, the rudiments of this world, or the traditions of men (Col. 2:8). The standard theological curriculum (including Semitic languages, Greek, Systematic Theology, Biblical Theology, Hermeneutics, Textual Criticism, Manuscript Evidence, etc.) in ANY conservative school is nothing but two to six years of learning how to change the word of God. Is this the instruction of “divine wisdom?” (vs. 9). Did Jesus Christ teach you to correct the AV 1611, by which He blessed the church in the greatest period of evangelism and missionary activity the world ever saw? Or did you get it from men (John 18:34)?  \cite{Ruckman1972Proverbs}  }
[11] \textcolor[rgb]{0.00,0.00,1.00}{For by me thy days shall be multiplied, and the years of thy life shall be increased.}\index[AWIP]{For!Proverbs!Pro 09:011}\index[AWIP]{by!Proverbs!Pro 09:011}\index[AWIP]{me!Proverbs!Pro 09:011}\index[AWIP]{thy!Proverbs!Pro 09:011}\index[AWIP]{days!Proverbs!Pro 09:011}\index[AWIP]{shall!Proverbs!Pro 09:011}\index[AWIP]{be!Proverbs!Pro 09:011}\index[AWIP]{multiplied!Proverbs!Pro 09:011}\index[AWIP]{and!Proverbs!Pro 09:011}\index[AWIP]{the!Proverbs!Pro 09:011}\index[AWIP]{years!Proverbs!Pro 09:011}\index[AWIP]{of!Proverbs!Pro 09:011}\index[AWIP]{thy!Proverbs!Pro 09:011 (2)}\index[AWIP]{life!Proverbs!Pro 09:011}\index[AWIP]{shall!Proverbs!Pro 09:011 (2)}\index[AWIP]{be!Proverbs!Pro 09:011 (2)}\index[AWIP]{increased!Proverbs!Pro 09:011}\index[NWIV]{17!Proverbs!Pro 09:011}
[12] \textcolor[rgb]{0.00,0.00,1.00}{If thou be wise, thou shalt be wise for thyself: but \emph{if} thou scornest, thou alone shalt bear \emph{it}.}\index[AWIP]{If!Proverbs!Pro 09:012}\index[AWIP]{thou!Proverbs!Pro 09:012}\index[AWIP]{be!Proverbs!Pro 09:012}\index[AWIP]{wise!Proverbs!Pro 09:012}\index[AWIP]{thou!Proverbs!Pro 09:012 (2)}\index[AWIP]{shalt!Proverbs!Pro 09:012}\index[AWIP]{be!Proverbs!Pro 09:012 (2)}\index[AWIP]{wise!Proverbs!Pro 09:012 (2)}\index[AWIP]{for!Proverbs!Pro 09:012}\index[AWIP]{thyself!Proverbs!Pro 09:012}\index[AWIP]{but!Proverbs!Pro 09:012}\index[AWIP]{\emph{if}!Proverbs!Pro 09:012}\index[AWIP]{thou!Proverbs!Pro 09:012 (3)}\index[AWIP]{scornest!Proverbs!Pro 09:012}\index[AWIP]{thou!Proverbs!Pro 09:012 (4)}\index[AWIP]{alone!Proverbs!Pro 09:012}\index[AWIP]{shalt!Proverbs!Pro 09:012 (2)}\index[AWIP]{bear!Proverbs!Pro 09:012}\index[AWIP]{\emph{it}!Proverbs!Pro 09:012}\index[NWIV]{19!Proverbs!Pro 09:012}\footnote{[RUCKMAN] The LXX, unable to understand the proverb, alters it completely and adds eight words found nowhere except in the imagination of Westcott and Hort. Job 22:2 (in another “wisdom book”) explains the text nicely without the benefit of the bungling bamboozlers of Alexandria. The thought is: whether you are a wise man or a scorner (second clause in vs. 12), you will have to live with it and bear the consequences of it. In the prison house of flesh a man has to live with whatever he builds. The phony “Septuagint,” running true to form, now fabricates nearly fifty words and adds them to the text, without one peep from Kenneth Wuest, Billy Graham, Bernard Ranim, or Harold Ockenga. All of these accept the theory that Christians in the first century reverenced the Septuagint! I will not waste paper quoting the addition. It is a typical Clementine Philonic Origenic “gasser.” One can almost see the Greek gnostic in a passive trance (trying to “get in the spirit”), and mechanically writing out the leadings of his depraved imagination and then staring at it in awe and reverence as though God had “used him.” \cite{Ruckman1972Proverbs}  }
[13] \textcolor[rgb]{0.00,0.00,1.00}{A foolish woman \emph{is} clamorous: \emph{she} \emph{is} simple, and knoweth nothing.}\index[AWIP]{A!Proverbs!Pro 09:013}\index[AWIP]{foolish!Proverbs!Pro 09:013}\index[AWIP]{woman!Proverbs!Pro 09:013}\index[AWIP]{\emph{is}!Proverbs!Pro 09:013}\index[AWIP]{clamorous!Proverbs!Pro 09:013}\index[AWIP]{\emph{she}!Proverbs!Pro 09:013}\index[AWIP]{\emph{is}!Proverbs!Pro 09:013 (2)}\index[AWIP]{simple!Proverbs!Pro 09:013}\index[AWIP]{and!Proverbs!Pro 09:013}\index[AWIP]{knoweth!Proverbs!Pro 09:013}\index[AWIP]{nothing!Proverbs!Pro 09:013}\index[NWIV]{11!Proverbs!Pro 09:013}\footnote{[RUCKMAN] No boy who ever rode the alleys of the towns in Indiana, Illinois, Iowa, or Kansas could possibly miss the proverb if it had read, ``Stolen apples are sweet,'' and no cracker or red neck or ridge runner (or briar hopper) in the fields of Alabama, Mississippi, Tennessee, or Kentucky could misinterpret the passage if it had read “Stolen melons are sweet.” A melon stolen from a field and rushed to the river (in a blast of bacon rind from a shot gun), then floated down stream for fifteen minutes, brought out, kicked open on the bank, and eaten, has a flavor (alas!) which could never be found in the Piggly Wiggly or the A\&P! The best bread I ever ate was stolen from the galley of the Cape Mendecino (1945) en route to the Philippines. We got it while it was hot, ducked three cooks and bakers, two M.P.’s and several shipmates, and finally hid in a huddle by one of the air vents on the forward deck. There (at 2 a.m.) we ate it without butter, and I must confess that it had a flavor (alas!) that I have never tasted since. Such is human nature. On another occasion I became so hard up for the “Bread of Life” that I stole a Bible out of a rooming house on West Garden Street, Pensacola, Florida, (1949), and that was one loaf that was bitter to swallow and bitter to the taste, but soon became “sweeter also than honey” (Psa. 19:10). I must confess that sweeter than stolen bread is the sweetness of  the Living Word when it is purchased lawfully and comes into the hand with that smell that only a new Bible has! If you are a lover of the Book, you know exactly what I mean.\cite{Ruckman1972Proverbs}  } \footnote{[RUCKMAN] The woman may have been Job’s wife! (Job 2:10). Whoever she is, she is anything but a “blessing,” and she is as far removed from the “prudent wife” of Proverbs 19:14 as an alley cat is removed from a meadow lark. (Again the corrupt “Septuagint,” written A.D. 300 or later, adds nine words which have nothing to do with anything.) This woman has a house on Knob Hill (vs. 14) which often becomes “mortgage hill” to those who try to climb it too quickly. She does not roam the streets as the harlot of chapter 7, but she sits comfortably in a recliner (or more likely on a lawn chair in shorts) and invites the mailman, newspaper boy, ice man, gas man, etc., to stay awhile. Her “line” is quite simple and quite ancient (vs. 17). There is much truth in it. Eve could not resist this proposition. The “forbidden” has always been more attractive than the lawful and legal. Human nature is that way and has been that way ever since Adam and Eve fell. No amount of satisfaction can ever cure the incurable covetousness of the flesh (Eccl. 6:7; Prov. 27:20). Hence, it is the utmost folly to assume that “sex education,” “drink education,” and “dope education” will remedy the problem. The reasoning of the modern international socialist is simply this: “If we remove the bans and taboos off the  ‘forbidden,’ they will no longer be attractive, nor attract, and therefore human nature will level off to sensible living.” This would be fine if it were true. However, it is not true. Money grabbers are not satisfied with money, land grabbers are not satisfied with land, flesh addicts are not satisfied with the flesh pots of Egypt, and drunkards are not satisfied with a limit of liquor. The flesh is insatiable (“incurably wicked” is the Hebrew word, anash), and you can pamper it, prime it, primp it, culture it, cultivate it, crucify it, exercise it, examine it, display it, or develop it, and it is still FLESH (John 3:3).  \cite{Ruckman1972Proverbs}  }
[14] \textcolor[rgb]{0.00,0.00,1.00}{For she sitteth at the door of her house, on a seat in the high places of the city,}\index[AWIP]{For!Proverbs!Pro 09:014}\index[AWIP]{she!Proverbs!Pro 09:014}\index[AWIP]{sitteth!Proverbs!Pro 09:014}\index[AWIP]{at!Proverbs!Pro 09:014}\index[AWIP]{the!Proverbs!Pro 09:014}\index[AWIP]{door!Proverbs!Pro 09:014}\index[AWIP]{of!Proverbs!Pro 09:014}\index[AWIP]{her!Proverbs!Pro 09:014}\index[AWIP]{house!Proverbs!Pro 09:014}\index[AWIP]{on!Proverbs!Pro 09:014}\index[AWIP]{a!Proverbs!Pro 09:014}\index[AWIP]{seat!Proverbs!Pro 09:014}\index[AWIP]{in!Proverbs!Pro 09:014}\index[AWIP]{the!Proverbs!Pro 09:014 (2)}\index[AWIP]{high!Proverbs!Pro 09:014}\index[AWIP]{places!Proverbs!Pro 09:014}\index[AWIP]{of!Proverbs!Pro 09:014 (2)}\index[AWIP]{the!Proverbs!Pro 09:014 (3)}\index[AWIP]{city!Proverbs!Pro 09:014}\index[NWIV]{19!Proverbs!Pro 09:014}
[15] \textcolor[rgb]{0.00,0.00,1.00}{To call passengers who go right on their ways:}\index[AWIP]{To!Proverbs!Pro 09:015}\index[AWIP]{call!Proverbs!Pro 09:015}\index[AWIP]{passengers!Proverbs!Pro 09:015}\index[AWIP]{who!Proverbs!Pro 09:015}\index[AWIP]{go!Proverbs!Pro 09:015}\index[AWIP]{right!Proverbs!Pro 09:015}\index[AWIP]{on!Proverbs!Pro 09:015}\index[AWIP]{their!Proverbs!Pro 09:015}\index[AWIP]{ways!Proverbs!Pro 09:015}\index[NWIV]{9!Proverbs!Pro 09:015}
[16] \textcolor[rgb]{0.00,0.00,1.00}{Whoso \emph{is} simple, let him turn in hither: and \emph{as} \emph{for} him that wanteth understanding, she saith to him,}\index[AWIP]{Whoso!Proverbs!Pro 09:016}\index[AWIP]{\emph{is}!Proverbs!Pro 09:016}\index[AWIP]{simple!Proverbs!Pro 09:016}\index[AWIP]{let!Proverbs!Pro 09:016}\index[AWIP]{him!Proverbs!Pro 09:016}\index[AWIP]{turn!Proverbs!Pro 09:016}\index[AWIP]{in!Proverbs!Pro 09:016}\index[AWIP]{hither!Proverbs!Pro 09:016}\index[AWIP]{and!Proverbs!Pro 09:016}\index[AWIP]{\emph{as}!Proverbs!Pro 09:016}\index[AWIP]{\emph{for}!Proverbs!Pro 09:016}\index[AWIP]{him!Proverbs!Pro 09:016 (2)}\index[AWIP]{that!Proverbs!Pro 09:016}\index[AWIP]{wanteth!Proverbs!Pro 09:016}\index[AWIP]{understanding!Proverbs!Pro 09:016}\index[AWIP]{she!Proverbs!Pro 09:016}\index[AWIP]{saith!Proverbs!Pro 09:016}\index[AWIP]{to!Proverbs!Pro 09:016}\index[AWIP]{him!Proverbs!Pro 09:016 (3)}\index[NWIV]{19!Proverbs!Pro 09:016}
[17] \textcolor[rgb]{0.00,0.00,1.00}{Stolen waters are sweet, and bread \emph{eaten} in secret is pleasant.}\index[AWIP]{Stolen!Proverbs!Pro 09:017}\index[AWIP]{waters!Proverbs!Pro 09:017}\index[AWIP]{are!Proverbs!Pro 09:017}\index[AWIP]{sweet!Proverbs!Pro 09:017}\index[AWIP]{and!Proverbs!Pro 09:017}\index[AWIP]{bread!Proverbs!Pro 09:017}\index[AWIP]{\emph{eaten}!Proverbs!Pro 09:017}\index[AWIP]{in!Proverbs!Pro 09:017}\index[AWIP]{secret!Proverbs!Pro 09:017}\index[AWIP]{is!Proverbs!Pro 09:017}\index[AWIP]{pleasant!Proverbs!Pro 09:017}\index[NWIV]{11!Proverbs!Pro 09:017}\footnote{[RUCKMAN] No boy who ever rode the alleys of the towns in Indiana, Illinois, Iowa, or Kansas could possibly miss the proverb if it had read, ``Stolen apples are sweet,'' and no cracker or red neck or ridge runner (or briar hopper) in the fields of Alabama, Mississippi, Tennessee, or Kentucky could misinterpret the passage if it had read “Stolen melons are sweet.” A melon stolen from a field and rushed to the river (in a blast of bacon rind from a shot gun), then floated down stream for fifteen minutes, brought out, kicked open on the bank, and eaten, has a flavor (alas!) which could never be found in the Piggly Wiggly or the A\&P! The best bread I ever ate was stolen from the galley of the Cape Mendecino (1945) en route to the Philippines. We got it while it was hot, ducked three cooks and bakers, two M.P.’s and several shipmates, and finally hid in a huddle by one of the air vents on the forward deck. There (at 2 a.m.) we ate it without butter, and I must confess that it had a flavor (alas!) that I have never tasted since. Such is human nature. On another occasion I became so hard up for the “Bread of Life” that I stole a Bible out of a rooming house on West Garden Street, Pensacola, Florida, (1949), and that was one loaf that was bitter to swallow and bitter to the taste, but soon became “sweeter also than honey” (Psa. 19:10). I must confess that sweeter than stolen bread is the sweetness of  the Living Word when it is purchased lawfully and comes into the hand with that smell that only a new Bible has! If you are a lover of the Book, you know exactly what I mean.\cite{Ruckman1972Proverbs}  }
[18] \textcolor[rgb]{0.00,0.00,1.00}{But he knoweth not that the dead \emph{are} there; \emph{and} \emph{that} her guests \emph{are} in the depths of hell.}\index[AWIP]{But!Proverbs!Pro 09:018}\index[AWIP]{he!Proverbs!Pro 09:018}\index[AWIP]{knoweth!Proverbs!Pro 09:018}\index[AWIP]{not!Proverbs!Pro 09:018}\index[AWIP]{that!Proverbs!Pro 09:018}\index[AWIP]{the!Proverbs!Pro 09:018}\index[AWIP]{dead!Proverbs!Pro 09:018}\index[AWIP]{\emph{are}!Proverbs!Pro 09:018}\index[AWIP]{there!Proverbs!Pro 09:018}\index[AWIP]{\emph{and}!Proverbs!Pro 09:018}\index[AWIP]{\emph{that}!Proverbs!Pro 09:018}\index[AWIP]{her!Proverbs!Pro 09:018}\index[AWIP]{guests!Proverbs!Pro 09:018}\index[AWIP]{\emph{are}!Proverbs!Pro 09:018 (2)}\index[AWIP]{in!Proverbs!Pro 09:018}\index[AWIP]{the!Proverbs!Pro 09:018 (2)}\index[AWIP]{depths!Proverbs!Pro 09:018}\index[AWIP]{of!Proverbs!Pro 09:018}\index[AWIP]{hell!Proverbs!Pro 09:018}\index[NWIV]{19!Proverbs!Pro 09:018}\footnote{[RUCKMAN] Many a ``deep sermon'' in a modernistic church is not ``deep'' at all; it is muddy, and that is why you cannot see the bottom. Under the mud is the bloody spike. The word for ``dead'' (in vs. 18) is the word for ``giants'' in Genesis 6:4 (Rephaim); consequently, the scribe of the Septuagint gets all carried away with himself and makes the word ``guests'' in the second clause to read ``giants'' (which it cannot read!). Then, carried away in the mighty gust of self inflated helium, he adds nearly fifty words to the verse which are not worth reproducing. Such is the Greek text of the manuscripts used for the ASV (1901) and the NEB (1970) and the NASV (1960)! The text should read as it stands in the AV 1611. Variations are interesting but not particularly significant. The passage may have reference to the giants, but to limit it to this class is to ignore the plain statements of verse 16: ``Whoso is simple...him that wanteth understanding.''\cite{Ruckman1972Proverbs}  }






